\documentclass[12pt, a4paper, english]{report}
\usepackage[utf8]{inputenc}
\usepackage{makecell}
\usepackage{fourier}
\usepackage{caption}
\usepackage{amssymb}
\usepackage[linesnumbered]{algorithm2e} %,lined,boxed,commentsnumbered
\usepackage{subcaption}
\usepackage{epigraph}
\usepackage{tikz}
\usepackage{babel}
\usepackage{amsfonts}
\usepackage{geometry}
\usepackage{hyperref}
\usepackage{csquotes}
\usepackage{underscore}
\usepackage{amsmath}
\usepackage{biblatex}
\usepackage[rightcaption]{sidecap}
\usepackage{graphicx}
\usepackage{tabularx}
\graphicspath{ {img/} }

\author{Niccolò Simonato}
\title{Cryptography: notes}

\begin{document}

\maketitle

\tableofcontents

\chapter{Elementary operations}
\section{Notation}
\begin{itemize}
    \item Let $b$ be a numeric base.
    \item Let $n$ be a number in $N$.
    \item Length of a number: $l_{b}(n)$, $k$. It's equal to $log(n)$.
    \item $(a,b)$ is the Maximum Common Divisor of $a,b$.
    \item Let $n \in N$: $n = (d_{k-1}, d_{k-2}, \dots, d_{1}, d_{0})$\footnote{$d_{k-1} \neq 0$}.
    \item $\varphi(n)$: the number of elements $a$ in $[1,n]$ such that $(a,n) = 1$.
    \item $\equiv_{p}$ is the equivalence in base $p$. Ex.: $5 \equiv_{3} = 5 \bmod 3 = 2 $.
    \item Let $\mathbb{Z}_{n}[X]$ be the set of polynomials in $X$ with coefficients in $Z_{n}$.
\end{itemize}


\section{Classification of the algorithms' complexity}
In order to better identify the classes of complexity of the algorithms, the following 3 classes are defined:
\begin{itemize}
    \item Polynomial time: $O(log^{\alpha}(n))$ bit operations, where $\alpha > 0$.
    \item Exponential time: $O(exp(c \cdot log(n)))$ bit operations, where $c > 0$.
    \item Sub-exponential time: $O(exp(c \cdot log(n))^{\alpha})$ bit operations, where $c > 0, \alpha \in ]0, 1[$.
\item
\end{itemize}


\section{Basic bit operations}
\subsection{Sum of 3 bits - 3-bit-sum}
Given $n_{1}, n_{2}$ their sum produces $n_{1} + n_{2}$ and their carry. \newline
Since $n_{1}, n_{2} \in [0,1]$, then this operation can be done in $O(1)$.

\subsection{Summation of 2 numbers}
Given $n_{1}, n_{2}$ their sum produces $n_{1} + n_{2}$. \newline
Since the sum is computed bit by bit, the 3-bit-sum is performed\\$max\{lenght(n_{1}), length(n_{2})\}$ times.\newline
Each time the carry on of the previous sum is added to the two digits. \newline
This operation has then complexity $O(max\{lenght(n_{1}), length(n_{2})\}) = O(max\{log(n_{1}), log(n_{2})\})$

\subsection{Summation of $n$ numbers}
The summation of $n$ numbers is simply the sum of two numbers, but performed $n - 1$ times. \newline
Let's assume that $\forall i \in [1,n]: M \geq a_{i}$. \newline
The complexity of this operation is then $O((n-1) \cdot log(M)) = O(n)$. \newline

\subsection{Product of 2 numbers}
If we consider the classic implementation of the binary multiplication, that is just a sequence of summations. \newline
\begin{itemize}
    \item The number of summations to execute is equal to the length of the smallest number, $O(log(n))$.
    \item The maximum cost of a single summation is $O(log(m))$.
    \item Then, $T(m \cdot n) = O(log(m) \cdot log(n))$, but, if we consider the worst case\footnote{two numbers that are equally large}, that becomes $O(log^{2}(m))$.
\end{itemize}

\subsection{Division of 2 numbers}
Let's consider the division of two numbers $m, n$. This operations consists in \\ finding two numbers $q, r$ such that $m = q \cdot n + r$. \newline
This is achieved by performing a succession of subtractions, until the ending condition $0 \leq r < n$ is reached. \newline
\begin{itemize}
    \item Let's consider that the number of steps of this algorithm is $O(log(q))$.
    \item Moreover, $q \leq m \therefore \#steps = O(log(m))$.
    \item It's assumed that the cost of the single subtraction is $O(log(n))$.
    \item Then, $T(\frac{m}{n}) = O(log(n) \cdot log(m))$.
\end{itemize}

\subsection{Production of $n$ numbers}
Let's assume that $j \in [1, s+1]$ and $M = max(m_{j})$. \newline
The cost of the operation $\prod_{j = 1}^{s+1} m_{j}$ is then $O(s^{2} \cdot log^{2}(M))$. This will now be considered our inductive hypothesis.\newline
Proof by induction, on $s$:
\begin{itemize}
    \item (1) Base case: $T(m_{1} \cdot m_{2}) = O(log(m_{1}) \cdot log(m_{2})) = O(k_{1} \cdot k_{2}) \leq c \cdot k_{M}^{2}$.
    \item (2) Base case: $T(m_{1} \cdot m_{2} \cdot m_{3}) = T(m_{1} \cdot m_{2}) + T((m_{1} \cdot m_{2}) \cdot m_{3})$ \\ $\leq c \cdot k_{M}^{2} + c \cdot k_{m_{1} \cdot m_{2}} + k_{m_{3}}$ \\ $\leq c \cdot k_{M}^{2} + c \cdot k_{M^{2}} + k_{M}$
    \item Inductive step: we assume the inductive hypothesis to be true up to $s$. Then,
    \\ $T(\prod_{j = 1}^{s+1} m_{j}) = T([\prod_{j = 1}^{s} m_{j}] \cdot m_{s+1})$
    \\ $\leq c \cdot \sum_{j=1}^{s} (j \cdot k_{M}^{2})$
    \\ $= c \cdot k_{M}^{2} \cdot \frac{s \cdot (s-1)}{2}$
    \\ $= O(k_{M}^{2} \cdot s^{2})$
    \\ $= O(s^{2} \cdot log^{2}(M))$
\end{itemize}
\subsubsection{Applications}
\begin{itemize}
    \item An analogous dimonstration can be used to prove that $T(\prod_{j = 1}^{s+1} m_{j} \bmod n) = O(s \cdot log^{2}(M))$
    \item This proof can be used to show that $T(m!) = O(m \cdot log^{2}(m))$.
\end{itemize}

\section{Optimizations of more complex operations}
\subsection{Powers \& Modular Powers}
Let's consider what follows: $a^n = a \cdot a \cdot a \cdot \dots \cdot a$, where $a$ is repeated $n$ times.
\subsubsection{Trivial implementation}
The most trivial implementation would consists in computing the product $\prod_{j = 1}^{n} a$. This would imply a cost of $O(n^{2} \cdot log^{2}(a)))$.\newline
What follows is a suggestion that could improve the cost of this operation.
\subsubsection{Square \& Multiply method for scalars, modular powers}
Each number in $\mathbb{Z}$ can be represented in a binary notation. \newline
Let's consider $n = (b_{k-1}, b_{k-2}, \dots, b_{0}) = \sum_{i=0}^{k-1} b_{i} \cdot 2^{i}$. \newline
It is clear that we can spare a lot of computational resources by just calculating the powers of 2 and summing the ones that have $b_{i} = 1$. The following algorithm explains the procedure in detail.
\RestyleAlgo{ruled}
\begin{algorithm}
\caption{The Square \& Multiply Method}\label{alg:SquareMultiply}
$P \gets 1$\;
$M \gets m$\;
$A \gets a \bmod n$\;\label{instr_3}
\While{$M > 0$}{
    $q \gets \lfloor \frac{M}{2} \rfloor$\;\label{instr_5}
    $r \gets M - s \cdot q$\;\label{instr_6}
    \If{$r = 1$}{
        $P \gets P \cdot A \bmod n$\;\label{instr_8}
    }
    $A \gets A^{2} \bmod n$\;\label{instr_10}
    $M \gets q$\;

}
\Return{P}
\end{algorithm}
Let's compute the complexity of this algorithm:
\begin{itemize}
    \item All of the assignments $X \gets Y$ are implemented in $O(log(Y))$.
    \item The cost of \ref{instr_3} is $O(log(a) \cdot log(n))$, because it ensures that $A \leq n$.
    \item Instructions \ref{instr_5} and \ref{instr_6} can be executed in $O(log(m))$.
    \item Instrucion \ref{instr_8} can be executed in $O(log^{2}(n))$.
    \item The cost of \ref{instr_10} is $O(log^{2}(n))$, because it ensures that $A \leq n$.
    \item The loop is executed $log(m)$ times.
    \item The total cost of this algorithm is then $O(log(n) \cdot log(a) + log(m) \cdot (log^{2}(n) + log(m)))$ \\ $= O(log^{2}(m) + log(m) \cdot log(n))$.
\end{itemize}
This algorithm can be easily converted for the computation of non-modular powers by applying the following changes:
\RestyleAlgo{ruled}
\begin{algorithm}
    $A \gets a \bmod n$ $\Longrightarrow$ $A \gets a$\;\label{instr_3a}
    $P \gets P \cdot A \bmod n$ $\Longrightarrow$ $P \gets P \cdot A$\;\label{instr_8a}
    $A \gets A^{2} \bmod n$ $\Longrightarrow$ $A \gets A^{2}$\;\label{instr_10a}
\end{algorithm}
\subsubsection{Square \& Multiply method for polynomials}
Let's consider $\Re = \frac{\mathbb{Z}_{n}[x]}{x^{r} - 1}$. The modular powers of the elements in this set can be computed by using a variation of the Square \& Multiply method. \newline
\begin{itemize}
    \item Assume that $f, g \in \Re$.
    \item Let $h(x) = f(x) \cdot g(x) = \sum_{j = 0}^{2r - 2} h_{j} \cdot x^{j}$.
    \item Where $h(j) = (\sum_{i=0}^{j} f_{i} \cdot g_{j - i} \bmod n) \bmod n$.
    \item Then, $T(h_{j}) = O(j \cdot log^{2}(n))$.
    \item Then, $T(h(x)) = O(\sum_{j = 0}^{log^{2}(n)}) = O(r^{2} \cdot log^{2}(n))$.
\end{itemize}
This result will be useful in the following computations. \newline
Let's now consider $\frac{h(x)}{x^{r} - 1}$. \newline
\begin{itemize}
    \item When $j > r - 1$, $h_{j} \cdot x^{j}$ does not take any part in the computations.
    \item When $j = r$, then, $\frac{h_{r} \cdot x^{r}}{x^{r} - 1 } = h_{r} + \frac{h_{r}}{x^{r} - 1}$ \\ Or, in other words: $h_{r} \cdot x^{r} \equiv_{x^{r} - 1} h_{r}$.
    \item In the other cases: $h_{r+i} \cdot x^{r+i} \equiv_{x^{r} - 1} h_{r-i} \cdot x^{r-i}$ for $1 \leq i \leq r - 2$.
\end{itemize}
%An image can be useful
Then, $h(x) \equiv_{(n, x^{r} - 1)} f(x) \cdot g(x) \equiv$ \\ $[\sum_{j=0}^{r - 2}((h_{j} + h_{r + j}) \bmod n) \cdot x^{j}] + h_{r-1} \cdot x^{r-1}$.\label{comp:partial_prod_pol} \newline
Therefore, $T(h(x) \bmod (n, x^{r} - 1)) = O(r^{2} \cdot log^{2}(n))$. \newline
Finally, we can analyze the complexity of the computation of the modular power $h(x)$ elevated to $n$. \newline
In order to optimize the use of the computational resources, we can use a variation of the Square \& Multiply method (See \ref{alg:SquareMultiply});
although, this time, the computation of the partial products will be conducted by using the previously explained procedure (See \ref{comp:partial_prod_pol}). \newline
The cost of this method would then be $T(\#Loops \cdot (h(x) \bmod (n, x^{r} - 1))) =$ \\ $O(log(n) \cdot r^{2} \cdot log(n)) = O(r^{2} \cdot log^{3}(n))$.

\subsection{Finding the $b$ representation of $n$ ($n_{b}$)}
Let's consider the cost in bit operations of the conversion of a number $n$ to a new base $b$. \newline
The algorithm used will be the classical: a succession of divisions by $b$. \newline
\begin{itemize}
    \item Let's consider $r_{i} \in \{0, 1, \dots, b-1\}$.
    \item Let $n_{b} = (r_{k+1}, r_{k}, \dots, r_{1}, r_{0})$.
    \item Then: \begin{itemize}
                    \item $n = q_{0} \cdot b + r_{0}$
                    \item $q_{0} = q_{1} \cdot b + r_{1}$
                    \item $\dots$
                    \item $q_{k} = 0 \cdot b + q_{k}$
                \end{itemize}
    \item Consider then that $q_{k} = r_{k+1}$
    \item And that $b^{k+2} > n > b^{k+1} \rightarrow (k+2) \cdot log(b) \leq log(n) \leq (k+1) \cdot log(b)$.
    \item $\therefore k = O(\frac{log(n)}{log(b)})$.
\end{itemize}
We can now proceed with the computation of the cost of this operation: \newline
$T(n_{b}) = T(\#Divisions \cdot q_{i} \bmod b) = O(\frac{log(n)}{log(b)} \cdot log(n) \cdot log(b)) = $ \\
$O(log^{2}(n))$ 
\subsection{Modular inverses}


\section{Reminders of Modular Arithmetic}
\subsection{Little Fermat's Theorem}\label{little_fermat_th}
\begin{itemize}
    \item Let $p$ be a prime number.
    \item Then, $a^{p-1} \equiv_{p} 1$.
\end{itemize}

\subsection{Euler-Fermat's Theorem}\label{euler_fermat_th}
\begin{itemize}
    \item Let $n \in \mathbb{Z}$ be a number.
    \item Then, $a^{\varphi(n)} \equiv_{n} 1$.
\end{itemize}



\section{Useful Facts}
\begin{itemize}
    \item The \href{https://gmplib.org/}{GMP library} is a free library for arbitrary precision arithmetic. It implements all the basic arithmetic operations with the maximum efficency possible.
\end{itemize}

\listofalgorithms

\end{document}
