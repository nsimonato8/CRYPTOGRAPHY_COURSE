\section{Three step protocol}
\begin{definition}[Three step protocol]
    Consider as follows:
    \begin{itemize}
        \item Let $k_{A}$ be the key of $A$, and $k_{B}$ be the key of $B$.
        \item Let $L_{A}$ be the lock of $A$, and $L_{B}$ be the lock of $B$.
        \item Assume that only the owner of the lock can open and close it.
        \item Assume that the two locks are independent and both unbreakable.
    \end{itemize}
    Then:
    \begin{itemize}
        \item[\textbf{Step 1 - User $A$:}] $A$ inserts the lock $L_{A}$ and sends the box to $B$. Now the box has only the key of $A$.
        \item[\textbf{Step 2 - User $B$:}] $B$ inserts the lock $L_{B}$ and sends the box to $B$. Now the box has both the locks.
        \item[\textbf{Step 3 - User $A$:}] $A$ removes the lock $L_{A}$ and sends the box to $B$. Now the box has only the key of $B$. $B$ can now receive securely the box that contains the secret, because he's the only one that can open the lock $L_{B}$.
    \end{itemize}
\end{definition}
This protocol can easily implemented by using the XOR operator as the enciphering function, as it satisfies the assumptions of this protocol.
