\section{Definitions of Number Theory}
\subsection{The cyclic group $\mathbb{Z}_{n}^{*}$}
\begin{definition}[The cyclic group $\mathbb{Z}_{n}^{*}$]
    The cyclic group $\mathbb{Z}_{n}^{*}$ is defined as
    \[
    \mathbb{Z}_{n}^{*} = \{a \in \mathbb{Z}_{n}: (a,n) = 1\}
    \]
    The \textbf{generator} of $\mathbb{Z}_{n}^{*}$ is a number in $\mathbb{Z}_{n}^{*}$ such that:
    \[
    \forall a \in \mathbb{Z}_{n}^{*}: \exists i: a \equiv_{m} g^{i}
    \]
\end{definition}
For this reason, $\mathbb{Z}_{n}^{*}$ is also referred as $< g >$.\newline
An interesting property is that only for $[1,2,4, \phi, \phi^{\alpha}, 2 \phi^{\alpha}]$, $\mathbb{Z}_{n}^{*}$ is cyclic, where $\phi$ is a prime number and $\phi^{\alpha}$ is a power of a prime number\ref{gauss_theorem}.

\subsection{Pseudoprime number}
\begin{definition}[Pseudoprime number]
    For an integer $a > 1$, if a composite integer $x$ divides $a^{x-1} - 1$, then $x$ is called a \textbf{Fermat pseudoprime} to base $a$.
    \[
    x \text{ is spsp(a) } \iff x|a^{x - 1} - 1
    \]
\end{definition}
In other words, a composite integer is a Fermat pseudoprime to base a if it successfully passes the Fermat primality test for the base $a$.\newline
$x$ is $spsp(a)$ is a predicate, that means that some $x$ is a \emph{strong pseudoprime number in base $a$}.

\subsection{Carmichael numbers}
\begin{definition}[Carmichael numbers]
    Let $n$ be a composite number. Then:
    \[b^{n} \equiv_{n} b \implies b \text{ is a \textbf{Carmichael number}}\]
\end{definition}

\subsection{B-smoothness}
\begin{definition}[B-smoothness]
    Let $n, B > 0 \in \mathbb{Z}$. Let also $p$ be a prime number. Then:
    \[(p|n \implies p \leq B) \iff n \text{ is a \textbf{B-smooth number}}\]
\end{definition}
That is, if all the prime divisors of $n$ are smaller than $B$.

\section{Reminders of Modular Arithmetic}
\subsection{Little Fermat's Theorem}\label{little_fermat_theorem}
\begin{theorem}[Little Fermat's Theorem]
    Consider what follows:
    \begin{itemize}
        \item Let $p$ be a prime number and $p \nmid a$.
        \item Then \[a^{p-1} \equiv_{p} 1\]
    \end{itemize}
\end{theorem}
\begin{proof}
    Consider what follows:
    \begin{itemize}
        \item Let's consider: \[A = \{na \bmod p: n \in [1, p-1]\}\]
        \item $A$ has all distinct elements due to the Bijection Lemma (Lemma~\ref{bijection_lemma}).
        \item Then \[A = \mathbb{Z}_{n}^{*}\]
        \item Then, \[\prod\limits_{n \in A} n \equiv_{p} \prod\limits_{n=1}^{p-1} na \implies (p-1)! \equiv_{p} (p-1)!a^{p-1}\]
        \item Therefore, $a^{p-1} \equiv_{p} 1$ for the Wilson's Theorem~\ref{wilson_th}.
    \end{itemize}
\end{proof}

\subsection{Wilson's Theorem}\label{wilson_th}
\begin{theorem}[Wilson's Theorem]
    Consider what follows:
    \begin{itemize}
        \item Let $n \in \mathbb{N} \setminus \{0,1\}$
        \item Then, \[(n-1)! \equiv_{n} -1\]
    \end{itemize}
\end{theorem}
\begin{proof}
    The proof is \textbf{by induction} on $n$:
    \begin{itemize}
        \item \textbf{Base case}: $n = 2$
        \[1! \equiv_{2} -1\]
        \item \textbf{Inductive step}:
        The theorem is assumed to be true up until $n - 1$. Let's consider the case of $n$:
        \begin{itemize}
            \item Consider the polynomial \[g(x) = (x-1)(x-2) \dots (x - (n-1))\]
            \item $g$ has degree $p-1$ and costant term $(p-1)!$. It's roots are in $[1,p-1]$.
            \item Consider \[h(x) = x^{p-1} - 1\] $h$ has also degree $p-1$ and leading term $x^{p-1}$.
            \item Let $f(x) = g(x) - h(x)$.
            \item Then, $f$ has degree at most $p - 2$ (since the leading terms cancel), and modulo $p$ also has the $n - 1$ roots $1, 2, ..., n - 1$.
            \item But Lagrange's theorem says it cannot have more than $p - 2$ roots.
            \item \[\therefore f \equiv_{n} 0\]
            \item Its costant term, \[(n-1)! + 1 \equiv_{n} 0 \iff (n-1)! \equiv_{n} -1\]
        \end{itemize}
    \end{itemize}
\end{proof}

\subsection{Euler-Fermat's Theorem}\label{euler_fermat_th}
\begin{theorem}[Euler-Fermat's Theorem]
    Consider what follows:
    \begin{itemize}
        \item Let $n \in \mathbb{Z}$ be a number.
        \item Then, \[a^{\varphi(n)} \equiv_{n} 1\]
    \end{itemize}
\end{theorem}

\subsection{Bézout's identity}
\begin{theorem}[Bezout's identity]
    Consider what follows:
    \begin{itemize}
        \item Let $a$ and $b$ be integers with greatest common divisor $d$.
        \item Then there exist integers $x$ and $y$ such that $ax + by = d$.
        \item Moreover, the integers of the form $az + bt$ are exactly the multiples of $d$.
    \end{itemize}
\end{theorem}

\subsection{Chinese Reminder's Theorem}
\begin{theorem}[Chinese Reminder's Theorem]
    Given a system of congruences:
    \begin{align*}
        x & \equiv_{m_{1}} a_{1} \\
        x & \equiv_{m_{2}} a_{2} \\
        &\dots \\
        x & \equiv_{m_{r}} a_{r}
    \end{align*}
    In which each module is prime with each others, that is:
    \[
    \forall i \neq j: (m_{i}, m_{j}) = 1
    \]
    then \textbf{there exists a simultaneous solution $x$ to all of the congruences}, and \textbf{any two solutions are congruent to one another modulo}.
    \[M = m_{1}m_{2}\dots m_{r}\]
\end{theorem}

\begin{proof}
    Consider what follows:
    \begin{itemize}
        \item Suppose that $x'$ and $x"$ are two solutions.
        \item Let $x = x' - x''$.
        \item Then x must be congruent to $0$ modulo each $m_{i}$ and hence modulo $M$.
        \item Let $M_{i} = \frac{M}{m_{i}}$, to be the product of all of the moduli except for the i-th.
        \item Then $GCD(m_{i}, M_{i}) = 1$ and therefore $\exists N_{i}: M_{i}N_{i} \equiv_{m_{i}} 1$.
        \item Set $x = \sum_{i}a_{i}M_{i}N_{i}$;
        \item Then, for each $i$ we see that the terms in the sum other than the i-th term are all divisible by $m_{i}$, because $m_{i} | M_{j}$ when $i \neq j$.
        \item Thus, for each $i$ we have: $x \equiv_{m_{i}} a_{i}M_{i}N_{i} \equiv_{m_{i}} a_{i}$,
    \end{itemize}
\end{proof}

\subsection{Bijection of a modular function Lemma}
\begin{lemma}[Bijection of a modular function Lemma]\label{bijection_lemma}
    Consider:
    \begin{itemize}
        \item \[a \in \mathbb{Z}_{n}^{*}\]
        \item \[f: \mathbb{Z}_{n} \rightarrow \mathbb{Z}_{n}\]
    \end{itemize}
    Then, $f$ is a bijection.\newline
\end{lemma}
\begin{proof}
    Part 1: $f$ is injective.
    \begin{itemize}
        \item Assume $f(x_{1}) = f(x_{2}) \in \mathbb{Z}_{n} \iff ax_{1} \equiv_{n} ax_{2}$.
        \item Then $\exists k \in \mathbb{Z}: ax_{1} - ax_{2} = kn$.
        \item Therefore,
        \begin{align*}
            a(x_{1} - x_{2}) = kn & \iff a^{-1}a(x_{1} - x_{2}) = a^{-1}kn\\
            \iff (x_{1} - x_{2}) = a^{-1}kn & \iff x_{1} - x_{2} \equiv_{m} 0.
        \end{align*}
        \item $\therefore x_{1} \equiv_{m} x_{2}$, so $f$ is an injection.
    \end{itemize}
    Part 2: $f$ is surjective.
    \begin{itemize}
        \item Let $b \in \mathbb{Z}_{n}$.
        \item Let $\overline{x} = a^{-1}b \in \mathbb{Z}_{n}$.
        \item Then, $f(\overline{x}) \equiv_{m} a \cdot \overline{x}$\\
        $\equiv_{m} a \cdot a^{-1} \cdot b \equiv_{m} b$.
        \item Therefore, $f$ is surjective.
    \end{itemize}
    Since $f$ is injective $\land$ $f$ is surjective, then $f$ is bijective.
\end{proof}

\subsection{Euler's $\varphi$ function Lemmas}
\begin{lemma}[Sum of the prime divisors' $\varphi$-function]
    \[\forall n \in \mathbb{N} \backslash \{0\}: \sum_{\frac{n}{d}} \varphi(d) = n\]
    Where $\frac{n}{d}$ is the set of prime divisors of $n$.\newline
\end{lemma}
\begin{proof}
    The proof is by \textbf{contradiction}:
    \begin{itemize}
        \item Let $B$ be \[\{\frac{h}{n}: h \in \mathbb{Z}_{n} \land n \in \mathbb{N} \}\]
        \item Therefore, \[B = \cup_{\frac{n}{d}}\{a \in \{1, \dots, n\} \land (a,d) = 1\}\]
        \item Then, \[n = |B| = \sum_{\frac{d}{n}} \varphi(d)\]
        \item Consider \[(a_{1}, d_{1}) = 1 \land (a_{2}, d_{2}) = 1 \text{, where} d_{1}|n \land d_{2}|n\]
        \item Then, \[\frac{a_{1}}{d_{1}} = \frac{a_{2}}{d_{2}} \iff a_{1}d_{2} = a_{2}d_{1}\]
        \item Then, \[d_{1}|a_{1}d_{2} \implies d_{1}|d_{2} \land
        d_{2}|a_{2}d_{1} \implies d_{2}|d_{1}\]
        \item Therefore,  $d_{1} = d_{2}$. This is clearly a \emph{contradiction}, because in $B$ each divisor is counted once.
    \end{itemize}
\end{proof}

\begin{lemma}[Number of divisors of a prime number's power]
    Let $p \in \mathbb{N}$  be a prime number, and $\alpha \in \mathbb{N}$.\\
    Then, \[\varphi(p^{\alpha}) = p^{\alpha - 1}(p - 1)\]
\end{lemma}
\begin{proof}
    The proof is by \textbf{induction} on $\alpha$.\newline
    \textbf{Case base:} $\alpha = 1$
    \[p = \sum_{\frac{d}{p}} \varphi(d) = \varphi(1) + \varphi(p) = 1 + \varphi(p) \implies \varphi(p) = p - 1\]
    \textbf{Case base:} $\alpha = 2$
    \begin{align*}
        p^{2} & = \sum_{\frac{d}{p^{2}}} \varphi(d) = \varphi(1) + \varphi(p) + \varphi(p^{2})\\
        & = 1 + p - 1 + \varphi(p^{2}) \implies \varphi(p^{2}) = p^{2} - p - 1 + 1\\
        & = p \cdot (p - 1) = p^{\alpha - 1}(p - 1)
    \end{align*}
    \textbf{Inductive Step:} we can now assume that $\varphi(p^{\alpha}) = p^{\alpha - 1}(p - 1)$ up until $\alpha - 1$. We'll proceed now to demonstrate that this is also valid for each $\alpha$.
    \begin{align*}
        p^{\alpha} & = \sum_{\frac{d}{p^{\alpha}}} \varphi(d)\\
        & = \sum_{i = 0}^{\alpha - 1} \varphi(p^{i}) + \varphi(p^{\alpha})\\
        & \implies \varphi(p^{\alpha}) = p^{\alpha} - p^{\alpha - 1} = p^{\alpha - 1}(p - 1).
    \end{align*}
    \newline
\end{proof}

\subsection{Multiplicativity of the Euler's $\varphi$-function}
\begin{theorem}[Multiplicativity of the Euler's $\varphi$-function]
    Let's consider the Euler's function $\varphi(n) = |\mathbb{Z}_{n}^{*}$.\\
    Let's consider that $n = \prod_{i=1}^{r} p_{i}^{\alpha_{i}}$, where $p_{i}$ is a prime number, and $\alpha_{i} \in \mathbb{N}$.\\
    Then, $=$
\end{theorem}

\section{Theorems for Cryptography purposes}
\subsection{The Miller-Rabin Theorem}
\begin{theorem}[The Miller-Rabin Theorem]
    The \textbf{Miller-Rabin Theorem states that}:
    \begin{itemize}
        \item Let $p$ be a prime number.
        \item Let $a \in \mathbb{N}: (a, p) = 1$
        \item Let $p - 1 = 2^{s}d$, where $d$ is odd.
        \item Then:
        \begin{itemize}
            \item $a^{d} \equiv_{p} 1$ or
            \item $a^{2^{r}d} \equiv_{p} -1 \text{ for some r } \in \{0, 1, \dots, s-1\}$.
        \end{itemize}
    \end{itemize}
\end{theorem}
\begin{proof}
    The first part of the proof consists in showing that, if $n$ is a prime, then $\sqrt{1} \equiv_{n} \pm 1$:
    \begin{align*}
        &x^{2} \equiv_{n} 1 &\iff x^{2} - 1 \equiv_{n} 0\\
        \iff& (x + 1)(x - 1) \equiv_{n} 0 &\iff x = \pm 1 \bmod n
    \end{align*}
    The second part of the proof consist in showing  that, if $n$ is an odd prime, then it is $spsp(a)$.
    \begin{itemize}
        \item Let $n$ be prime;
        \item Let $a: (a,n) = 1$;
        \item Let $s,d: n - 1 = 2^s \cdot d$, where $s > 0$ and $d$ is odd.
        \item Then $a^{n-1} \equiv_{n} 1 \iff a^{2^s \cdot d} \equiv_{n} 1$, due to Fermat's Little Theorem.
        \item We can now consider $a^{2^r \cdot d}$ successive square roots, for $1 \leq r < s$.
        \item Therefore, starting from $r = s - 1$, we can compute $a^{2^r \cdot d} \bmod n$. If it's $-1$, then $n$ is $spsp(a)$. If it's $1$, we can repeat the computation for $r = s - 2$, and so on. Due to the previous part, there are no other possible cases.
    \end{itemize}
\end{proof}

\subsection{Miller's Theorem}
\begin{theorem}[Miller's Theorem]
    The Miller's Theorem states what follow:
    \begin{itemize}
        \item Let $n$ be a composite and odd number.
        \item Then, $n$ is $spsp(a)$ for at most $\frac{1}{4}$ of the $a_{i} \in \mathbb{Z}_{n}^{*}$
    \end{itemize}
\end{theorem}
The following lemma is a consequence of this theorem.
\begin{lemma}[Corollary of the Miller's Theorem]
    If $n$ is composite and odd, then \[\exists a \in \mathbb{Z}_{n}^{*}: a \leq b \text{, such that $n$ is not $spsp(a)$} \]
\end{lemma}

\subsection{Ankey-Montgomery-Bach Theorem}
\begin{theorem}[Ankey-Montgomery-Bach Theorem]
    The Ankey-Montgomery-Bach Theorem states that:
    \begin{itemize}
        \item If the GRH\footnote{Generalized Riemann Hypothesis} holds;
        \item If $n$ is composite and odd;
    \end{itemize}
    Then, \[\exists \in [2, 2 log^{2}(n)] \text{ such that $n$ is not $spsp(a)$}\]
\end{theorem}

\section{Euler's Product Theorem}
\begin{theorem}[Euler's Product Theorem]
    If $ \mathbb{R}e(s) > 1$, then \[\zeta(s) = \sum_{n = 1}^{+ \infty} \frac{1}{n^{2}} = \prod\limits_{p}(1 - \frac{1}{p^{2}})^{-1}\]
\end{theorem}

\section{Discrete Logarithm problem}
\begin{definition}[Discrete Logarithm problem]\label{discrete_logarithm_prob}
    Given $p, g, y$ where:
    \begin{itemize}
        \item $p$ is a large prime number;
        \item $g$ is the generator of $\mathbb{Z}_{p}^{*}$;
        \item $y \in \mathbb{Z}_{p}^{*}$.
    \end{itemize}
    Find $x \in \mathbb{Z}_{p-1}: g^{x} = y$
\end{definition}

\section{Gauss' Theorem}
\begin{definition}[Gauss' Theorem]\label{gauss_theorem}
    The theorem states as follows:
    \begin{itemize}
        \item $\mathbb{Z}_{n}^{*}$ is cyclic $\iff n = 1, 2, 4, p^{\alpha}, 2p^{\alpha}$
        \item Assume that we know $g$ such that $<g> = \mathbb{Z}_{p}^{*}$:
        therefore we can build $<g_{1}> = \mathbb{Z}_{p^{\alpha}}^{*}, <g_{2}> = \mathbb{Z}_{2p^{\alpha}}^{*}$ in polinomial time.
    \end{itemize}
\end{definition}
