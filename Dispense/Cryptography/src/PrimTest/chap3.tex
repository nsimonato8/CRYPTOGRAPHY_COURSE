\section{Miller-Rabin probabilistic primality algorithm}
\RestyleAlgo{ruled}
\begin{algorithm}
\KwData{$n \in \mathbb{N}$, an \emph{odd} number}
\KwResult{$r$}
\caption{Miller-Rabin primality test}\label{alg:miller_rabin_ptest}
Compute $s, d$ such that: $n - 1 = 2^{s} \cdot d$\;
Randomly choose $a \in \mathbb{Z}_{n}^{*}$\;
\If{$(a,n) > 1$}{
    \Return{$n$ is composite}\;
}
$b \gets a^{d} \bmod n$\;
\If{$b \equiv_{n} \pm 1$}{
    \Return{$n$ is prime or spsp}\;
}
$e \gets 0$\;
\While{$b \not\equiv_{n} \pm 1 \land e \leq s-2 $}{
    $b \gets b^{2} \bmod n$
}
\If{$b \not\equiv_{n} 1$}{
    \Return{$n$ is composite}\;
}
\Return{$n$ is prime or spsp}\;
\end{algorithm}
\subsection{Computational complexity of the Miller-Rabin test}
Testing the primality for a single value of $a$ has cost of $O(log^{3}(n))$ b.o.. Although, we could test for each value in $\mathbb{Z}_{n}^{*}$, and that would cost $O(\varphi(n) \cdot log^{3}(n))$ b.o..

\section{Primality Test in $\mathbb{P}$, AKS algorithm}
\subsection{Useful Lemmas}
\begin{lemma}[Newton's formula lemma]\label{newtown_form_lemma}
    This lemma states as follows:\newline
    $n$ is prime $\iff (x + b)^{n} \equiv_{n} x^{n} + b$.
\end{lemma}
\begin{proof}
    The proof is by identity:
    \begin{itemize}
        \item $(x + b)^{n} \equiv_{n} x^{n} + b \implies n$ is prime.
        \begin{itemize}
            \item By \emph{contradiction}:
            \begin{itemize}
                \item Assume that $n$ is composite.
                \item Then $\exists p | n$, where $p < n$ is a prime number.
                \item Consider \[\binom{n}{p} = \frac{n \cdot n-1 \cdot \dots \cdot n - p + 1}{p \cdot p - 1 \cdot \dots \cdot 1} > 1\]
                \item Assume that: \[p^{\alpha} || n\]
                \item Then \[p \nmid n\]
                \item Let \[N = \prod_{i = n-p+1}^{n-1} i\]
                \item Let \[M = \prod_{i = 1}^{p-1} i\]
                \item Then, \[\binom{n}{p} = \frac{p^{\alpha} \cdot N}{p \cdot M} = p^{\alpha - 1} \cdot \frac{N}{M}\]
                \item \[(N,p) = (M, p) = 1 \implies p^{\alpha - 1} | \binom{n}{p} \land p^{\alpha} \nmid \binom{n}{p}\]
                \item Therefore: \[p^{\alpha - 1} || \binom{n}{p} \text{ and } \binom{n}{p} \equiv_{p} 0\]
                \item But this is a \textbf{contradiction}, since $\binom{n}{p} \not\equiv_{p} 0$, therefore $n$ is prime.
            \end{itemize}

        \end{itemize}
        \item $n$ is prime $\implies (x + b)^{n} \equiv_{n} x^{n} + b$
        \begin{itemize}
            \item Consider that $p | \binom{p}{k}$, for $k < n$.
            \item Since $\binom{p}{k} \equiv_{p} 0$, then
            \[(x + b)^{p} = \sum_{k=0}^{p} \binom{p}{k} x^{p-k} \cdot b^{k} \equiv_{p} x^{p} + b^{p} \equiv_{p} b\]
        \end{itemize}
    \end{itemize}
\end{proof}

\begin{lemma}[Nair's Lemma]\label{nair_lemma}
    This lemma states as follows:
    \begin{itemize}
        \item Let $m \geq 7 \in \mathbb{Z}$
        \item Let $\operatorname{LCM}(x,y)$ be the Least Common Multiplier of $x$ and $y$.
        \item Let $n \leq m \in \mathbb{Z}$.
        \item Then: \[LCM(m,n) \geq 2^{m}\]
    \end{itemize}
\end{lemma}

\begin{lemma}[AKS Lemma]\label{aks_lemma}
    This lemma states as follows:
    \begin{itemize}
        \item Let $n \geq 4$
        \item Then: \[\exists r \leq \lceil log_{2}^{5}(n) \rceil \text{ such that } d = ord(n)_{\mathbb{Z}_n^*} > log_2^2(n)\]
    \end{itemize}
\end{lemma}
\begin{proof}
    The proof is by \emph{contradiction}:
    \begin{itemize}
        \item Let $n \geq 4 \Rightarrow \lceil log_{2}^{5}(n) \rceil \geq 32$.
        \item Let $V$ be $\lceil log_{2}^{5}(n) \rceil$.
        \item Let \[\Pi = n^{\lfloor log_{2}(V) \rfloor} \cdot \prod_{i=1}^{\lceil log_{2}^{2}(n) \rceil}(n^{i} - 1) \].
        \item Let $\nu$ be $\{s \in \{\, \dots, \nu\}: s \nmid \Pi\}$
        \item Assume by contradiction that$ \nu = 0 $:
        \begin{itemize}
            \item Then, by definition: $\forall s \in \nu: s \nmid \Pi$.
            \item Consider that $lcm\{1, \dots, V\} | \Pi$
            \item Consider that:
            \begin{align}
                \Pi \leq n^{\lfloor log_{2}(V) \rfloor} \cdot \prod_{i=1}^{\lceil log_{2}^{2}(n) \rceil} n^{i} & =\\
                n^{\lfloor log_{2}(V) \rfloor + \sum_{i=1}^{\lfloor log_{2}^{2}(n) \rfloor} i} & =  \\
                n^{\lfloor log_{2}(V) \rfloor + \frac{1}{2} \lfloor log_{2}^{2}(n) \rfloor \cdot (\lfloor log_{2}^{2}(n) \rfloor + 1)}
                & < \lfloor (log_{2}(n))^{4} \rfloor \\
                & = 2^{log_{2}(n) \cdot \lfloor (log_{2}(n))^{4} \rfloor} \\
                & < 2^{log_{2}^{5}(n)} \\
                & < 2^{V}
            \end{align}

            \item So, $lcm\{1, \dots, V\} | \Pi \implies lcm\{1, \dots, V\} \leq \Pi < 2^{V}$
            \item Since $V \geq 32$ and $lcm\{1, \dots, V\} \geq 2^{V}$ due to Lemma\ref{nair_lemma}, we have a \textbf{contradiction}.
            \item Therefore, $\nu \neq 0$
        \end{itemize}
        \item Let then $r$ be $\operatorname{min}(\nu) \geq 2$
        \item Assume that $q$ is a prime and $q | r$
        \begin{itemize}
            \item Consider also that $r | V$, since $r \leq V \implies q^{\alpha} | V \implies \alpha \leq \lfloor log_{2}(V) \rfloor$
        \end{itemize}
        \item Assume also that every $q | r$, also $q | n$.
        \item Then, $r = \prod_{q | r} q^{\alpha} | \prod_{q | r} q^{\lfloor log_{2}(V) \rfloor} | \prod_{q | n} q^{\lfloor log_{2}(V) \rfloor}$,  where $p$ is a prime number.
        \item Let $n$ be $\prod_{q | n} q^{\beta}$, where $\beta \geq 1$
        \item Then, we have $n^{\lfloor log_{2}(V) \rfloor} = \prod_{q | n} q^{\beta \cdot \lfloor log_{2}(V) \rfloor}$
        \item Before, we proved:
        \[r | \prod_{q | n} q^{\lfloor log_{2}(V) \rfloor} | n^{\lfloor log_{2}(V) \rfloor} | \Pi\]
        \item Therefore, $r | \Pi$, but $r \in \nu$, so $r \nmid \Pi$, that is a \textbf{Contradiction}.
        \item Then not every prime divisor of $n$ is a prime divisor of $r$.
        \item Consider that $\frac{r}{(r,n)} \in \nu \implies \frac{r}{(r,n)} \leq r = \operatorname{min}(\nu) \implies \frac{r}{(r,n)} = r \implies \frac{r}{(r,n)} = 1$
        \begin{itemize}
            \item By \emph{contradiction}:
            \item Assume that $\frac{r}{(r,n)} \not\in \nu$
            \item Then, $\frac{r}{(r,n)} | \Pi$
            \item Let $r = \prod_{p|r} p^{\alpha}$
            \item If $p | r \land p \nmid n \implies p | \frac{r}{(r,n)} \implies p^{\alpha} | \frac{r}{(r,n)}$
            \item But, if $\frac{r}{(r,n)} | \Pi \implies p^{\alpha} | \prod_{i=1}^{\dots} (n^{i} - 1) | \Pi \implies r | \Pi$, that is a \textbf{contradiction}. Therefore, $\frac{r}{(r,n)} \in \nu$
        \end{itemize}
        \item So, $\operatorname{ord}(n)_{\mathbb{Z}_{r}^{*}} > \lfloor log_{2}^{2}(n) \rfloor$
        \begin{itemize}
            \item By \emph{contradiction}:
            \item $\exists i \leq \lfloor log_{2}^{2}(n) \rfloor$ such that $n^{i} \equiv_{r} = 1$
            \item $\implies r | \prod_{i = 1}^{\lfloor log_{2}^{2}(n) \rfloor}(n^{i} - 1) | \Pi$, but this is a \textbf{contradiction}.
        \end{itemize}

    \end{itemize}
\end{proof}

\subsection{Agrawal - Kayal - Saxema Theorem}
\begin{theorem}[Agrawal - Kayal - Saxema Theorem]\label{aks_theorem}
    Let $n \geq 4 \in \mathbb{N}$, and let $0 < r < n$ such that $(n,r) = 1$ and $order(n) > (log_{2}(n))^{2}$. Then:
    \[
    n \text{is prime} \iff
         \begin{cases}
           n \text{is not a perfect power} \\
           \not\exists p \leq r \\
           (x + b)^{n} \equiv_{(n, x^{r} - 1)} x^{n} + b \text{\, for every} b \in \mathbb{N} \text{ s.t. } 1 \leq b \leq \sqrt{n} \cdot log_{2}(n)
         \end{cases}
    \]
\end{theorem}

\subsection{AKS algorithm}
\begin{algorithm}
    \KwData{$n \in \mathbb{N}$}
    \KwResult{$TRUE$ if $n$ is prime}
    \caption{AKS primality test pseudocode}\label{alg:aks_pscd_ptest}
    \If{$n = \alpha^{\beta} \text{, where } \alpha \text{, } \beta > 1 \in \mathbb{N}$}{\Return{$FALSE$}}\label{step_1_aks}
    $r \gets \argmin_{x} (x,n) = 1$\;
    $d \gets ord(n)_{\mathbb{Z}_{r}^{*}} > \lceil log_{2}^{2}(n) \rceil$\;\label{step_2_aks}
    \If{$\exists b \leq r : 1 < (b,n) < n$}{\Return{$FALSE$}}\label{step_3_aks}
    \If{$n \leq r$}{\Return{$TRUE$}}\label{step_4_aks}
    \If{$\exists b \in \mathbb{N}: 1 \leq b \leq \sqrt{r} \cdot log_{2}(n) \land (x + b)^{n} \not\equiv_{(x^{r} - 1, n)} x^{n} + b$}{\Return{$FALSE$}}\label{step_5_aks}
    \Return{$TRUE$}\;\label{step_6_aks}
\end{algorithm}

\subsubsection{Correctness}
\begin{theorem}[Correctness of the AKS algorithm]
    The AKS algorithm for the primality test of a given number $n$ is correct.\newline
    $n$ is prime $\iff$ the algorithm returns $TRUE$.
\end{theorem}
\begin{proof}
    The proof examines the execution case by case.\newline
    \begin{itemize}
        \item Let's assume that $n$ is prime:
        \begin{itemize}
            \item Then, the algorithm cannot stop at step \ref{step_1_aks} or at step \ref{step_3_aks}.
            \item By Lemma \ref{newtown_form_lemma}, $ (x + b)^{n} \equiv_{n} x^{n} + b \forall b \in \mathbb{Z} \therefore$ the algorithm cannot terminate at step \ref{step_5_aks}.
            \item Therefore, the algorithm can only terminate at step \ref{step_4_aks} or \ref{step_6_aks}, so: $n$ is prime $\implies$ the algorithm returns $TRU£$.
        \end{itemize}
        \item Let's assume that the algorithm returns $TRUE$:
        \begin{itemize}
            \item Then, the algorithm has terminated at step \ref{step_4_aks} or \ref{step_6_aks}.
            \item If the algorithm has terminated at step \ref{step_4_aks}, then $n \leq r$. Since we checked at \ref{step_3_aks} that $(b,n)$ is trivial $\forall b \leq r$, then $n$ has no trivial divisors, hence it's prime.
            \item If the algorithm has terminated at step \ref{step_6_aks}, let's consider that at \ref{step_1_aks} and at \ref{step_3_aks} we verified that condition $1$ and $3$ of the Theorem \ref{aks_theorem} hold, respectively.
            \item Then, it's verified that: the algorithm returns $TRUE \Rightarrow n$ is prime.
        \end{itemize}
        \item Therefore, is verified that $n$ is prime $\iff$ the algorithm returns $TRUE$.
    \end{itemize}
\end{proof}

\subsubsection{Complexity}
\begin{theorem}[Complexity of the AKS algorithm]
    The AKS algorithm sustains the following costs for each step:
    \begin{itemize}
        \item Step \ref{step_1_aks} (checking if $n = a^{b}$) costs $O((log(n))^{3 + \epsilon})$ b.o..
        \item Step \ref{step_2_aks} (picking $r$) has cost of $O((log(n))^{7 + \epsilon})$ b.o., in the worst case.
        \item Step \ref{step_3_aks} (computing $(b,n)$ multiple times) has cost of $O((log(n))^{7 + \epsilon})$ b.o., in the worst case.
        \item Step \ref{step_5_aks} (verifying that $(x + b)^{n} \not\equiv_{n} x^{n} + b$) has cost of $O(log(n) \cdot log(r))$ b.o.
        \item The total cost of the AKS algorithm is then $O(r^{5/2} log^{4}(n))$ b.o.
    \end{itemize}
\end{theorem}
