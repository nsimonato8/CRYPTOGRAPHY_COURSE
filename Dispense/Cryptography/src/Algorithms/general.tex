\section{Cryptosystems}
\subsection{Definitions}
\begin{definition}[Cryptographic transformation]
    A cryptographic transformation is defined as \textbf{any injective function} $\mathscr{f}$ such that:
    \[\mathscr{f}: \mathcal{M} \rightarrow \mathcal{C}\]
\end{definition}
\begin{definition}[Enciphering and Deciphering functions]
    An \textbf{Enciphering function} is a cryptographic transformation. Its inverse $\mathscr{f}^{-1}$ is called \textbf{Deciphering function} and is defined as:
    \[\mathscr{f}^{-1}: \mathcal{C} \rightarrow \mathcal{M}\]
\end{definition}
\begin{definition}[Cryptosystem]
    A \textbf{Cryptosystem} is a tuple $(\mathcal{M},\mathcal{C}, \mathcal{K}, \mathscr{f}_{k_{e}},\mathscr{f}_{k_{d}}^{-1})$, where:
    \begin{itemize}
        \item $\mathcal{M}$ is the space of clear-text messages;
        \item $\mathcal{C}$ is the space of ciphered messages;
        \item $\mathcal{K}$ is the space of the keys;
        \item $\mathscr{f}_{k_e}$ is the enciphering function with key $k_{e}$;
        \item $\mathscr{f}_{k_d}^{-1}$ is the deciphering function with key $k_{d}$.
    \end{itemize}
\end{definition}

\subsection{Different kinds of cryptosystems}
\subsubsection{Affine cryptosystems}
\begin{definition}[Affine cryptosystems]
    A cryptosystem is defined \textbf{Affine} when the following conditions are met:
    \begin{itemize}
        \item It's a block cipher with lenght $l \geq 1$;
        \item Its enciphering function depends on the key $k = (a,b)$ such that:\\
        $f_{k}: \mathbb{Z}_{N}^{l} \rightarrow \mathbb{Z}_{N}^{l}$ and \\
        $f_{k}(m) = A \cdot m + b \bmod N$
        \item Its deciphering function depends on the key $k = (a,b)$ such that:\\
        $f_{k}^{-1}: \mathbb{Z}_{N}^{l} \rightarrow \mathbb{Z}_{N}^{l}$ and \\
        $f_{k}^{-1}(c) = A^{-1} \cdot (c - b) \bmod N$
        \item $A$ is an invertible $l \times l$ matrix, $b$ is a vector in $\Sigma^{l}$.
    \end{itemize}
\end{definition}

\subsubsection{Perfect cryptosystems}
\begin{definition}[Perfect cryptosystems]
    A cryptosystem is called a \textbf{Perfect cryptosystem} if and only if:
    \[\mathbb{P}(M = m|C = c) = \mathbb{P}(M = m)\]
\end{definition}
This means that a perfect cryptosystem is perfect if and only if the cleartext message and the ciphered message are independent.\newline
Consider what follows:
\begin{itemize}
    \item Let $\mathbb{M}$ be the space of the cleartext messages, and $M$ the correspondant aleatory variable;
    \item Let $\mathbb{C}$ be the space of the ciphered messages, and $C$ the correspondant aleatory variable;
    \item Let $\mathbb{K}$ be the space of the keys, and $K$ the correspondant aleatory variable.
\end{itemize}
Then, we have the following situation:
\begin{itemize}
    \item $\mathbb{P}_{\mathbb{M}}(M = m)$ is the probability of having the cleartext message $m$.
    \item $\mathbb{P}_{\mathbb{K}}(K = k)$ is the probability of having the key $k$.
    \item $\mathbb{P}(C = c) = \sum_{k \in \mathbb{K}} \mathbb{P}_{\mathbb{K}}(K = k) \cdot \mathbb{P}_{\mathbb{M}}(M = f_{k}^{-1}(c))$ is the probability of having the enciphered message $c$.
    \item Therefore, the probability of $c$ being the ciphered text of the cleartext message $m$ is given by:\\
    $\mathbb{P}(C = c|M = m) = \sum_{k : } m = f_{k}^{-1}(c) \mathbb{P}_{\mathbb{K}}(K = k)$
\end{itemize}
The Shannon's theorem helps us to decide if a given cryptosystem is perfect:
\begin{theorem}[Shannon's theorem]
    Let $(\mathcal{M},\mathcal{C}, \mathcal{K}, \mathscr{f}_{k_{e}},\mathscr{f}_{k_{d}}^{-1})$ be a given cryptosystem.\newline Assume that $|\mathcal{M}| = |\mathcal{C}| = |\mathcal{K}|$ and $\mathbb{P}_{\mathbb{M}}(M = m) > 0 \forall m \in \mathcal{M}$. \newline
    That cryptosystem is perfect if and only if:
    \begin{itemize}
        \item $\forall k \in \mathcal{K}: \mathbb{P}_{\mathbb{K}}(K = k) = \frac{1}{|\mathcal{K}|}$
        \item $\forall m \in \mathcal{M}: \exists ! k \in \mathcal{K}: f_{k}(m) = c$
    \end{itemize}
\end{theorem}
\begin{proof}
    The proof proceeds by proving both the direction of the implication:
    \begin{itemize}
        \item The cryptosystem is perfect $\implies \forall k \in \mathcal{K}: \mathbb{P}_{\mathbb{K}}(K = k) = \frac{1}{|\mathcal{K}|} \land\forall m \in \mathcal{M}: \exists ! k \in \mathcal{K}: f_{k}(m) = c$:
        Let $c \in \mathcal{C}$ be fixed. Let $m \in \mathcal{M}$
        \noindent
        \begin{gather*}
            \mathbb{P}(m) > 0 \implies \mathbb{P}(m|c) > 0\\
            \text{Then, } \mathbb{P}(m) > 0 \implies \mathbb{P}(c|m) > 0\\ \text{Therefore, } \sum_{k \in \mathcal{K}}\mathbb{P}(k) > 0 \\
            \text{So, } \exists k \in \mathcal{K}: \mathcal{f}_{k}(m) = c\\
            \implies |\mathcal{C}| = |\{\mathcal{f}_{k}(m): k \in \mathcal{K}\}| \leq \mathcal{K}\\
            \text{By hypothesis, we have that } |\mathcal{K}| = |\mathcal{M}| = |\mathcal{C}|\\
            \implies |\{\mathcal{f}_{k}(m): k \in \mathcal{K}\}| = \mathcal{K}\\
            \implies \forall c \in \mathcal{C} \exists ! k \in \mathcal{K} : \mathcal{f}_{k}(m) = c\\
            \text{By Bayes' Theorem }\mathbb{P}(m|c) = \frac{\mathbb{P}(c|m) \cdot \mathbb{P}(m)}{\mathbb{P}(c)} = \frac{\mathbb{P}(k_{m}) \cdot \mathbb{P}(m)}{\mathbb{P}(c)}\\
            \text{By perfect secrecy }\mathbb{P}(m|c) = \mathbb{P}(m) > 0\\
            \implies 1 = \frac{\mathbb{P}(k_{m})}{\mathbb{P}(c)} \implies \mathbb{P}(c) = \mathbb{P}(k_{m})\\
            \text{Therefore, } \mathbb{P}(k_{m}) \text{ is constant, since } c \text{ is fixed.}\\
            \text{Therefore, } \mathbb{P}(k) \text{ is equally distributed.}
        \end{gather*}
        \item $\forall k \in \mathcal{K}: \mathbb{P}_{\mathbb{K}}(K = k) = \frac{1}{|\mathcal{K}|} \land\forall m \in \mathcal{M}: \exists ! k \in \mathcal{K}: f_{k}(m) = c \implies$ The cryptosystem is perfect.
        \begin{itemize}
            \item[Hypothesis 1]: $\forall k \in \mathcal{K}: \mathbb{P}_{\mathbb{K}}(K = k) = \frac{1}{|\mathcal{K}|}$
            \noindent
            \begin{gather*}
                \mathbb{P}(c) = \sum_{k \in \mathcal{K}: f_{k}(m) = m} \mathbb{P}(k) \cdot \mathbb{P}(f_{k}^{-1}(c)) = 1
            \end{gather*}
            \item[Hypothesis 2]: $\forall m \in \mathcal{M}: \exists ! k \in \mathcal{K}: f_{k}(m) = c$
            \noindent
            \begin{gather*}
                \mathbb{P}(c) = \frac{1}{|\mathcal{K}|}\sum_{m \in \mathcal{M}} \mathbb{P}(m) = \frac{1}{|\mathcal{K}|}\\
                \text{Since } \sum_{m \in \mathcal{M}} \mathbb{P}(m) = 1\\
                \text{Then, } \mathbb{P}(c) = \sum_{k \in \mathcal{K}: f_{k}^{-1}(c) = m} \mathbb{P}(k) = \mathbb{P}(k_{m,c}) = \frac{1}{|\mathcal{K}|}\\
                \text{So, }\mathbb{P}(c|m) = \mathbb{P}(c)\\
                \text{Therefore, the cryptosystem is perfect.}
            \end{gather*}
        \end{itemize}
    \end{itemize}
\end{proof}

\paragraph{Vernam's cipher aka One Time Pad}
This is currently the only known perfect cipher. \newline
\begin{definition}[One Time Pad]
    Consider what follows:
    \begin{itemize}
        \item Let $\mathcal{M} = \mathcal{C} = \mathcal{K} = \mathbb{Z}_{N}$
        \item The key $k$ is randomly generated and only the two ends of the communication have it. \textbf{The key has the same length of the message} and \textbf{is used exactly one time}, hence the name "One Time Pad".
        \item The sender sends the ciphered message $c = m \oplus k$ to the receiver and destroys the key.
        \item The receiver deciphers the message: $m = c \oplus k$ and destroys the key.
    \end{itemize}
\end{definition}
It's important that th keys are not leaked, and that, eventually, the keys needed for the communication are not exchanged by using the 3-step-protocol \ref{3-step-protocol}. This is because if an intruder collects all of the exchanged messages it may able to discover $m$ by applyng the XOR function to them:
\begin{gather*}
    c_{1} \oplus c_{2} \oplus c_{3} =\\
    m \oplus k_{A} \oplus m \oplus k_{A} \oplus k_{B} \oplus m \oplus k_{B} =\\
    m \oplus m \oplus m = m \\
\end{gather*}
That's because the XOR operator is commutative and associative, and also because $\forall a: a \oplus a = 1$, therefore $k_{A}, k_{B}$ cancel themselves out.
