\section{RSA}
This section is dedicate to the entire RSA cryptosystem.
\subsection{RSA cryptosystem}
\begin{definition}[RSA cryptosystem]
    The RSA cryptosystem is defined as $(\mathcal{M},\mathcal{C},\mathcal{K}, \mathscr{f}_{k_{e}},\mathscr{f}_{k_{d}}^{-1})$, where:
    \begin{itemize}
        \item $\mathcal{M} = \mathcal{C} = \mathbb{Z}_{n}^{*}$;
        \item $\mathcal{K} = \mathbb{Z}_{\varphi(n)}^{*}$
        \item $\mathscr{f}_{A}(x) = x^{e} \bmod n$
        \item $\mathscr{f}_{A}^{-1}(y) = y^{d} \bmod n$
    \end{itemize}
    The meaning of the values $A, e, n$ will be explained in the following paragraph.
\end{definition}

\subsection{How RSA works}
\subsubsection{Initialization}
In order to communicate with the RSA cryptosystem, each user $A$ must follow these steps:
\begin{enumerate}
    \item Pick $p, q \in \mathbb{Z}: p \neq q$, two large prime numbers.
    \item Compute $n = p \cdot q$;
    \item Compute $\varphi(n) = (p - 1)(q - 1)$;
    \item Pick $e \in \mathbb{N}: (e, \varphi(n)) = 1$
    \item Compute $d \in \mathbb{Z}_{\varphi(n)}^{*}: e \cdot d \equiv_{\varphi(n)} 1$;
    \item Then, $(e,n)$ will be $A$'s public key;
    \item $(p, q, d)$ will then be his private key.
\end{enumerate}
\subsubsection{1 on 1 Communication}
If the user $A$ wants to communicate to $B$, it must encrypt its message with $B$'s public key. Assume that $M$ is the clear-text message and that $C$ is the enciphered message. Then, $A$ computes as follows:
\[
    C = M^{e_{B}} \bmod n_{B} \\
\]
In order to read the ciphered message, $B$ has to decipher $C$ by using the deciphering function with his private key. $B$ computes as follows:
\begin{align*}
    C^{d_{B}} \bmod n_{B} &= (M^{e_{B}})^{d_{B}} \bmod n_{B} \\
    &= M^{e_{B} \cdot d_{B}} \bmod n_{B} \\
    &= M \bmod n_{B} \\
    &= M
\end{align*}
This works because $e_{B}$ and $d_{B}$ are picked in such a way that they're each other's inverse in $\mathbb{Z}_{\varphi(n)}^{*}$, and therefore:
\[
    e_{B} \cdot d_{B} \equiv_{n_{B}} 1
\]
Also, since $n_{B}$ is supposed to be a really large number, is safe to assume that $M \leq n_{b}$, therefore:
\[
M \bmod n_{B} = M
\] 

\subsection{Broadcast communications with RSA}
In order to communicate with multiple subjects, each user has just to repeat the Initialization and Communication procedure described before, one time for each other node in the network. \newline
It's important that the user does not use the same public keys multiple times, in order to prevent Broadcast attacks.
