This section explores the state of the art on the factoring problem, that is at the base of the attacks of numerous crypto algorithms. \newline
\section{Eratostene's sieve}
In ancient times, a Greek mathematician named Eratostene came up with an intuitive factoring method, based on the extraction of the prime numbers up to a given $N$.
\RestyleAlgo{ruled}
\begin{algorithm}
\KwData{$N \in \mathbb{N}$}
\KwResult{$c$, the list of booleans that represents the prime integers up to $N$}
\caption{Eratostene's sieve}\label{alg:eratostene_sieve}
$c[N] \gets \{True * N\}$\;
$p \gets 2$;
\While{$p^{2} \leq N$}{
    $n \gets p^{2}$\;
    \While{$n \leq N$}{
        $c[N] \gets False$\;
        $n \gets p + n$\;
    }
    \Repeat{$c[p] = True$}{
    $p \gets p + 1$\;
    }
}
\Return{$c$}\;
\end{algorithm}
Let's consider the cost of this algorithm:
\begin{itemize}
    \item For each $p \leq N^{1/2}$, one squaring operation is executed, and
    \[
    l_{p} = \lfloor \frac{N}{p} \rfloor - p \leq \frac{N}{p}
    \]
    sums are executed.
    \item Since $p^{2} + kp \leq N$ for $k \leq l_{p}$, then:
    \begin{align*}
        \sum_{p \leq N^{1/2}} (\operatorname{log}^{2}(p) + \sum_{k=1}^{l_{p}} \operatorname{log}(p^{2} + kp)) & \leq \\
        \sum_{p \leq N^{1/2}} (\operatorname{log}^{2}(p) + \operatorname{log}(N) \sum_{p \leq N^{1/2}} \frac{N}{p})\\
        & = N \operatorname{log}(N) \sum_{p \leq N^{1/2}} \frac{1}{p} + O(N^{1/2} \operatorname{log}(N)) \\
        & = O(N \operatorname{log}(N) \operatorname{log}(\operatorname{log}(N)))
    \end{align*}
\end{itemize}

\section{Trial division method}
The simplest algorithm to factorize a given number is to proceed by attempts. \newline
This algorithm tries to do it in the most efficient way possible, but it is still less efficient than the methods that will be proposed later.
\RestyleAlgo{ruled}
\begin{algorithm}
\KwData{$N \in \mathbb{N}$, $L$ the list of prime numbers up to $\sqrt{N}$}
\KwResult{$c$, the list of booleans that represents the prime integers up to $N$}
\caption{Trial-division method}\label{alg:trial_division_method}
$c \gets$ \List{empty}\;
\For{$m \in L$}{
    \If{$m | N$}{
        $c \gets $\Append{$c,m$}\; \Comment{Appends the element $m$ to the list $c$}
    }
}
\Return{$c$}\;
\end{algorithm}
If the list of prime numbers up to $\sqrt{N}$ is provided in input, it is necessary to compute, in the worst case, $\sqrt{N}$ reminders, each one at the cost of $\operatorname{log}^{2}(N)$. \newline
Therefore, the cost of this method is $O(\sqrt{N} \operatorname{log}^{2}(N))$.

\section{Fermat factoring method}
This method aims to factorize a number $N$ in two numbers $p, q$, such that $N = p \cdot q$. Also, there must be some $x,y$ such that:
\[
N = x^{2} - y^{2} = (x + y)(x - y)
\]
If $N$ is odd it can be shown that $y = \frac{N - 1}{2}, x = \frac{N + 1}{2}$.
\RestyleAlgo{ruled}
\begin{algorithm}
\KwData{$N \in \mathbb{N}$}
\KwResult{$p,q$}
\caption{Fermat's factoring method}\label{alg:fermat_factoring_method}
$y \gets$ 1\;
\Repeat{$y = \frac{N - 1}{2}$}{
    $x \gets N + y^{2}$\;
    \If{$(\lfloor \sqrt{N + y^{2}}\rfloor)^{2} = N + y^{2}$}{}\Comment{Checks if $N + y^{2}$ is a perfect square}
    \Return{$x,y$}
    $y \gets y + 1$\;
}
\Return{$p = x + y, q = x - y$}
\end{algorithm}
Let's now consider the cost of this algorithm:
\begin{itemize}
    \item Each iteration has a cost of
    \begin{align*}
        O(\operatorname{log}^{2}(y) + \operatorname{log}^{2}(N + y^{2}) + \operatorname{log}^{3}(N + y^{2})) & = \\
        O(\operatorname{log}^{3}(N + y^{2}))
    \end{align*}
    \item The loop is repeated $O(\operatorname{log}^{A}(N))$ times.
    \item Therefore, the complexity of this algorithm is $O(\operatorname{log}^{A + 3}(N))$ b.o..
\end{itemize}

\section{Pollard's rho-method}
The Pollard's $\rho$-method tries to factorize a number $N$, by attempting to find a collision when applying the same function multiple times. That can be summarized as follows:
\begin{itemize}
    \item Let $F: \mathbb{Z}_{N}^{*} \rightarrow \mathbb{Z}_{N}^{*}$.
    \item Let $x_{0} \in \mathbb{Z}_{N}^{*}$ be a randomly chosen seed.
    \item Let $F^{(i)}(x_{0}) = F \circ F \circ \dots \circ F(x_{0})$.
    \item We want to find $i,j$ such that $F^{(i)}(x_{0}) \equiv_{N} F^{(j)}(x_{0})$, and compute $(N, F^{(i)}(x_{0}) - F^{(i)}(x_{0}))$
\end{itemize}
Let's now investigate why this algorithm is correct:
\begin{itemize}
    \item Assume that $p$ is prime and that $p|N$.
    \item Build a sequence of $T$ numbers: $\{F_{x_{0}}\}_{k \leq T \in \mathbb{N}}$
    \item Assume that exists a collision, so:
    \begin{align*}
        \exists F^{(i)}(x_{0}) \equiv_{p} F^{(j)}(x_{0}) & \\
        & \iff p | F^{(i)}(x_{0}) - F^{(j)}(x_{0}) \\
        & \iff (p, F^{(i)}(x_{0}) - F^{(j)}(x_{0})) > 1
    \end{align*}
    \item Due to the Birthday Paradox, we know that:
    \begin{align*}
        \mathbb{P}[\exists F^{(i)}(x_{0}) \equiv_{p} F^{(j)}(x_{0})] = \frac{1}{2} \text{, when } T \leq \sqrt{p} \\
        \text{Since } p|N \implies p \leq \sqrt{N} \implies T = O(\sqrt[4]{N})
    \end{align*}
    \item Therefore, probably we will find the collision.
\end{itemize}
The cost of the algorithm is easy to compute:
\begin{itemize}
    \item We have approximately $O(T^{2}) = O(\sqrt{N})$ steps, that is the quantity necessary to make the Birthday Paraodx hypothesis hold;
    \item Each one has a cost of $O(log^{2}(N))$ b.o.
    \item The expected cost is then O$(\sqrt(N) log^{2}(N))$ b.o.
\end{itemize}

\RestyleAlgo{ruled}
\begin{algorithm}
\KwData{$N,T \in \mathbb{N}, F: \mathbb{Z}_{N}^{*} \rightarrow \mathbb{Z}_{N}^{*}$}
\KwResult{List of $d$, non-trivial factors of $N$}
\caption{Pollard's $\rho$-method}\label{alg:pollard_rho_method}

$x_{0}$ is randomly chosen in $\mathbb{Z}_{N}^{*}$\;
$m \gets 1$\;
$y_{1} \gets F(x_{0})$\;
$y_{2} \gets F(y_{1})$\;
\While{$m \leq T$}{
    $d \gets (N, y_{1} - y_{2})$\;
    \If{$d > 1 \land d < N$}{
        \Return{$d$}
    }
    $m \gets$ m +1\;
    $y_{1} \gets F(y_{1})$\;
    $y_{2} \gets F(F(y_{2}))$\;
}
\end{algorithm}

\section{Pollard's $p - 1$ method}
This method is an alternative to the $\rho$-method. Consider what follows:
\begin{itemize}
    \item Let $N$ be an odd number.
    \item Choose $a \in \mathbb{Z}_{N}^{*}$.
    \item Recall that due to the Little Fermat Theorem (\ref{little_fermat_theorem}):
    \[
    p \text{ is prime } \iff a^{k} \equiv_{p} 1
    \]
    When $k$ is a multiple of $p-1$.
    \item Therefore:
    \[p | (a^{k} - 1, N)\]
    \item So, we can assume that if $p | N$, then:
    \[
    p - 1 = \prod_{q} q^{\alpha}
    \]
    Where $q$ is prime, and $\alpha$ is a generic exponent.
    \item Also, we assume that:
    \[\exists B \in \mathbb{N}: p | N \land (q^{\alpha}| p - 1 \implies q^{\alpha} \leq B)\]
\end{itemize}
 The algorithm \ref{alg:pollard_p-1_method} resumes the method proposed by Pollard. \newline
 The cost of this algorithm can be estimated as follows:
 \begin{itemize}
     \item Step \ref{pol_p_step_1} costs $O(B^{2} \operatorname{log}^{2}(B))$;
     \item Step \ref{pol_p_step_2} costs $O(\operatorname{log}^{2}(N))$;
     \item Step \ref{pol_p_step_3} costs $O(\operatorname{log}^{2}(N)\operatorname{log}(k))$;
     \item The total cost is then $O(B^{2} \operatorname{log}^{2}(B))$ per attempt.
 \end{itemize}

\RestyleAlgo{ruled}
\begin{algorithm}
\KwData{$N \in \mathbb{N}$}
\KwResult{List of $d$, non-trivial factors of $N$}
\caption{Pollard's $p-1$ method}\label{alg:pollard_p-1_method}
$B$ is randomly chosen\label{pol_p_step_0}\;
$k \gets B!$\;\label{pol_p_step_1}
$a$ is randomly chosen in $\mathbb{Z}_{N}^{*}$\;\label{pol_p_step_2}
$d \gets (N, a^{k} - 1)$\;\label{pol_p_step_3}
\If{$d = 1 \lor d = N$}{
    \GoTo{\ref{pol_p_step_2}}
    OR\;
    \GoTo{\ref{pol_p_step_0}}
}
\end{algorithm}

\section{Pomerance's quadratic sieve}
This algorithm is subexponential, and performs the factorization of a given number in $O(\operatorname{exp}(c_{1}\sqrt{\operatorname{log}(N)\operatorname{log}(\operatorname{log}(N))}))$ b.o.
\subsection{Kraithcik's idea}
This algorithm follows the Kraithcik's idea, that can be seen as a generalization of the Fermat's Factoring method. It's summerized as follows:
\begin{itemize}
    \item Find a sequence of congruences:
    \[A_{i} \equiv_{N} B_{i}\]
    Where $A_{i} \neq B_{i}$.
    \item Find the complete factorization for a large enough number of $A_{i}, B_{i}$.
    \item Find a set $S$ of congruences built starting from the ones found previously, such that:
    \begin{align*}
        \prod_{i \in S'} A_{i} &\equiv_{N} X^{2} \\
        \prod_{i \in S} B_{i} &\equiv_{N} Y^{2}
    \end{align*}
    And therefore $X^{2} \equiv_{N} Y^{2}$
    \item Compute:
    \[ d = (X - Y, N)\]
    Since we know the factorization of $A_{i}$ (and $B_{i}$), we have that:
    \begin{align*}
        A_{i} = \prod_{p_{j} \leq B} p^{\alpha_{ij}}, \alpha_{ij} \in \mathbb{N} \\
        \therefore A_{1} \cdot A_{2} = \prod_{p_{j} \in B} p^{\alpha_{1j} + \alpha_{2j}} \\
        A_{1} \cdot A_{2} \text{ is a square } \iff \alpha_{1j} + \alpha_{2j} \text{ is even}
    \end{align*}
    We can know map this problem to a linear system, where each $A_{i}$ can be represented as a vector, in which each element is the exponent $\alpha_{i,j} \bmod 2$. \newline
    We now have a matrix of $k = \pi(B)$ columns and $T = |S|$ rows, and we want to solve the system $\lambda \cdot v_{2}(A)= 0$.
\end{itemize}
\subsection{The algorithm}
In order to better understand the algorithm, the following clarification are needed:
\begin{itemize}
    \item Let $A_{j} \in \mathbb{N}$ s.t.:
    \[
    Q(A_{j}) = \prod_{p \in \mathbb{B}}p^{\alpha_{p,j}}
    \]
    Where $\alpha_{p,j} \in \mathbb{N} \land k = |\mathbb{B}|$
    \item Let $\mathbb{v}(A_{j}) = (\alpha_{2,j}, \alpha_{3,j}, \dots, \alpha_{p,j})$
    \item Let $\mathbb{v}_{2}(A_{j}) = (\alpha_{2,j} \bmod 2, \alpha_{3,j} \bmod 2, \dots, \alpha_{p,j} \bmod 2)$
\end{itemize}
This algorithm has a cost of $O(\operatorname{exp}(c \sqrt{\operatorname{log}(N)\operatorname{log}(\operatorname{log}(N))}))$

\RestyleAlgo{ruled}
\begin{algorithm}
\KwData{$N,B \in \mathbb{N}$}
\KwResult{List of $d$, non-trivial factors of $N$}
\caption{Pomerance's quadratic sieve}\label{alg:pomerance_sieve}
$s \gets \lfloor \sqrt{N} \rfloor$\;
Pick randomly $A \in \mathbb{N}$\;
$Q(A) \gets (A + s)^{2} + N$\;
$\mathbb{B} \gets \{2\} \cup \{p: p \text{ is prime } \land p \leq B \land p | Q(A)\}$\;
$X \gets \sqrt{\prod_{i=1}^{m}(A_{i} + s)^{2} \bmod N}$\;
$Y \gets \sqrt{\prod_{i=1}^{m} Q(A_{i}) \bmod N}$\;
$d \gets (X - Y, N)$\;
\end{algorithm}
