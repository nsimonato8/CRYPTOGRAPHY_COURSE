\documentclass[12pt, a4paper, english]{report}
\usepackage[utf8]{inputenc}
\usepackage{makecell}
\usepackage{fourier}
\usepackage{caption}
\usepackage{amssymb}
\usepackage[linesnumbered]{algorithm2e} %,lined,boxed,commentsnumbered
\usepackage{subcaption}
\usepackage{epigraph}
\usepackage{tikz}
\usepackage{mathrsfs}
\usepackage{babel}
\usepackage{amsfonts}
\usepackage{geometry}
\usepackage{hyperref}
\usepackage{csquotes}
\usepackage{underscore}
\usepackage{amsmath}
\usepackage{amsthm}
\usepackage{thmtools}
\usepackage{biblatex}
\usepackage[rightcaption]{sidecap}
\usepackage{graphicx}
\usepackage{tabularx}
\usepackage{boondox-cal}

\SetKwComment{Comment}{/*}{*/}
\SetKwRepeat{Do}{do}{while}
\SetKw{GoTo}{GoTo}
\SetKwData{List}{List}
\SetKwFunction{Append}{append}

\newtheorem{theorem}{Theorem}
\newtheorem{definition}{Definition}
\newtheorem{lemma}{Lemma}
\newtheorem{proposition}{Proposition}

\DeclareMathOperator*{\argmin}{arg\,min}

\graphicspath{ {img/} }

\author{Niccolò Simonato}
\title{Cryptography: notes}

\begin{document}

\maketitle

\tableofcontents

\chapter{Elements of Number Theory}
\section{Definitions of Number Theory}
\subsection{The cyclic group $\mathbb{Z}_{n}^{*}$}
\begin{definition}[The cyclic group $\mathbb{Z}_{n}^{*}$]
    The cyclic group $\mathbb{Z}_{n}^{*}$ is defined as
    \[
    \mathbb{Z}_{n}^{*} = \{a \in \mathbb{Z}_{n}: (a,n) = 1\}
    \]
    The \textbf{generator} of $\mathbb{Z}_{n}^{*}$ is a number in $\mathbb{Z}_{n}^{*}$ such that:
    \[
    \forall a \in \mathbb{Z}_{n}^{*}: \exists i: a \equiv_{m} g^{i}
    \]
\end{definition}
For this reason, $\mathbb{Z}_{n}^{*}$ is also referred as $< g >$.\newline
An interesting property is that only for $[1,2,4, \phi, \phi^{\alpha}, 2 \phi^{\alpha}]$, $\mathbb{Z}_{n}^{*}$ is cyclic, where $\phi$ is a prime number and $\phi^{\alpha}$ is a power of a prime number\ref{gauss_theorem}.

\subsection{Pseudoprime number}
\begin{definition}[Pseudoprime number]
    For an integer $a > 1$, if a composite integer $x$ divides $a^{x-1} - 1$, then $x$ is called a \textbf{Fermat pseudoprime} to base $a$.
    \[
    x \text{ is spsp(a) } \iff x|a^{x - 1} - 1
    \]
\end{definition}
In other words, a composite integer is a Fermat pseudoprime to base a if it successfully passes the Fermat primality test for the base $a$.\newline
$x$ is $spsp(a)$ is a predicate, that means that some $x$ is a \emph{strong pseudoprime number in base $a$}.

\subsection{Carmichael numbers}
\begin{definition}[Carmichael numbers]
    Let $n$ be a composite number. Then:
    \[b^{n} \equiv_{n} b \implies b \text{ is a \textbf{Carmichael number}}\]
\end{definition}

\subsection{B-smoothness}
\begin{definition}[B-smoothness]
    Let $n, B > 0 \in \mathbb{Z}$. Let also $p$ be a prime number. Then:
    \[(p|n \implies p \leq B) \iff n \text{ is a \textbf{B-smooth number}}\]
\end{definition}
That is, if all the prime divisors of $n$ are smaller than $B$.

\section{Reminders of Modular Arithmetic}
\subsection{Little Fermat's Theorem}\label{little_fermat_theorem}
\begin{theorem}[Little Fermat's Theorem]
    Consider what follows:
    \begin{itemize}
        \item Let $p$ be a prime number and $p \nmid a$.
        \item Then \[a^{p-1} \equiv_{p} 1\]
    \end{itemize}
\end{theorem}
\begin{proof}
    Consider what follows:
    \begin{itemize}
        \item Let's consider: \[A = \{na \bmod p: n \in [1, p-1]\}\]
        \item $A$ has all distinct elements due to the Bijection Lemma (Lemma~\ref{bijection_lemma}).
        \item Then \[A = \mathbb{Z}_{n}^{*}\]
        \item Then, \[\prod\limits_{n \in A} n \equiv_{p} \prod\limits_{n=1}^{p-1} na \implies (p-1)! \equiv_{p} (p-1)!a^{p-1}\]
        \item Therefore, $a^{p-1} \equiv_{p} 1$ for the Wilson's Theorem~\ref{wilson_th}.
    \end{itemize}
\end{proof}

\subsection{Wilson's Theorem}\label{wilson_th}
\begin{theorem}[Wilson's Theorem]
    Consider what follows:
    \begin{itemize}
        \item Let $n \in \mathbb{N} \setminus \{0,1\}$
        \item Then, \[(n-1)! \equiv_{n} -1\]
    \end{itemize}
\end{theorem}
\begin{proof}
    The proof is \textbf{by induction} on $n$:
    \begin{itemize}
        \item \textbf{Base case}: $n = 2$
        \[1! \equiv_{2} -1\]
        \item \textbf{Inductive step}:
        The theorem is assumed to be true up until $n - 1$. Let's consider the case of $n$:
        \begin{itemize}
            \item Consider the polynomial \[g(x) = (x-1)(x-2) \dots (x - (n-1))\]
            \item $g$ has degree $p-1$ and costant term $(p-1)!$. It's roots are in $[1,p-1]$.
            \item Consider \[h(x) = x^{p-1} - 1\] $h$ has also degree $p-1$ and leading term $x^{p-1}$.
            \item Let $f(x) = g(x) - h(x)$.
            \item Then, $f$ has degree at most $p - 2$ (since the leading terms cancel), and modulo $p$ also has the $n - 1$ roots $1, 2, ..., n - 1$.
            \item But Lagrange's theorem says it cannot have more than $p - 2$ roots.
            \item \[\therefore f \equiv_{n} 0\]
            \item Its costant term, \[(n-1)! + 1 \equiv_{n} 0 \iff (n-1)! \equiv_{n} -1\]
        \end{itemize}
    \end{itemize}
\end{proof}

\subsection{Euler-Fermat's Theorem}\label{euler_fermat_th}
\begin{theorem}[Euler-Fermat's Theorem]
    Consider what follows:
    \begin{itemize}
        \item Let $n \in \mathbb{Z}$ be a number.
        \item Then, \[a^{\varphi(n)} \equiv_{n} 1\]
    \end{itemize}
\end{theorem}

\subsection{Bézout's identity}
\begin{theorem}[Bezout's identity]
    Consider what follows:
    \begin{itemize}
        \item Let $a$ and $b$ be integers with greatest common divisor $d$.
        \item Then there exist integers $x$ and $y$ such that $ax + by = d$.
        \item Moreover, the integers of the form $az + bt$ are exactly the multiples of $d$.
    \end{itemize}
\end{theorem}

\subsection{Chinese Reminder's Theorem}
\begin{theorem}[Chinese Reminder's Theorem]
    Given a system of congruences:
    \begin{align*}
        x & \equiv_{m_{1}} a_{1} \\
        x & \equiv_{m_{2}} a_{2} \\
        &\dots \\
        x & \equiv_{m_{r}} a_{r}
    \end{align*}
    In which each module is prime with each others, that is:
    \[
    \forall i \neq j: (m_{i}, m_{j}) = 1
    \]
    then \textbf{there exists a simultaneous solution $x$ to all of the congruences}, and \textbf{any two solutions are congruent to one another modulo}.
    \[M = m_{1}m_{2}\dots m_{r}\]
\end{theorem}

\begin{proof}
    Consider what follows:
    \begin{itemize}
        \item Suppose that $x'$ and $x"$ are two solutions.
        \item Let $x = x' - x''$.
        \item Then x must be congruent to $0$ modulo each $m_{i}$ and hence modulo $M$.
        \item Let $M_{i} = \frac{M}{m_{i}}$, to be the product of all of the moduli except for the i-th.
        \item Then $GCD(m_{i}, M_{i}) = 1$ and therefore $\exists N_{i}: M_{i}N_{i} \equiv_{m_{i}} 1$.
        \item Set $x = \sum_{i}a_{i}M_{i}N_{i}$;
        \item Then, for each $i$ we see that the terms in the sum other than the i-th term are all divisible by $m_{i}$, because $m_{i} | M_{j}$ when $i \neq j$.
        \item Thus, for each $i$ we have: $x \equiv_{m_{i}} a_{i}M_{i}N_{i} \equiv_{m_{i}} a_{i}$,
    \end{itemize}
\end{proof}

\subsection{Bijection of a modular function Lemma}
\begin{lemma}[Bijection of a modular function Lemma]\label{bijection_lemma}
    Consider:
    \begin{itemize}
        \item \[a \in \mathbb{Z}_{n}^{*}\]
        \item \[f: \mathbb{Z}_{n} \rightarrow \mathbb{Z}_{n}\]
    \end{itemize}
    Then, $f$ is a bijection.\newline
\end{lemma}
\begin{proof}
    Part 1: $f$ is injective.
    \begin{itemize}
        \item Assume $f(x_{1}) = f(x_{2}) \in \mathbb{Z}_{n} \iff ax_{1} \equiv_{n} ax_{2}$.
        \item Then $\exists k \in \mathbb{Z}: ax_{1} - ax_{2} = kn$.
        \item Therefore,
        \begin{align*}
            a(x_{1} - x_{2}) = kn & \iff a^{-1}a(x_{1} - x_{2}) = a^{-1}kn\\
            \iff (x_{1} - x_{2}) = a^{-1}kn & \iff x_{1} - x_{2} \equiv_{m} 0.
        \end{align*}
        \item $\therefore x_{1} \equiv_{m} x_{2}$, so $f$ is an injection.
    \end{itemize}
    Part 2: $f$ is surjective.
    \begin{itemize}
        \item Let $b \in \mathbb{Z}_{n}$.
        \item Let $\overline{x} = a^{-1}b \in \mathbb{Z}_{n}$.
        \item Then, $f(\overline{x}) \equiv_{m} a \cdot \overline{x}$\\
        $\equiv_{m} a \cdot a^{-1} \cdot b \equiv_{m} b$.
        \item Therefore, $f$ is surjective.
    \end{itemize}
    Since $f$ is injective $\land$ $f$ is surjective, then $f$ is bijective.
\end{proof}

\subsection{Euler's $\varphi$ function Lemmas}
\begin{lemma}[Sum of the prime divisors' $\varphi$-function]
    \[\forall n \in \mathbb{N} \backslash \{0\}: \sum_{\frac{n}{d}} \varphi(d) = n\]
    Where $\frac{n}{d}$ is the set of prime divisors of $n$.\newline
\end{lemma}
\begin{proof}
    The proof is by \textbf{contradiction}:
    \begin{itemize}
        \item Let $B$ be \[\{\frac{h}{n}: h \in \mathbb{Z}_{n} \land n \in \mathbb{N} \}\]
        \item Therefore, \[B = \cup_{\frac{n}{d}}\{a \in \{1, \dots, n\} \land (a,d) = 1\}\]
        \item Then, \[n = |B| = \sum_{\frac{d}{n}} \varphi(d)\]
        \item Consider \[(a_{1}, d_{1}) = 1 \land (a_{2}, d_{2}) = 1 \text{, where} d_{1}|n \land d_{2}|n\]
        \item Then, \[\frac{a_{1}}{d_{1}} = \frac{a_{2}}{d_{2}} \iff a_{1}d_{2} = a_{2}d_{1}\]
        \item Then, \[d_{1}|a_{1}d_{2} \implies d_{1}|d_{2} \land
        d_{2}|a_{2}d_{1} \implies d_{2}|d_{1}\]
        \item Therefore,  $d_{1} = d_{2}$. This is clearly a \emph{contradiction}, because in $B$ each divisor is counted once.
    \end{itemize}
\end{proof}

\begin{lemma}[Number of divisors of a prime number's power]
    Let $p \in \mathbb{N}$  be a prime number, and $\alpha \in \mathbb{N}$.\\
    Then, \[\varphi(p^{\alpha}) = p^{\alpha - 1}(p - 1)\]
\end{lemma}
\begin{proof}
    The proof is by \textbf{induction} on $\alpha$.\newline
    \textbf{Case base:} $\alpha = 1$
    \[p = \sum_{\frac{d}{p}} \varphi(d) = \varphi(1) + \varphi(p) = 1 + \varphi(p) \implies \varphi(p) = p - 1\]
    \textbf{Case base:} $\alpha = 2$
    \begin{align*}
        p^{2} & = \sum_{\frac{d}{p^{2}}} \varphi(d) = \varphi(1) + \varphi(p) + \varphi(p^{2})\\
        & = 1 + p - 1 + \varphi(p^{2}) \implies \varphi(p^{2}) = p^{2} - p - 1 + 1\\
        & = p \cdot (p - 1) = p^{\alpha - 1}(p - 1)
    \end{align*}
    \textbf{Inductive Step:} we can now assume that $\varphi(p^{\alpha}) = p^{\alpha - 1}(p - 1)$ up until $\alpha - 1$. We'll proceed now to demonstrate that this is also valid for each $\alpha$.
    \begin{align*}
        p^{\alpha} & = \sum_{\frac{d}{p^{\alpha}}} \varphi(d)\\
        & = \sum_{i = 0}^{\alpha - 1} \varphi(p^{i}) + \varphi(p^{\alpha})\\
        & \implies \varphi(p^{\alpha}) = p^{\alpha} - p^{\alpha - 1} = p^{\alpha - 1}(p - 1).
    \end{align*}
    \newline
\end{proof}

\subsection{Multiplicativity of the Euler's $\varphi$-function}
\begin{theorem}[Multiplicativity of the Euler's $\varphi$-function]
    Let's consider the Euler's function $\varphi(n) = |\mathbb{Z}_{n}^{*}$.\\
    Let's consider that $n = \prod_{i=1}^{r} p_{i}^{\alpha_{i}}$, where $p_{i}$ is a prime number, and $\alpha_{i} \in \mathbb{N}$.\\
    Then, $=$
\end{theorem}

\section{Theorems for Cryptography purposes}
\subsection{The Miller-Rabin Theorem}
\begin{theorem}[The Miller-Rabin Theorem]
    The \textbf{Miller-Rabin Theorem states that}:
    \begin{itemize}
        \item Let $p$ be a prime number.
        \item Let $a \in \mathbb{N}: (a, p) = 1$
        \item Let $p - 1 = 2^{s}d$, where $d$ is odd.
        \item Then:
        \begin{itemize}
            \item $a^{d} \equiv_{p} 1$ or
            \item $a^{2^{r}d} \equiv_{p} -1 \text{ for some r } \in \{0, 1, \dots, s-1\}$.
        \end{itemize}
    \end{itemize}
\end{theorem}
\begin{proof}
    The first part of the proof consists in showing that, if $n$ is a prime, then $\sqrt{1} \equiv_{n} \pm 1$:
    \begin{align*}
        &x^{2} \equiv_{n} 1 &\iff x^{2} - 1 \equiv_{n} 0\\
        \iff& (x + 1)(x - 1) \equiv_{n} 0 &\iff x = \pm 1 \bmod n
    \end{align*}
    The second part of the proof consist in showing  that, if $n$ is an odd prime, then it is $spsp(a)$.
    \begin{itemize}
        \item Let $n$ be prime;
        \item Let $a: (a,n) = 1$;
        \item Let $s,d: n - 1 = 2^s \cdot d$, where $s > 0$ and $d$ is odd.
        \item Then $a^{n-1} \equiv_{n} 1 \iff a^{2^s \cdot d} \equiv_{n} 1$, due to Fermat's Little Theorem.
        \item We can now consider $a^{2^r \cdot d}$ successive square roots, for $1 \leq r < s$.
        \item Therefore, starting from $r = s - 1$, we can compute $a^{2^r \cdot d} \bmod n$. If it's $-1$, then $n$ is $spsp(a)$. If it's $1$, we can repeat the computation for $r = s - 2$, and so on. Due to the previous part, there are no other possible cases.
    \end{itemize}
\end{proof}

\subsection{Miller's Theorem}
\begin{theorem}[Miller's Theorem]
    The Miller's Theorem states what follow:
    \begin{itemize}
        \item Let $n$ be a composite and odd number.
        \item Then, $n$ is $spsp(a)$ for at most $\frac{1}{4}$ of the $a_{i} \in \mathbb{Z}_{n}^{*}$
    \end{itemize}
\end{theorem}
The following lemma is a consequence of this theorem.
\begin{lemma}[Corollary of the Miller's Theorem]
    If $n$ is composite and odd, then \[\exists a \in \mathbb{Z}_{n}^{*}: a \leq b \text{, such that $n$ is not $spsp(a)$} \]
\end{lemma}

\subsection{Ankey-Montgomery-Bach Theorem}
\begin{theorem}[Ankey-Montgomery-Bach Theorem]
    The Ankey-Montgomery-Bach Theorem states that:
    \begin{itemize}
        \item If the GRH\footnote{Generalized Riemann Hypothesis} holds;
        \item If $n$ is composite and odd;
    \end{itemize}
    Then, \[\exists \in [2, 2 log^{2}(n)] \text{ such that $n$ is not $spsp(a)$}\]
\end{theorem}

\section{Euler's Product Theorem}
\begin{theorem}[Euler's Product Theorem]
    If $ \mathbb{R}e(s) > 1$, then \[\zeta(s) = \sum_{n = 1}^{+ \infty} \frac{1}{n^{2}} = \prod\limits_{p}(1 - \frac{1}{p^{2}})^{-1}\]
\end{theorem}

\section{Discrete Logarithm problem}
\begin{definition}[Discrete Logarithm problem]\label{discrete_logarithm_prob}
    Given $p, g, y$ where:
    \begin{itemize}
        \item $p$ is a large prime number;
        \item $g$ is the generator of $\mathbb{Z}_{p}^{*}$;
        \item $y \in \mathbb{Z}_{p}^{*}$.
    \end{itemize}
    Find $x \in \mathbb{Z}_{p-1}: g^{x} = y$
\end{definition}

\section{Gauss' Theorem}
\begin{definition}[Gauss' Theorem]\label{gauss_theorem}
    The theorem states as follows:
    \begin{itemize}
        \item $\mathbb{Z}_{n}^{*}$ is cyclic $\iff n = 1, 2, 4, p^{\alpha}, 2p^{\alpha}$
        \item Assume that we know $g$ such that $<g> = \mathbb{Z}_{p}^{*}$:
        therefore we can build $<g_{1}> = \mathbb{Z}_{p^{\alpha}}^{*}, <g_{2}> = \mathbb{Z}_{2p^{\alpha}}^{*}$ in polinomial time.
    \end{itemize}
\end{definition}


\chapter{Efficient implementations of elementary operations}
\section{Notation}
\begin{itemize}
    \item Let $b$ be a numeric base.
    \item Let $n$ be a number in $N$.
    \item Length of a number: $l_{b}(n)$, $k$. It's equal to $\operatorname{log}(n)$.
    \item $(a,b)$ is the Maximum Common Divisor of $a,b$.
    \item Let $n \in N$: $n = (d_{k-1}, d_{k-2}, \dots, d_{1}, d_{0})$\footnote{$d_{k-1} \neq 0$}.
    \item $\varphi(n)$: the number of elements $a$ in $[1,n]$ such that $(a,n) = 1$.
    \item $\equiv_{p}$ is the equivalence in base $p$. Ex.: $5 \equiv_{3} = 5 \bmod 3 = 2 $.
    \item Let $\mathbb{Z}_{n}[X]$ be the set of polynomials in $X$ with coefficients in $\mathbb{Z}_{n}$.
\end{itemize}


\section{Classification of the algorithms' complexity}
In order to better identify the classes of complexity of the algorithms, the following 3 classes are defined:
\begin{itemize}
    \item Polynomial time: $O(\operatorname{log}^{\alpha}(n))$ bit operations, where $\alpha > 0$.
    \item Exponential time: $O(exp(c \cdot \operatorname{log}(n)))$ bit operations, where $c > 0$.
    \item Sub-exponential time: $O(exp(c \cdot \operatorname{log}(n))^{\alpha})$ bit operations, where $c > 0, \alpha \in ]0, 1[$.
\end{itemize}



\section{Basic bit operations}
\subsection{Sum of 3 bits - 3-bit-sum}
Given $n_{1}, n_{2}$ their sum produces $n_{1} + n_{2}$ and their carry. \newline
Since $n_{1}, n_{2} \in [0,1]$, then this operation can be done in $O(1)$.

\subsection{Summation of 2 numbers}
Given $n_{1}, n_{2}$ their sum produces $n_{1} + n_{2}$. \newline
Since the sum is computed bit by bit, the 3-bit-sum is performed \[\operatorname{max}\{\operatorname{lenght}(n_{1}), \operatorname{length}(n_{2})\}\] times.\newline
Each time the carry-on of the previous sum is added to the two digits. \newline
This operation has then complexity:
\[O(\operatorname{max}\{lenght(n_{1}), \operatorname{length}(n_{2})\}) = O(\operatorname{max}\{\operatorname{log}(n_{1}), \operatorname{log}(n_{2})\})\]

\subsection{Summation of $n$ numbers}
The summation of $n$ numbers is simply the sum of two numbers, but performed $n - 1$ times. \newline
Let's assume that: \[\forall i \in [1,n]: M \geq a_{i}\]
The complexity of this operation is then: \[O((n-1) \cdot \operatorname{log}(M)) = O(n)\] \newline

\subsection{Product of 2 numbers}
If we consider the classic implementation of the binary multiplication, that is just a sequence of summations. \newline
\begin{itemize}
    \item The number of summations to execute is equal to the length of the smallest number, $O(\operatorname{log}(n))$.
    \item The maximum cost of a single summation is $O(\operatorname{log}(m))$.
    \item Then, $T(m \cdot n) = O(\operatorname{log}(m) \cdot \operatorname{log}(n))$, but, if we consider the worst case\footnote{two numbers that are equally large}, that becomes $O(\operatorname{log}^{2}(m))$.
\end{itemize}

\subsection{Division of 2 numbers}
Let's consider the division of two numbers $m, n$. This operation consists in \\ finding two numbers $q, r$ such that $m = q \cdot n + r$. \newline
This is achieved by performing a succession of subtractions, until the ending condition $0 \leq r < n$ is reached. \newline
\begin{itemize}
    \item Let's consider that the number of steps of this algorithm is $O(\operatorname{log}(q))$.
    \item Moreover: \[q \leq m \therefore \#steps = O(\operatorname{log}(m))\]
    \item It's assumed that the cost of the single subtraction is $O(\operatorname{log}(n))$.
    \item Then: \[T(\frac{m}{n}) = O(\operatorname{log}(n) \cdot \operatorname{log}(m))\]
\end{itemize}

\subsection{Production of $n$ numbers}
Let's assume that: \[j \in [1, s+1] \text{ and } M = \operatorname{max}(m_{j})\]
The cost of the operation $\prod_{j = 1}^{s+1} m_{j}$ is then $O(s^{2} \cdot \operatorname{log}^{2}(M))$. This will now be considered our inductive hypothesis.\newline
Proof by induction, on $s$:
\begin{itemize}
    \item (1) \textbf{Base case}: \[T(m_{1} \cdot m_{2}) = O(\operatorname{log}(m_{1}) \cdot \operatorname{log}(m_{2})) = O(k_{1} \cdot k_{2}) \leq c \cdot k_{M}^{2}\]
    \item (2) \textbf{Base case}:
    \begin{align*}
        T(m_{1} \cdot m_{2} \cdot m_{3}) & = T(m_{1} \cdot m_{2}) + T((m_{1} \cdot m_{2}) \cdot m_{3}) \\
        & \leq c \cdot k_{M}^{2} + c \cdot k_{m_{1} \cdot m_{2}} + k_{m_{3}} \\
        & \leq c \cdot k_{M}^{2} + c \cdot k_{M^{2}} + k_{M}
    \end{align*}
    \item Inductive step: we assume the inductive hypothesis to be true up to $s$. Then:
    \begin{align*}
        T(\prod_{j = 1}^{s+1} m_{j}) & = T([\prod_{j = 1}^{s} m_{j}] \cdot m_{s+1}) \\
        & \leq c \cdot \sum_{j=1}^{s} (j \cdot k_{M}^{2}) \\
        & = c \cdot k_{M}^{2} \cdot \frac{s \cdot (s-1)}{2} \\
        & = O(k_{M}^{2} \cdot s^{2}) \\
        & = O(s^{2} \cdot \operatorname{log}^{2}(M)) \\
    \end{align*}
\end{itemize}
\subsubsection{Applications}
\begin{itemize}
    \item An analogous dimonstration can be used to prove that \[T(\prod_{j = 1}^{s+1} m_{j} \bmod n) = O(s \cdot \operatorname{log}^{2}(M))\]
    \item This proof can be used to show that \[T(m!) = O(m \cdot \operatorname{log}^{2}(m))\]
\end{itemize}

\section{Optimizations of more complex operations}
\subsection{Powers \& Modular Powers}
Let's consider what follows: \[a^n = a \cdot a \cdot a \cdot \dots \cdot a\]
Where $a$ is repeated $n$ times.
\subsubsection{Trivial implementation}
The most trivial implementation would consists in computing the product $\prod_{j = 1}^{n} a$. This would imply a cost of $O(n^{2} \cdot \operatorname{log}^{2}(a)))$.\newline
What follows is a suggestion that could improve the cost of this operation.
\subsubsection{Square \& Multiply method for scalars, modular powers}
Each number in $\mathbb{Z}$ can be represented in a binary notation. \newline
Let's consider
\[n = (b_{k-1}, b_{k-2}, \dots, b_{0}) = \sum_{i=0}^{k-1} b_{i} \cdot 2^{i}\]
It is clear that we can spare a lot of computational resources by just calculating the powers of 2 and summing the ones that have $b_{i} = 1$. The following algorithm explains the procedure in detail.
\RestyleAlgo{ruled}
\begin{algorithm}
\caption{The Square \& Multiply Method}\label{alg:SquareMultiply}
$P \gets 1$\;
$M \gets m$\;
$A \gets a \bmod n$\;\label{instr_3}
\While{$M > 0$}{
    $q \gets \lfloor \frac{M}{2} \rfloor$\;\label{instr_5}
    $r \gets M - s \cdot q$\;\label{instr_6}
    \If{$r = 1$}{
        $P \gets P \cdot A \bmod n$\;\label{instr_8}
    }
    $A \gets A^{2} \bmod n$\;\label{instr_10}
    $M \gets q$\;

}
\Return{P}
\end{algorithm}
Let's compute the complexity of this algorithm:
\begin{itemize}
    \item All of the assignments $X \gets Y$ are implemented in $O(\operatorname{log}(Y))$.
    \item The cost of \ref{instr_3} is $O(\operatorname{log}(a) \cdot \operatorname{log}(n))$, because it ensures that $A \leq n$.
    \item Instructions \ref{instr_5} and \ref{instr_6} can be executed in $O(\operatorname{log}(m))$.
    \item Instrucion \ref{instr_8} can be executed in $O(\operatorname{log}^{2}(n))$.
    \item The cost of \ref{instr_10} is $O(\operatorname{log}^{2}(n))$, because it ensures that $A \leq n$.
    \item The loop is executed $\operatorname{log}(m)$ times.
    \item The total cost of this algorithm is then $O(\operatorname{log}(n) \cdot \operatorname{log}(a) + \operatorname{log}(m) \cdot (\operatorname{log}^{2}(n) + \operatorname{log}(m)))$ \\ $= O(\operatorname{log}^{2}(m) + \operatorname{log}(m) \cdot \operatorname{log}(n))$.
\end{itemize}
This algorithm can be easily converted for the computation of non-modular powers by applying the following changes:
\RestyleAlgo{ruled}
\begin{algorithm}
    $A \gets a \bmod n$ $\Longrightarrow$ $A \gets a$\;\label{instr_3a}
    $P \gets P \cdot A \bmod n$ $\Longrightarrow$ $P \gets P \cdot A$\;\label{instr_8a}
    $A \gets A^{2} \bmod n$ $\Longrightarrow$ $A \gets A^{2}$\;\label{instr_10a}
\end{algorithm}
\subsubsection{Square \& Multiply method for polynomials}
Let's consider $\Re = \frac{\mathbb{Z}_{n}[x]}{x^{r} - 1}$. The modular powers of the elements in this set can be computed by using a variation of the Square \& Multiply method.
\begin{itemize}
    \item Assume that $f, g \in \Re$.
    \item Let \[h(x) = f(x) \cdot g(x) = \sum_{j = 0}^{2r - 2} h_{j} \cdot x^{j}\]
    \item Where \[h(j) = (\sum_{i=0}^{j} f_{i} \cdot g_{j - i} \bmod n) \bmod n\]
    \item Then: \[T(h_{j}) = O(j \cdot \operatorname{log}^{2}(n))\]
    \item Then: \[T(h(x)) = O(\sum_{j = 0}^{\operatorname{log}^{2}(n)}) = O(r^{2} \cdot \operatorname{log}^{2}(n))\]
\end{itemize}
This result will be useful in the following computations. \newline
Let's now consider $\frac{h(x)}{x^{r} - 1}$.
\begin{itemize}
    \item When $j > r - 1$, $h_{j} \cdot x^{j}$ does not take any part in the computations.
    \item When $j = r$, then: \[\frac{h_{r} \cdot x^{r}}{x^{r} - 1 } = h_{r} + \frac{h_{r}}{x^{r} - 1}\]
    Or, in other words: $h_{r} \cdot x^{r} \equiv_{x^{r} - 1} h_{r}$.
    \item In the other cases: \[h_{r+i} \cdot x^{r+i} \equiv_{x^{r} - 1} h_{r-i} \cdot x^{r-i} \text{ for } 1 \leq i \leq r - 2\]
\end{itemize}
%An image can be useful
Then:
\[h(x) \equiv_{(n, x^{r} - 1)} f(x) \cdot g(x) \equiv
[\sum_{j=0}^{r - 2}((h_{j} + h_{r + j}) \bmod n) \cdot x^{j}] + h_{r-1} \cdot x^{r-1}\]\label{comp:partial_prod_pol}

Therefore: \[T(h(x) \bmod (n, x^{r} - 1)) = O(r^{2} \cdot \operatorname{log}^{2}(n))\]
Finally, we can analyze the complexity of the computation of the modular power $h(x)$ elevated to $n$. \newline
In order to optimize the use of the computational resources, we can use a variation of the Square \& Multiply method (See \ref{alg:SquareMultiply});
although, this time, the computation of the partial products will be conducted by using the previously explained procedure (See \ref{comp:partial_prod_pol}). \newline
The cost of this method would then be
\begin{align*}
    T(\#Loops \cdot (h(x) \bmod (n, x^{r} - 1))) = O(\operatorname{log}(n) \cdot r^{2} \cdot \operatorname{log}(n)) \\
    & = O(r^{2} \cdot \operatorname{log}^{3}(n))
\end{align*}



\subsection{Finding the $b$-expansion of $n$ ($n_{b}$)}
Let's consider the cost in bit operations of the conversion of a number $n$ to a new base $b$. \newline
The algorithm used will be the classical: a succession of divisions by $b$. \newline
\begin{itemize}
    \item Let's consider \[r_{i} \in \{0, 1, \dots, b-1\}\]
    \item Let \[n_{b} = (r_{k+1}, r_{k}, \dots, r_{1}, r_{0})\]
    \item Then: \begin{align*}
                    n &= q_{0} \cdot b + r_{0} \\
                    q_{0} &= q_{1} \cdot b + r_{1} \\
                    \dots \\
                    q_{k} &= 0 \cdot b + q_{k}
                \end{align*}
    \item Consider then that \[q_{k} = r_{k+1}\]
    \item And that \[b^{k+2} > n > b^{k+1} \iff (k+2) \cdot \operatorname{log}(b) \leq \operatorname{log}(n) \leq (k+1) \cdot \operatorname{log}(b)\]
    \item Therefore, \[k = O(\frac{\operatorname{log}(n)}{\operatorname{log}(b)})\]
\end{itemize}
We can now proceed with the computation of the cost of this operation:
\begin{align*}
    T(n_{b}) &= T(\#Divisions \cdot q_{i} \bmod b)\\
    & = O(\frac{\operatorname{log}(n)}{\operatorname{log}(b)} \cdot \operatorname{log}(n) \cdot \operatorname{log}(b)) \\
    &= O(\operatorname{log}^{2}(n))
\end{align*}


\subsection{How to use Bezout formula to compute modular inverses}
An efficient way of computing the modular inverse of a given number $a$ with in the group $\mathbb{Z}_{m}^{*}$ uses the corollary of the \emph{Bezout identity} and the \emph{Extended Euclidean Algorithm}.\newline
That is, given $a \cdot x \equiv_{m} 1$, we want to compute $x$.\newline
\emph{Extended Euclidean Algorithm}, given $a, m$ computes the $gcd(a,m)$ and also returns the coefficients $x,y$ for which $ax + my = 1$. \newline
At this point, the modular inverse of $a$ in $\mathbb{Z}_{m}^{*}$ is $x$:
\begin{itemize}
    \item Let's consider that $ax + my \equiv_{m} 1$;
    \item Since $my \equiv_{m} 0$, then $ax \equiv_{m} 1$;
    \item $\therefore x \bmod n$ is the modular inverse of $a$ in $\mathbb{Z}_{m}^{*}$ for its definition.
\end{itemize}
The complexity of this operation is then $O(\operatorname{log}(x)\operatorname{log}(n))$, because we have to compute the remainder of the division between $x$ and $n$ (this does not take into account the execution of the \emph{Extended Euclidean Algorithm}).

\subsection{Computing the order of an element in a cyclic group}
The order of an element $a$ in $\mathbb{Z}_{p}^{*}$ ($m = order(a)$) is the minimum $m$ such that $a^{m} \equiv_{p} 1$.\newline
This problem is computationally hard, because the most efficient way to compute $\operatorname{order}(a)$ is to brute force its value.\newline
The only optimization available is that we don't have to compute the modulars powers of $a$ from scratch each time, but we can save the results at each iteration. Therefore, at each step we can only compute the modular product $(a^{p-1} \cdot a) \bmod p$, that has a cost $O(\operatorname{log}^{2}(p))$.
The cost of this algorithm is then $O(order(a) \cdot \operatorname{log}^{2}(p))$, because we have to compute $a^{i} \bmod p$ for each attempt to find $order(a)$.

\subsection{Extended Euclidean Algorithm}
The Extended Euclidean Algorithm is a variation of the classic Euclidean Algorithm, that computes the GCD between two numbers $a, b$. \newline
It also provides the coefficients $\lambda, \mu$ such that $\lambda \cdot a + \mu \cdot b = \operatorname{GCD}(a,b)$.\newline
\RestyleAlgo{ruled}
\begin{algorithm}
\KwData{$a, b$}
\KwResult{$(\lambda, \mu, GCD(a,b))$}
\caption{The Extended Euclidean Algorithm}\label{alg:ExtEucAlg}
$old\_r \gets a$\;
$r \gets b$\;
$old\_s \gets 1$\;
$s \gets 0$\;
$old\_t \gets 0$\;
$t \gets 1$\;
\While{$r \neq 0$}{
    $quotient \gets floor(old\_r / r)$\;
    $old\_r \gets r$\;
    $old\_s \gets s$\;
    $old\_t \gets t$\;

    $r \gets old\_r - quotient \cdot r$\;
    $s \gets old\_s - quotient \cdot s$\;
    $t \gets old\_t - quotient \cdot t$\;
    }
\Return{(s,t,old_r)}
\end{algorithm}
This algorithm has a cost $O(\operatorname{log}^{3}(max\{a, b \}))$.

\subsection{Computation of square and m-th root of n}
The following algorithm can be used to compute efficiently $\lfloor \sqrt[m]{n} \rfloor$.\newline
It is assumed that the length of the result is known and is $l$.\newline
\RestyleAlgo{ruled}
\begin{algorithm}
\KwData{$n, m$}
\KwResult{$\lfloor \sqrt[m]{n} \rfloor$}
\caption{The Efficient m-th root of n}\label{alg:m_root}
$x_{0} \gets 2^{l - 1}$\;
\For{$i \gets 1$ \KwTo $l - 1$}{
    $x_{i} \gets x_{i-1} + 2^{l - i - 1}$\;
    \If{$x_{i}^{m} > n$}{
        $x_{i} \gets x_{i-1}$\;
    }
}
\Return{$x_{l-1}$}
\end{algorithm}
Let's consider the cost of this algorithm:
\begin{itemize}
    \item Computing $x_{i}^{m}$ has cost $O(\operatorname{log}^{2}(n))$
    \item Comparing $x_{i}^{m}$ and $n$ has cost $O(\operatorname{log}(n))$.
    \item The length of the loop is $O(\operatorname{log}(n))$ iterations.
    \item The total cost is therefore $O(\operatorname{log}^{3}(n))$.
\end{itemize}

\subsection{Compute $n, m$ given $n^{m}$}
We can extract the base and the exponent of an integer by making different attempts.\newline
Let's consider the cost of this operation:
\begin{itemize}
    \item We have to make at most $m$ attempts by brute force;
    \item At each attempt we have to compute $\lfloor sqrt[m_{i}]{n^{m}} \rfloor$;
    \item This operation has total cost of \[\sum_{m=3}^{\operatorname{log}(n)} O(\frac{\operatorname{log}(n)}{m} \cdot \operatorname{log}^{2}(m) \cdot \operatorname{log}(m)) + O(\operatorname{log}^{3}(n))\]\footnote{$\frac{\operatorname{log}(n)}{m}$ is the length of the loop};
    \item That is equal to \[O(\operatorname{log}^{3}(n)) \sum_{m=3}^{\operatorname{log}(n)} O(\frac{\operatorname{log}^{3}(m)}{m}) + O(\operatorname{log}^{3}(n))\]
    \item \[\sum_{m=3}^{\operatorname{log}(n)} O(\frac{\operatorname{log}^{3}(m)}{m})\] can be approximated by calculating the correspondant integral, to $O(\operatorname{log}\operatorname{log}(n))$.
    \item The final cost is therefore
    \begin{align*}
        O(\operatorname{log}^{3}(n) \cdot(\operatorname{log}\operatorname{log}(n))^{2})
        = O(\operatorname{log}^{3 + \epsilon}) \text{, with } \epsilon \in (0,1)
    \end{align*}
\end{itemize}


\chapter{Algorithms for primality test}
\section{Miller-Rabin probabilistic primality algorithm}
\RestyleAlgo{ruled}
\begin{algorithm}
\KwData{$n \in \mathbb{N}$, an \emph{odd} number}
\KwResult{$r$}
\caption{Miller-Rabin primality test}\label{alg:miller_rabin_ptest}
Compute $s, d$ such that: $n - 1 = 2^{s} \cdot d$\;
Randomly choose $a \in \mathbb{Z}_{n}^{*}$\;
\If{$(a,n) > 1$}{
    \Return{$n$ is composite}\;
}
$b \gets a^{d} \bmod n$\;
\If{$b \equiv_{n} \pm 1$}{
    \Return{$n$ is prime or spsp}\;
}
$e \gets 0$\;
\While{$b \not\equiv_{n} \pm 1 \land e \leq s-2 $}{
    $b \gets b^{2} \bmod n$
}
\If{$b \not\equiv_{n} 1$}{
    \Return{$n$ is composite}\;
}
\Return{$n$ is prime or spsp}\;
\end{algorithm}
\subsection{Computational complexity of the Miller-Rabin test}
Testing the primality for a single value of $a$ has cost of $O(log^{3}(n))$ b.o.. Although, we could test for each value in $\mathbb{Z}_{n}^{*}$, and that would cost $O(\varphi(n) \cdot log^{3}(n))$ b.o..

\section{Primality Test in $\mathbb{P}$, AKS algorithm}
\subsection{Useful Lemmas}
\begin{lemma}[Newton's formula lemma]\label{newtown_form_lemma}
    This lemma states as follows:\newline
    $n$ is prime $\iff (x + b)^{n} \equiv_{n} x^{n} + b$.
\end{lemma}
\begin{proof}
    The proof is by identity:
    \begin{itemize}
        \item $(x + b)^{n} \equiv_{n} x^{n} + b \implies n$ is prime.
        \begin{itemize}
            \item By \emph{contradiction}:
            \item Assume that $n$ is composite.
            \item Then $\exists p | n$, where $p < n$ is a prime number.
            \item Consider $\binom{n}{p} = \frac{n \cdot n-1 \cdot \dots \cdot n - p + 1}{p \cdot p - 1 \cdot \dots \cdot 1} > 1$
            \item Assume that $p^{\alpha} || n$ then $p \nmid n$
            \item Let $N = \prod_{i = n-p+1}^{n-1} i$.
            \item Let $M = \prod_{i = 1}^{p-1} i$.
            \item Then, $\binom{n}{p} = \frac{p^{\alpha} \cdot N}{p \cdot M} = p^{\alpha - 1} \cdot \frac{N}{M}$.
            \item Since $(N,p) = (M, p) = 1$, then $p^{\alpha - 1} | \binom{n}{p} \land p^{\alpha} \nmid \binom{n}{p}$, therefore $p^{\alpha - 1} || \binom{n}{p}$ and $\binom{n}{p} \equiv_{p} 0$
            \item But this is a \textbf{contradiction}, since $\binom{n}{p} \not\equiv_{p} 0$, therefore $n$ is prime.

        \end{itemize}
        \item $n$ is prime $\implies (x + b)^{n} \equiv_{n} x^{n} + b$
        \begin{itemize}
            \item $p | \binom{p}{k}$, for $k < n$.
            \item Since $\binom{p}{k} \equiv_{p} 0$, then
            \[(x + b)^{p} = \sum_{k=0}^{p} \binom{p}{k} x^{p-k} \cdot b^{k} \equiv_{p} x^{p} + b^{p} \equiv_{p} b\]
        \end{itemize}
    \end{itemize}
\end{proof}

\begin{lemma}[Nair's Lemma]\label{nair_lemma}
    This lemma states as follows:
    \begin{itemize}
        \item Let $m \geq 7 \in \mathbb{Z}$
        \item Let $LCM(x,y)$ be the Least Common Multiplier of $x$ and $y$.
        \item Let $n \leq m \in \mathbb{Z}$.
        \item Then, $LCM(m,n) \geq 2^{m}$.
    \end{itemize}
\end{lemma}

\begin{lemma}[AKS Lemma]\label{aks_lemma}
    This lemma states as follows:
    \begin{itemize}
        \item Let $n \geq 4$
        \item Then, $\exists r \leq \lceil log_{2}^{5}(n) \rceil$ such that $d = ord(n)_{\mathbb{Z}_n^*} > log_2^2(n)$.
    \end{itemize}
\end{lemma}
\begin{proof}
    The proof is by \emph{contradiction}:
    \begin{itemize}
        \item Let $n \geq 4 \Rightarrow \lceil log_{2}^{5}(n) \rceil \geq 32$.
        \item Let $V$ be $\lceil log_{2}^{5}(n) \rceil$.
        \item Let \[\Pi \text{be} n^{\lfloor log_{2}(V) \rfloor} \cdot \prod_{i=1}^{\lceil log_{2}^{2}(n) \rceil}(n^{i} - 1) \].
        \item Let $\nu$ be $\{s \in \{\, \dots, \nu\}: s \nmid \Pi\}$
        \item Assume by contradiction that$ \nu = 0 $:
        \begin{itemize}
            \item Then, by definition: $\forall s \in \nu: s \nmid \Pi$.
            \item Consider that $lcm\{1, \dots, V\} | \Pi$
            \item Consider that:
            \begin{align}
                \Pi \leq n^{\lfloor log_{2}(V) \rfloor} \cdot \prod_{i=1}^{\lceil log_{2}^{2}(n) \rceil} n^{i} & =\\
                n^{\lfloor log_{2}(V) \rfloor + \sum_{i=1}^{\lfloor log_{2}^{2}(n) \rfloor} i} & =  \\
                n^{\lfloor log_{2}(V) \rfloor + \frac{1}{2} \lfloor log_{2}^{2}(n) \rfloor \cdot (\lfloor log_{2}^{2}(n) \rfloor + 1)}
                & < \lfloor (log_{2}(n))^{4} \rfloor \\
                & = 2^{log_{2}(n) \cdot \lfloor (log_{2}(n))^{4} \rfloor} \\
                & < 2^{log_{2}^{5}(n)} \\
                & < 2^{V}
            \end{align}

            \item So, $lcm\{1, \dots, V\} | \Pi \implies lcm\{1, \dots, V\} \leq \Pi < 2^{V}$
            \item Since $V \geq 32$ and $lcm\{1, \dots, V\} \geq 2^{V}$ due to Lemma\ref{nair_lemma}, we have a \textbf{contradiction}.
            \item Therefore, $\nu \neq 0$
        \end{itemize}
        \item Let then $r$ be $\operatorname{min}(\nu) \geq 2$
        \item Assume that $q$ is a prime and $q | r$
        \begin{itemize}
            \item Consider also that $r | V$, since $r \leq V \implies q^{\alpha} | V \implies \alpha \leq \lfloor log_{2}(V) \rfloor$
        \end{itemize}
        \item Assume also that every $q | r$, also $q | n$.
        \item Then, $r = \prod_{q | r} q^{\alpha} | \prod_{q | r} q^{\lfloor log_{2}(V) \rfloor} | \prod_{q | n} q^{\lfloor log_{2}(V) \rfloor}$,  where $p$ is a prime number.
        \item Let $n$ be $\prod_{q | n} q^{\beta}$, where $\beta \geq 1$
        \item Then, we have $n^{\lfloor log_{2}(V) \rfloor} = \prod_{q | n} q^{\beta \cdot \lfloor log_{2}(V) \rfloor}$
        \item Before, we proved:
        \[r | \prod_{q | n} q^{\lfloor log_{2}(V) \rfloor} | n^{\lfloor log_{2}(V) \rfloor} | \Pi\]
        \item Therefore, $r | \Pi$, but $r \in \nu$, so $r \nmid \Pi$, that is a \textbf{Contradiction}.
        \item Then not every prime divisor of $n$ is a prime divisor of $r$.
        \item Consider that $\frac{r}{(r,n)} \in \nu \implies \frac{r}{(r,n)} \leq r = \operatorname{min}(\nu) \implies \frac{r}{(r,n)} = r \implies \frac{r}{(r,n)} = 1$
        \begin{itemize}
            \item By \emph{contradiction}:
            \item Assume that $\frac{r}{(r,n)} \not\in \nu$
            \item Then, $\frac{r}{(r,n)} | \Pi$
            \item Let $r = \prod_{p|r} p^{\alpha}$
            \item If $p | r \land p \nmid n \implies p | \frac{r}{(r,n)} \implies p^{\alpha} | \frac{r}{(r,n)}$
            \item But, if $\frac{r}{(r,n)} | \Pi \implies p^{\alpha} | \prod_{i=1}^{\dots} (n^{i} - 1) | \Pi \implies r | \Pi$, that is a \textbf{contradiction}. Therefore, $\frac{r}{(r,n)} \in \nu$
        \end{itemize}
        \item So, $\operatorname{ord}(n)_{\mathbb{Z}_{r}^{*}} > \lfloor log_{2}^{2}(n) \rfloor$
        \begin{itemize}
            \item By \emph{contradiction}:
            \item $\exists i \leq \lfloor log_{2}^{2}(n) \rfloor$ such that $n^{i} \equiv_{r} = 1$
            \item $\implies r | \prod_{i = 1}^{\lfloor log_{2}^{2}(n) \rfloor}(n^{i} - 1) | \Pi$, but this is a \textbf{contradiction}.
        \end{itemize}

    \end{itemize}
\end{proof}

\subsection{Agrawal - Kayal - Saxema Theorem}
\begin{theorem}[Agrawal - Kayal - Saxema Theorem]\label{aks_theorem}
    Let $n \geq 4 \in \mathbb{N}$, and let $0 < r < n$ such that $(n,r) = 1$ and $order(n) > (log_{2}(n))^{2}$. Then:
    \[
    n \text{is prime} \iff
         \begin{cases}
           n \text{is not a perfect power} \\
           \not\exists p \leq r \\
           (x + b)^{n} \equiv_{(n, x^{r} - 1)} x^{n} + b \text{\, for every} b \in \mathbb{N} \text{ s.t. } 1 \leq b \leq \sqrt{n} \cdot log_{2}(n)
         \end{cases}
    \]
\end{theorem}

\subsection{AKS algorithm}
\subsubsection{Pseudocode}
\begin{algorithm}
    \KwData{$n \in \mathbb{N}$}
    \KwResult{$TRUE$ if $n$ is prime}
    \caption{AKS primality test pseudocode}\label{alg:aks_pscd_ptest}
    \If{$n = \alpha^{\beta} \text{, where } \alpha \text{, } \beta > 1 \in \mathbb{N}$}{\Return{$FALSE$}}\label{step_1_aks}
    $r \gets \argmin_{x} (x,n) = 1$\;
    $d \gets ord(n)_{\mathbb{Z}_{r}^{*}} > \lceil log_{2}^{2}(n) \rceil$\;\label{step_2_aks}
    \If{$\exists b \leq r : 1 < (b,n) < n$}{\Return{$FALSE$}}\label{step_3_aks}
    \If{$n \leq r$}{\Return{$TRUE$}}\label{step_4_aks}
    \If{$\exists b \in \mathbb{N}: 1 \leq b \leq \sqrt{r} \cdot log_{2}(n) \land (x + b)^{n} \not\equiv_{(x^{r} - 1, n)} x^{n} + b$}{\Return{$FALSE$}}\label{step_5_aks}
    \Return{$TRUE$}\;\label{step_6_aks}
\end{algorithm}

\subsubsection{Correctness}
\begin{theorem}[Correctness of the AKS algorithm]
    The AKS algorithm for the primality test of a given number $n$ is correct.\newline
    $n$ is prime $\iff$ the algorithm returns $TRUE$.
\end{theorem}
\begin{proof}
    The proof examines the execution case by case.\newline
    \begin{itemize}
        \item Let's assume that $n$ is prime:
        \begin{itemize}
            \item Then, the algorithm cannot stop at step \ref{step_1_aks} or at step \ref{step_3_aks}.
            \item By Lemma \ref{newtown_form_lemma}, $ (x + b)^{n} \equiv_{n} x^{n} + b \forall b \in \mathbb{Z} \therefore$ the algorithm cannot terminate at step \ref{step_5_aks}.
            \item Therefore, the algorithm can only terminate at step \ref{step_4_aks} or \ref{step_6_aks}, so: $n$ is prime $\implies$ the algorithm returns $TRU£$.
        \end{itemize}
        \item Let's assume that the algorithm returns $TRUE$:
        \begin{itemize}
            \item Then, the algorithm has terminated at step \ref{step_4_aks} or \ref{step_6_aks}.
            \item If the algorithm has terminated at step \ref{step_4_aks}, then $n \leq r$. Since we checked at \ref{step_3_aks} that $(b,n)$ is trivial $\forall b \leq r$, then $n$ has no trivial divisors, hence it's prime.
            \item If the algorithm has terminated at step \ref{step_6_aks}, let's consider that at \ref{step_1_aks} and at \ref{step_3_aks} we verified that condition $1$ and $3$ of the Theorem \ref{aks_theorem} hold, respectively.
            \item Then, it's verified that: the algorithm returns $TRUE \Rightarrow n$ is prime.
        \end{itemize}
        \item Therefore, is verified that $n$ is prime $\iff$ the algorithm returns $TRUE$.
    \end{itemize}
\end{proof}

\subsubsection{Complexity}
\begin{theorem}[Complexity of the AKS algorithm]
    The AKS algorithm sustains the following costs for each step:
    \begin{itemize}
        \item Step \ref{step_1_aks} (checking if $n = a^{b}$) costs $O((log(n))^{3 + \epsilon})$ b.o..
        \item Step \ref{step_2_aks} (picking $r$) has cost of $O((log(n))^{7 + \epsilon})$ b.o., in the worst case.
        \item Step \ref{step_3_aks} (computing $(b,n)$ multiple times) has cost of $O((log(n))^{7 + \epsilon})$ b.o., in the worst case.
        \item Step \ref{step_5_aks} (verifying that $(x + b)^{n} \not\equiv_{n} x^{n} + b$) has cost of $O(log(n) \cdot log(r))$ b.o.
        \item The total cost of the AKS algorithm is then $O(r^{5/2} log^{4}(n))$ b.o.
    \end{itemize}
\end{theorem}


\chapter{Factoring Problem}
\section{Factoring Problem}
This section explores the state of the art on the factoring problem, that is at the base of the attacks of numerous crypto algorithms. \newline
\subsection{Eratostene's sieve}
In ancient times, a Greek mathematician named Eratostene came up with an intuitive factoring method, based on the extraction of the prime numbers up to a given $N$.
\RestyleAlgo{ruled}
\begin{algorithm}
\KwData{$N \in \mathbb{N}$}
\KwResult{$c$, the list of booleans that represents the prime integers up to $N$}
\caption{Eratostene's sieve}\label{alg:eratostene_sieve}
$c[N] \gets \{True * N\}$\;
$p \gets 2$;
\While{$p^{2} \leq N$}{
    $n \gets p^{2}$\;
    \While{$n \leq N$}{
        $c[N] \gets False$\;
        $n \gets p + n$\;
    }
    \Repeat{$c[p] = True$}{
    $p \gets p + 1$\;
    }
}
\Return{$c$}\;
\end{algorithm}
Let's consider the cost of this algorithm:
\begin{itemize}
    \item For each $p \leq N^{1/2}$, one squaring operation is executed, and
    \[
    l_{p} = \lfloor \frac{N}{p} \rfloor - p \leq \frac{N}{p}
    \]
    sums are executed.
    \item Since $p^{2} + kp \leq N$ for $k \leq l_{p}$, then:
    \begin{align*}
        \sum_{p \leq N^{1/2}} (\operatorname{log}^{2}(p) + \sum_{k=1}^{l_{p}} \operatorname{log}(p^{2} + kp)) & \leq \\
        \sum_{p \leq N^{1/2}} (\operatorname{log}^{2}(p) + \operatorname{log}(N) \sum_{p \leq N^{1/2}} \frac{N}{p})\\
        & = N \operatorname{log}(N) \sum_{p \leq N^{1/2}} \frac{1}{p} + O(N^{1/2} \operatorname{log}(N)) \\
        & = O(N \operatorname{log}(N) \operatorname{log}(\operatorname{log}(N)))
    \end{align*}
\end{itemize}

\subsection{Trial division method}
The simplest algorithm to factorize a given number is to proceed by attempts. \newline
This algorithm tries to do it in the most efficient way possible, but it is still less efficient than the methods that will be proposed later.
\RestyleAlgo{ruled}
\begin{algorithm}
\KwData{$N \in \mathbb{N}$, $L$ the list of prime numbers up to $\sqrt{N}$}
\KwResult{$c$, the list of booleans that represents the prime integers up to $N$}
\caption{Trial-division method}\label{alg:trial_division_method}
$c \gets$ \List{empty}\;
\For{$m \in L$}{
    \If{$m | N$}{
        $c \gets $\Append{$c,m$}\; \Comment{Appends the element $m$ to the list $c$}
    }
}
\Return{$c$}\;
\end{algorithm}
If the list of prime numbers up to $\sqrt{N}$ is provided in input, it is necessary to compute, in the worst case, $\sqrt{N}$ reminders, each one at the cost of $\operatorname{log}^{2}(N)$. \newline
Therefore, the cost of this method is $O(\sqrt{N} \operatorname{log}^{2}(N))$.

\subsection{Fermat factoring method}
This method aims to factorize a number $N$ in two numbers $p, q$, such that $N = p \cdot q$. Also, there must be some $x,y$ such that:
\[
N = x^{2} - y^{2} = (x + y)(x - y)
\]
If $N$ is odd it can be shown that $y = \frac{N - 1}{2}, x = \frac{N + 1}{2}$.
\RestyleAlgo{ruled}
\begin{algorithm}
\KwData{$N \in \mathbb{N}$}
\KwResult{$p,q$}
\caption{Fermat's factoring method}\label{alg:fermat_factoring_method}
$y \gets$ 1\;
\Repeat{$y = \frac{N - 1}{2}$}{
    $x \gets N + y^{2}$\;
    \If{$(\lfloor \sqrt{N + y^{2}}\rfloor)^{2} = N + y^{2}$}{
    \Return{$x,y$}
    }\Comment{Checks if $N + y^{2}$ is a perfect square}
    $y \gets y + 1$\;
}
\end{algorithm}
Let's now consider the cost of this algorithm:
\begin{itemize}
    \item Each iteration has a cost of
    \begin{align*}
        O(\operatorname{log}^{2}(y) + \operatorname{log}^{2}(N + y^{2}) + \operatorname{log}^{3}(N + y^{2})) & = \\
        O(\operatorname{log}^{3}(N + y^{2}))
    \end{align*}
    \item The loop is repeated $O(\operatorname{log}^{A}(N))$ times.
    \item Therefore, the complexity of this algorithm is $O(\operatorname{log}^{A + 3}(N))$ b.o..
\end{itemize}

\subsection{Pollard's rho-method}
The Pollard's $\rho$-method tries to factorize a number $N$, by attempting to find a collision when applying multiple times the same function. That can be summarized as follows:
\begin{itemize}
    \item Let $F: \mathbb{Z}_{N}^{*} \rightarrow \mathbb{Z}_{N}^{*}$.
    \item Let $x_{0} \in \mathbb{Z}_{N}^{*}$ be a randomly chosen seed.
    \item Let $F^{(i)}(x_{0}) = F \circ F \circ \dots \circ F(x_{0})$.
    \item We want to find $i,j$ such that $F^{(i)}(x_{0}) \equiv_{N} F^{(j)}(x_{0})$, and compute $(N, F^{(i)}(x_{0}) - F^{(i)}(x_{0}))$
\end{itemize}
Let's now investigate why this algorithm is correct:
\begin{itemize}
    \item Assume that $p$ is prime and that $p|N$.
    \item Build a sequence of $T$ numbers: $\{F_{x_{0}}\}_{k \leq T \in \mathbb{N}}$
    \item Assume that exists a collision, so:
    \begin{align*}
        \exists F^{(i)}(x_{0}) \equiv_{p} F^{(j)}(x_{0}) \\
        \iff p | F^{(i)}(x_{0}) - F^{(j)}(x_{0}) \\
        \iff (p, F^{(i)}(x_{0}) - F^{(j)}(x_{0})) > 1
    \end{align*}
    \item Due to the Birthday Paradox, we know that:
    \begin{align*}
        \mathbb{P}[\exists F^{(i)}(x_{0}) \equiv_{p} F^{(j)}(x_{0})] = \frac{1}{2} \text{ when } T \leq \sqrt{p} \\
        \text{Since } p|N \implies p \leq \sqrt{N} \implies T = O(\sqrt[4]{N})
    \end{align*}
    \item Therefore, probably we will find the collision.
\end{itemize}
The cost of the algorithm is easy to compute:
\begin{itemize}
    \item We have approximately $O(T^{2}) = O(\sqrt{N})$ steps, that is the quantity necessary to make the Birthday Paraodx hypothesis hold;
    \item Each one has a cost of $O(log^{2}(N))$ b.o.
    \item The expected cost is then O$(\sqrt(N) log^{2}(N))$ b.o.
\end{itemize}

\RestyleAlgo{ruled}
\begin{algorithm}
\KwData{$N,T \in \mathbb{N}, F: \mathbb{Z}_{N}^{*} \rightarrow \mathbb{Z}_{N}^{*}$}
\KwResult{List of $d$, non-trivial factors of $N$}
\caption{Pollard's $\rho$-method}\label{alg:pollard_rho_method}

$x_{0}$ is randomly chosen in $\mathbb{Z}_{N}^{*}$\;
$m \gets 1$\;
$y_{1} \gets F(x_{0})$\;
$y_{2} \gets F(y_{1})$\;
\While{$m \leq T$}{
    $d \gets (N, y_{1} - y_{2})$\;
    \If{$d > 1 \land d < N$}{
        \Return{$d$}
    }
    $m \gets$ m +1\;
    $y_{1} \gets F(y_{1})$\;
    $y_{2} \gets F(F(y_{2}))$\;
}
\end{algorithm}

\subsection{Pomerance's quadratic sieve}


\chapter{Attacks}
\section{Definition}
Given:
\begin{itemize}
    \item $\mathcal{l}$, the length of the block;
    \item $\Sigma$, the alphabet;
\end{itemize}
A block cipher is a cryptosystem such that:
\[M = \mathcal{C} = \Sigma^{\mathcal{l}}\]
\begin{proposition}
    Consider waht follows:\newline
    The enciphering function of a block cypher is a permutation of $\Sigma^{\mathcal{l}}$.
\end{proposition}
\begin{proof}
    The proof proceeds as follows:
    \begin{itemize}
        \item Let $\mathcal{f}: \Sigma^{\mathcal{l}} \rightarrow \Sigma^{\mathcal{l}}$.
        \item Let $\mathcal{f}(m) = \mathcal{f}([p_{1}, p_{2}, \dots, p_{e}]) = \Pi(p_{1}, p_{2}, \dots, p_{e})$, where $p_{i} \in \Sigma$.
        \item Then, let $\mathcal{f}^{-1}(c) = \mathcal{f}([c_{1}, c_{2}, \dots, c_{e}]) = \Pi(c_{1}, c_{2}, \dots, c_{e})$, where $c_{i} \in \Sigma$.
        \item Let then $K_{E} = \Pi$ be the set of encyphering keys;
        \item Let then $K_{D} = \Pi^{-1}$ be the set of decyphering keys;
        \item Then, $|K| = (|\Sigma|^{\mathcal{l}})!$
        \item Due to the Pigeonhole principle, if $\mathcal{f}$ is injective, then it's also surjective, and therefore a bijection.
    \end{itemize}
\end{proof}

\section{Enciphering methods}
Since block cyphers encrypt just message of length $\mathcal{l}$, it's necessary to define the different ways in which the blocks can be used in order to compute the ciphertext.
\subsection{Electronic Code Book mode (ECB)}
Consider what follows:
\begin{itemize}
    \item Let $M \in \Sigma^{t}$ be the message, and $t = |M|$.
    \item Let $\mathcal{l}$ be the length of the block, such that $\mathcal{l} \nmid t \land t > \mathcal{l}$.
    \item $M$ is splited in blocks of length $\mathcal{l}$, and the remaining part is filled with garbage. This concatenation is called $M'$.
    \item $M'$ is sent to the receiver.
    \item If $M'$ is received in the correct order, the  message is received correctly.
\end{itemize}

\begin{figure}[h]
    \centering
    \includegraphics[width=0.75\textwidth]{img/ECB.png}
\end{figure}

\subsection{Cipher Block Chaining mode (CBC)}
This method introduces a XOR encryption to the communication. \newline
Consider what follows:
\begin{itemize}
    \item Let $\Sigma = \{0,1\}, M, C \in \{0,1\}^{\mathcal{l}}$
    \item Let $\oplus$ be the XOR operator (as known as the sum in $\mathbb{Z}_{2}$).
    \item Let $P \in \{0,1\}^{\mathcal{l}}$ the initial fixed plaintext, pre-agreed.
    \item Assume that $|M| = k \cdot \mathcal{l}$.
    \item Alice sends the ciphered message $C = [c_{0}, c_{1}, \dots, c_{k}]$ to Bob. That is:
    \begin{align*}
        \begin{cases}
          c_{0} & = \mathcal{f}(P)\\
          c_{j} & = \mathcal{f}(c_{j-1} \oplus m_{j}) \text{, for } 1 \leq j \leq k
        \end{cases}
    \end{align*}
    \item In order to receive correctly the message, it's important that Bob receives correctly $c_{0}$ and computes $\mathcal{f}^{-1}(c_{0})$. If that matches the original $P$, then he can proceed by computing the other blocks:
    \[m_{j} = c_{j-1} \oplus \mathcal{f}^{-1}(c_{j}), 1 \leq j \leq k\]
    This works because:
    \begin{align*}
        c_{1} = \mathcal{f}(c_{0} \oplus m_{i}) & \land c_{0} = \mathcal{f}(c_{0}) \\
        c_{0} \oplus \mathcal{f}^{-1}(c_{1}) & = c_{0} \oplus \mathcal{f}^{-1}(\mathcal{f}(c_{0 \oplus m_{1}})) \\
        & = c_{0} \oplus c_{0} \oplus m_{1} = m_{1}
    \end{align*}
\end{itemize}

\begin{figure}[h]
    \centering
    \includegraphics[width=0.75\textwidth]{img/CBC.png}
\end{figure}


\subsection{Cipher Feedback mode (CFB)}
Let the common knowledge of the communication be:
\begin{itemize}
    \item $P$ an initial value;
    \item $\mathcal{l}$, the length of the block.
    \item $1 \leq r \leq \mathcal{l}$ be a number.
\end{itemize}
Then, let $M = [m_{1}, \dots, m_{t}]$ be the cleartext message, where:
\[|m_{i}| = r \implies r | |M| \]
The communication is executed as follows:
\RestyleAlgo{ruled}
\begin{algorithm}
\caption{Cipher FeedBack Mode communication (CFB) [Sender]}\label{alg:CFB_sender}
$I_{1} \gets P \in \{0,1\}^{\mathcal{l}}$\;
\For{$j \in 1, \dots, t$}{
    $O_{j} \gets \mathcal{f}(I_{j})$\;
    $u_{j} \gets O_{j} \bmod 2^{r}$\;
    $c_{j} \gets m_{j} \oplus u_{j}$\;
    $I_{j+1} \gets 2^{r} I_{j} + c_{j} \bmod 2^{\mathcal{l}}$\;
}
\Return{$C$}
\end{algorithm}

\begin{figure}[h]
    \centering
    \includegraphics[width=0.75\textwidth]{img/CFB.png}
\end{figure}

\RestyleAlgo{ruled}
\begin{algorithm}
\caption{Cipher FeedBack Mode communication (CFB) [Receiver]}\label{alg:CFB_receiver}
$I_{1} \gets P $\;
\For{$j \in 1, \dots, t$}{
    $O_{j} \gets \mathcal{f}(I_{j})$\;
    $u_{j} \gets O_{j} \bmod 2^{r}$\;
    $m_{j} \gets c_{j} \oplus u_{j}$\;
    $I_{j+1} \gets 2^{r} I_{j} + c_{j} \bmod 2^{\mathcal{l}}$\;
}
\Return{$C$}
\end{algorithm}


\subsection{Output Feedback mode (OFB)}
Each output feedback block cipher operation depends on all previous ones, and so cannot be performed in parallel. However, because the plaintext or ciphertext is only used for the final XOR, the block cipher operations may be performed in advance, allowing the final step to be performed in parallel once the plaintext or ciphertext is available.
\begin{figure}[h]
    \centering
    \includegraphics[width=0.75\textwidth]{img/OFB.png}
\end{figure}

\section{Feistel's Ciphers}
The Feistel's cipher cryptosystem is the predecessor of the DES. It is defined as follows:
\begin{align*}
    \Sigma = \{0,1\} \\
    \mathcal{M} = \mathcal{C} = \mathcal{K} = \Sigma^{\mathcal{l}}\\
    \mathcal{f}_{k}: \Sigma^{\mathcal{l}} \rightarrow \Sigma^{\mathcal{l}}\\
    \mathcal{F}_{k}: \Sigma^{2\mathcal{l}} \rightarrow \Sigma^{2\mathcal{l}}
\end{align*}
This cipher loops the enciphering function $F$ $r$ times, in which the blocks have a lenght of $2 \cdot \mathcal{l}$. $f$ is a sort of internal enciphering function. \newline
There's also a key generating function, that has the following signature:
\[\mathcal{K} \rightarrow \mathcal{K}^{r}\]
The algorithm enciphers as follows:
\begin{enumerate}
    \item The cleartext message $M$ is composed of two parts: $L_{0}, R_{0}$, where $|L_{0}| = |R_{0}| = \mathcal{l}$.
    \item Consider that $K_{i}$ is the $i$-th key produced.
    \item For every $i: 1 \leq i \leq r$, $C_{i}$ is computed:
    \[C_{i} = [L_{i}, R_{i}] = [R_{i-1}, L_{i-1} \oplus f_{K_{i}}(R_{i-1})\]
    \item The message that is sent is $F_{K_{r}} = [R_{r}, L_{r}]$
\end{enumerate}
The deciphering method proceeds as follows:
\begin{enumerate}
    \item Note that the keys must be used in reverse order.
    \item At each step, it is computed:
    \[R_{i} \gets L_{i-1} \oplus f_{K_{i}}(R_{i-1}), L_{i} \gets R_{i-1}\]
    \item Also, note that $f_{K_{i}} = f^{-1}_{K_{i}}$, since it's used with the XOR operator.
\end{enumerate}
An important remark is that the complexity of this algorithm depends on $f_{K}$

\section{Data Encryption Standard (DES)}
\subsection{Single DES}
The DES cryptosystem works as a Feistel's Cipher, where:
\begin{itemize}
    \item $\mathcal{l} = 32$ bits;
    \item $r = 16$ rounds;
\end{itemize}
Let now:
\begin{itemize}
    \item $\pi(b_{1}, \dots, b_{64}) = (b_{58}, b_{50}, b_{42}, \dots, b_{23}, b_{15}, b_{7})$ be a permutation;
    \item $E$ be an expansion;
    \item $S$ be a substitution;
    \item $P$ be a round permutation.
    \item Also, the set of the keys is defined as follows:
    \[\mathcal{K} = \{(b_{0}, b_{1}, \dots, b_{64}) \in \{0,1\}^{64}: \forall j \in \{0, 1, \dots,7\}: \sum_{i=1}^{8} b_{8j + i} \equiv_{2} 1\}\]
\end{itemize}

\RestyleAlgo{ruled}
\begin{algorithm}
\caption{Data Encryption Standard [Encryption]}\label{alg:DES_encrypt}
$M \rightarrow \pi(M) \in \{0,1\}^{64}$\;
\For{$1 \leq i \leq 16$}{
    $[L_{i}, R_{i}] \gets [R_{i-1}, L_{i-1} \oplus P(S(E(R_{i-1} \oplus k_{i})))]$\;
}
$C \gets \pi^{-1}[R_{16}, L_{16}]$\;
\Return{$C$}
\end{algorithm}
The keys are produced according to the following procedure:
\begin{itemize}
    \item To obtain $k_{i} \in \{0,1\}^{48}$ we have to remove the parity bits from the positions $8, 16, 24, 32, 40, 48, 56, 64$.
    \item Then, $k_{0} \gets \hat{k}$ (Then, $\hat{k} \in \{0,1\}^{56}$).
    \item $k_{i}$ is generated from $k_{i-1} = [left_part | right_part]$, in which both $left_part, right_part$ are circular shifted of 1 or 2 bits. Then, 48 bits out of the 56 are chosen. That is why $K_{i}$ is called the \emph{Rotated Key}.
\end{itemize}

The expansion function $E$ is defined as follows:
\[E: \{0,1\}^{32} \rightarrow \{0,1\}^{48}\]
\begin{itemize}
    \item Start with 6 bits and write them:
    \[b_{32}, b_{1}, b_{2}, b_{3}, b_{4}, b_{5};\]
    \item Come back of \textbf{two positions} and write the next line:
    \[b_{4}, b_{5}, b_{6}, b_{7}, b_{8}, b_{9};\]
    \item Repeat until there are no bits, and align.
\end{itemize}

The $S$ function, also called \emph{S-box}, defined as
\[S: \{0,1\}^{48} \rightarrow \{0,1\}^{32}\]
works as follows:
\begin{itemize}
    \item The function collects the output from 8 S-boxes:
    \[S_{j}; \{0,1\}^{6} \rightarrow \{0,1\}^{4}\]
    \item Let $T_{j}$ be a $4 \times 16$ matrix. Then, each cell $T_{a,b}$ can be represented as a couple of addresses with respectively 2 and 4 bits.
    \item Each row of $T$ has then a fixed permutation of $\mathbb{Z}_{16}$.
    \item Given the input $b$, $S_{j}$ returns the cell which row is at the address $(b_{1}b_{6}, b_{2}b_{3}b_{4}b_{5})$.
\end{itemize}
This is the main point of strength of the method, but if the S-boxes are not conserved properly, then the protocol is not safe anymore.

\RestyleAlgo{ruled}
\begin{algorithm}
\caption{Data Encryption Standard [Decryption]}\label{alg:DES_decrypt}
The keys are used in the inverse order $k_{16}, k_{15}, \dots, k_{1}$\;
\For{$1 \leq i \leq 16$}{
    $[R_{i-1}, L_{i-1}] \gets [L_{i}, R_{i} \oplus P(S(E(L_{i} \oplus k_{i})))]$\;
}
$M \gets [L_{1}, R_{1}]$\;
\Return{$M$}
\end{algorithm}

\subsection{Triple DES}
The Triple DES protocol uses three level of encryption, by adopting two different keys $k_{1} \neq k_{2}$.\newline
Let $E_{k_{i}}$ be the enciphering function of the Single DES, and let $D_{k_{i}}$ be the correspondent deciphering function. Then, the Triple DES enciphering function works as follows (EDE scheme):
\[m \rightarrow m_{1} = E_{k_{1}}(m) \rightarrow m_{2} = D_{k_{2}}(m_{1}) \rightarrow c = E_{k_{1}}(m_{2})\]
The deciphering function works analoguely (DED scheme):
\[c \rightarrow c_{1} = D_{k_{1}}(c) \rightarrow c_{2} = E_{k_{2}}(c_{1}) \rightarrow m = D_{k_{1}}(c_{2})\]
This means that by using $k_{1}, k_{2}$ we are using 112 bits for the key. \newline
Remark that Triple DES can be used as the Single DES, by picking $k_{1} = k_{2} = k$.

\section{Advanced Encryption Standard (AES)}
\subsection{Cryptosystem description}
The AES cryptosystem is defined as follows:
\begin{align*}
    \Sigma = \{0,1\} \\
    \mathcal{M} = \mathcal{C} = \Sigma^{128}\\
    \mathcal{K} =
    \begin{cases}
        \Sigma^{128} \text{ with AES128}\\
        \Sigma^{192} \text{ with AES192}\\
        \Sigma^{256} \text{ with AES256}\\
    \end{cases}\\
    r =
    \begin{cases}
        10 \text{ with AES128}\\
        12 \text{ with AES192}\\
        14 \text{ with AES256}\\
    \end{cases}
\end{align*}
The following auxiliary functions are defined:
\begin{itemize}
    \item $E$, the expansion function;
    \item $S$, the substitution function;
    \item $SR$, the row-shifting function;
    \item $MC$, the column-mixing function.
\end{itemize}

\subsection{AES arithmetic}
This cryptosystem uses the $\mathbb{F}_{256}$ arithmetic, that is a finite field: \[\mathbb{F}_{256} = \frac{\mathbb{F}_{2}[x]}{x^{8} + x^{4} + x^{3} + x + 1}\]
This means that each value is transformed in $\bmod (x^{8} + x^{4} + x^{3} + x + 1)$. This allows to implement some operation in a very efficient way, from the computational point of view and also from an hardware implementation point of view.\newline
Let's consider the polynomial $x^{8}$: if we transform it in the $\mathbb{F}_{256}$ field, we have that:
\[x^{8} \equiv x^{4} + x^{3} + x + 1\]
Also, we can consider just the coefficient of this result:
\[x^{8} \equiv 00011011_{2} = 1B_{16}\]
This result proves to be useful when we try to compute $\alpha \cdot x$, with $\alpha \in \mathbb{F}_{256}$:
\begin{itemize}
    \item Consider that $\alpha = b_{0} + b_{1} x + b_{2} x^{2} + \dots + b_{7} x^{7}$;
    \item Then, $\alpha \cdot x = b_{0} x + b_{1} x^{2} + b_{2} x^{3} + \dots + b_{7} x^{8}$;
    \item If we consider now the bit representation of $\alpha$ and $x$, we can observe that $\alpha = b_{0}b_{1}b_{2}\dots b_{7}, x = 00000010_{2}$, and therefore $\alpha \cdot x = (\alpha << 1) \oplus (1B)_{16}$.
    \item Also, we can easily compute the successive powers of $\alpha \cdot x^{i}$ by iterating that operation:
    \begin{itemize}
        \item $\alpha \cdot x^{2} = \alpha \cdot x \cdot x$. So, let $\beta = \alpha \cdot x \iff \alpha \cdot x^{2} = \beta \cdot x = (\beta << 1) \oplus (1B)_{16}$
    \end{itemize}
\end{itemize}

\subsection{Enciphering and Deciphering functions}
\RestyleAlgo{ruled}
\begin{algorithm}
\caption{Advanced Encryption Standard [Encryption]}\label{alg:AES_encrypt}
$(K_{0}, K_{1}, \dots, K_{10}) \gets E(k)$\;
$s \gets m \oplus K_{0}$\;
\For{$r = 1, \dots, 10$}{
    $s \gets S(s)$\;
    $s \gets SR(s)$\;
    \If{$r \leq 9$}{
        $s \gets MC(s)$\;
    }
    $s \gets s \oplus K_{r}$\;
}
\Return{$s$}
\end{algorithm}

In the decryption algorithm of AES, the round keys, and the operations as well, are used in the inverse order.
\RestyleAlgo{ruled}
\begin{algorithm}
\caption{Advanced Encryption Standard [DEcryption]}\label{alg:AES_decrypt}
$s \gets c \oplus K_{10}$\;
\For{$r = 10, \dots, 1$}{
    \If{$r > 9$}{
        $s \gets MC^{-1}(s)$\;
    }
    $s \gets SR^{-1}(s)$\;
    $s \gets S^{-1}(s)$\;
    $s \gets s \oplus K_{r}$\;
}
\Return{$s$}
\end{algorithm}

\subsection{Auxiliary functions}
\subsubsection{$E$, the expansion function}
This function serves the purpose of creating the round keys.\newline
This function uses \emph{round constants}, called $C_{i}$. These constants are composed as follows:
\[
C_{i} = [x^{i-1} | (00)_{16}| (00)_{16}| (00)_{16}]
\]
Where $x^{i-1} \in \frac{\mathbb{F}[x]}{x^{8} + x^{4} + x^{3} + 1}$. The first byte of each key contains the binary representation of the monomial $x^{i-1}$ in the aforementioned field.\newline
Consider now that the input of this function is the key $k$, that is composed of 4 words (recall that each word is 4 bytes). The function $\Pi_{S}^{'}$ applies the function $\Pi_{S}$ to each word of the input, separately. That is:
\[
\Pi_{S}^{'}(k[0],k[1],k[2],k[3]) \rightarrow (\Pi_{S}(k[0]),\Pi_{S}(k[1]),\Pi_{S}(k[2]),\Pi_{S}(k[3]))
\]
\RestyleAlgo{ruled}
\begin{algorithm}
\caption{AES expansion function ($E$)}\label{alg:AES_expansion}
$k_{0} \gets k$\;
\For{$j = 1, \dots, 10$}{
    $k_{i}[0] \gets k_{j-1}[0] \oplus C_{j} \oplus \Pi_{S}^{'}(k_{j-1}[3] << 8 )$\;
    \For{$i = 1, \dots, 3$}{
        $k_{j}[i] \gets k_{j-1}[i] \oplus k_{j}[i-1]$\;
    }
}
\Return{$(k_{0}, k_{1}, \dots, k_{10})$}
\end{algorithm}

\subsubsection{$S$, the substitution function}
This function serves a function similar to the S-boxes of the DES algorithm.\newline
Consider what follows:
\begin{itemize}
    \item Let $\operatorname{inv}: \mathbb{F}_{256}^{*} \rightarrow \mathbb{F}_{256}^{*}$ be the function that returns the inverse of the input in $\mathbb{F}_{256}^{*}$. Remark that $\operatorname{inv} = \operatorname{inv}^{-1}$, also that $\operatorname{inv}(0) = 0$ by definition.
    \item Let $\sigma: \mathbb{F}_{256} \rightarrow \mathbb{F}_{256}$ be an \textbf{affine transformation}: $\sigma(b_{0}b_{1}\dots b_{7}) = A \cdot (b_{0}b_{1}\dots b_{7}) + V$, where $A$ is a fixed matrix and $V$ is a fixed vector.
    \item Let $\Pi_{S} = \sigma \circ \operatorname{inv}: \mathbb{F}_{256} \rightarrow \mathbb{F}_{256}$. Since this is a function over a finite set, it's possible to implement it with a $16 \times 16$ matrix, where each entry's address is represented by a tuple $b_{0}b_{1}b_{2}b_{3}, b_{4}b_{5}b_{6}b_{7}$.
    \item Then $S: \{0, 1\}^{128} \rightarrow \{0, 1\}^{128}$ is defined as $S(\alpha_{0}, \alpha_{1}, \dots, \alpha_{15}) = (\Pi_{S}(\alpha_{0}), \Pi_{S}(\alpha_{1}), \dots, \Pi_{S}(\alpha_{15}))$, where $\alpha_{i}$ is the $i$-th byte of the input.
\end{itemize}
\subsubsection{$SR$, the row-shifting function}
This function serves the purpose of mixing the rows' content.\newline
Let: \[S = \begin{pmatrix}
s_{0} & s_{4} & s_{8} & s_{12} \\
s_{1} & s_{5} & s_{9} & s_{13} \\
s_{2} & s_{6} & s_{10} & s_{14} \\
s_{3} & s_{7} & s_{11} & s_{15}
\end{pmatrix}\]
The $SR$ function works as follows:
\begin{itemize}
    \item The first row is untouched.
    \item The second row is \textbf{round-shifted 1 byte to the left}.
    \item The third row is \textbf{round-shifted 2 bytes to the left}.
    \item The fourth row is \textbf{round-shifted 3 bytes to the left}.
\end{itemize}
Then: \[SR(S) = \begin{pmatrix}
s_{0} & s_{4} & s_{8} & s_{12} \\
s_{13} & s_{1} & s_{5} & s_{9} \\
s_{10} & s_{14} & s_{2} & s_{6} \\
s_{7} & s_{11} & s_{15} & s_{3}
\end{pmatrix}\]

\subsubsection{$MC$, the column-mixing function}
This function serves the purpose of mixing the columns' content.\newline
$MC$ is defined as $MC(\alpha_{0}, \alpha_{1}, \alpha_{2}, \alpha_{3}) = M \cdot (\alpha_{0}, \alpha_{1}, \alpha_{2}, \alpha_{3})$, where $M$ is a fixed matrix and $\alpha_{i}$ is the $i$-th word of the input.
Consider that if $M$ is invertible, then also $MC$ is invertible.


\chapter{Digital Signature}
\section{Definition}
Given:
\begin{itemize}
    \item $\mathcal{l}$, the length of the block;
    \item $\Sigma$, the alphabet;
\end{itemize}
A block cipher is a cryptosystem such that:
\[M = \mathcal{C} = \Sigma^{\mathcal{l}}\]
\begin{proposition}
    Consider waht follows:\newline
    The enciphering function of a block cypher is a permutation of $\Sigma^{\mathcal{l}}$.
\end{proposition}
\begin{proof}
    The proof proceeds as follows:
    \begin{itemize}
        \item Let $\mathcal{f}: \Sigma^{\mathcal{l}} \rightarrow \Sigma^{\mathcal{l}}$.
        \item Let $\mathcal{f}(m) = \mathcal{f}([p_{1}, p_{2}, \dots, p_{e}]) = \Pi(p_{1}, p_{2}, \dots, p_{e})$, where $p_{i} \in \Sigma$.
        \item Then, let $\mathcal{f}^{-1}(c) = \mathcal{f}([c_{1}, c_{2}, \dots, c_{e}]) = \Pi(c_{1}, c_{2}, \dots, c_{e})$, where $c_{i} \in \Sigma$.
        \item Let then $K_{E} = \Pi$ be the set of encyphering keys;
        \item Let then $K_{D} = \Pi^{-1}$ be the set of decyphering keys;
        \item Then, $|K| = (|\Sigma|^{\mathcal{l}})!$
        \item Due to the Pigeonhole principle, if $\mathcal{f}$ is injective, then it's also surjective, and therefore a bijection.
    \end{itemize}
\end{proof}

\section{Enciphering methods}
Since block cyphers encrypt just message of length $\mathcal{l}$, it's necessary to define the different ways in which the blocks can be used in order to compute the ciphertext.
\subsection{Electronic Code Book mode (ECB)}
Consider what follows:
\begin{itemize}
    \item Let $M \in \Sigma^{t}$ be the message, and $t = |M|$.
    \item Let $\mathcal{l}$ be the length of the block, such that $\mathcal{l} \nmid t \land t > \mathcal{l}$.
    \item $M$ is splited in blocks of length $\mathcal{l}$, and the remaining part is filled with garbage. This concatenation is called $M'$.
    \item $M'$ is sent to the receiver.
    \item If $M'$ is received in the correct order, the  message is received correctly.
\end{itemize}

\begin{figure}[h]
    \centering
    \includegraphics[width=0.75\textwidth]{img/ECB.png}
\end{figure}

\subsection{Cipher Block Chaining mode (CBC)}
This method introduces a XOR encryption to the communication. \newline
Consider what follows:
\begin{itemize}
    \item Let $\Sigma = \{0,1\}, M, C \in \{0,1\}^{\mathcal{l}}$
    \item Let $\oplus$ be the XOR operator (as known as the sum in $\mathbb{Z}_{2}$).
    \item Let $P \in \{0,1\}^{\mathcal{l}}$ the initial fixed plaintext, pre-agreed.
    \item Assume that $|M| = k \cdot \mathcal{l}$.
    \item Alice sends the ciphered message $C = [c_{0}, c_{1}, \dots, c_{k}]$ to Bob. That is:
    \begin{align*}
        \begin{cases}
          c_{0} & = \mathcal{f}(P)\\
          c_{j} & = \mathcal{f}(c_{j-1} \oplus m_{j}) \text{, for } 1 \leq j \leq k
        \end{cases}
    \end{align*}
    \item In order to receive correctly the message, it's important that Bob receives correctly $c_{0}$ and computes $\mathcal{f}^{-1}(c_{0})$. If that matches the original $P$, then he can proceed by computing the other blocks:
    \[m_{j} = c_{j-1} \oplus \mathcal{f}^{-1}(c_{j}), 1 \leq j \leq k\]
    This works because:
    \begin{align*}
        c_{1} = \mathcal{f}(c_{0} \oplus m_{i}) & \land c_{0} = \mathcal{f}(c_{0}) \\
        c_{0} \oplus \mathcal{f}^{-1}(c_{1}) & = c_{0} \oplus \mathcal{f}^{-1}(\mathcal{f}(c_{0 \oplus m_{1}})) \\
        & = c_{0} \oplus c_{0} \oplus m_{1} = m_{1}
    \end{align*}
\end{itemize}

\begin{figure}[h]
    \centering
    \includegraphics[width=0.75\textwidth]{img/CBC.png}
\end{figure}


\subsection{Cipher Feedback mode (CFB)}
Let the common knowledge of the communication be:
\begin{itemize}
    \item $P$ an initial value;
    \item $\mathcal{l}$, the length of the block.
    \item $1 \leq r \leq \mathcal{l}$ be a number.
\end{itemize}
Then, let $M = [m_{1}, \dots, m_{t}]$ be the cleartext message, where:
\[|m_{i}| = r \implies r | |M| \]
The communication is executed as follows:
\RestyleAlgo{ruled}
\begin{algorithm}
\caption{Cipher FeedBack Mode communication (CFB) [Sender]}\label{alg:CFB_sender}
$I_{1} \gets P \in \{0,1\}^{\mathcal{l}}$\;
\For{$j \in 1, \dots, t$}{
    $O_{j} \gets \mathcal{f}(I_{j})$\;
    $u_{j} \gets O_{j} \bmod 2^{r}$\;
    $c_{j} \gets m_{j} \oplus u_{j}$\;
    $I_{j+1} \gets 2^{r} I_{j} + c_{j} \bmod 2^{\mathcal{l}}$\;
}
\Return{$C$}
\end{algorithm}

\begin{figure}[h]
    \centering
    \includegraphics[width=0.75\textwidth]{img/CFB.png}
\end{figure}

\RestyleAlgo{ruled}
\begin{algorithm}
\caption{Cipher FeedBack Mode communication (CFB) [Receiver]}\label{alg:CFB_receiver}
$I_{1} \gets P $\;
\For{$j \in 1, \dots, t$}{
    $O_{j} \gets \mathcal{f}(I_{j})$\;
    $u_{j} \gets O_{j} \bmod 2^{r}$\;
    $m_{j} \gets c_{j} \oplus u_{j}$\;
    $I_{j+1} \gets 2^{r} I_{j} + c_{j} \bmod 2^{\mathcal{l}}$\;
}
\Return{$C$}
\end{algorithm}


\subsection{Output Feedback mode (OFB)}
Each output feedback block cipher operation depends on all previous ones, and so cannot be performed in parallel. However, because the plaintext or ciphertext is only used for the final XOR, the block cipher operations may be performed in advance, allowing the final step to be performed in parallel once the plaintext or ciphertext is available.
\begin{figure}[h]
    \centering
    \includegraphics[width=0.75\textwidth]{img/OFB.png}
\end{figure}

\section{Feistel's Ciphers}
The Feistel's cipher cryptosystem is the predecessor of the DES. It is defined as follows:
\begin{align*}
    \Sigma = \{0,1\} \\
    \mathcal{M} = \mathcal{C} = \mathcal{K} = \Sigma^{\mathcal{l}}\\
    \mathcal{f}_{k}: \Sigma^{\mathcal{l}} \rightarrow \Sigma^{\mathcal{l}}\\
    \mathcal{F}_{k}: \Sigma^{2\mathcal{l}} \rightarrow \Sigma^{2\mathcal{l}}
\end{align*}
This cipher loops the enciphering function $F$ $r$ times, in which the blocks have a lenght of $2 \cdot \mathcal{l}$. $f$ is a sort of internal enciphering function. \newline
There's also a key generating function, that has the following signature:
\[\mathcal{K} \rightarrow \mathcal{K}^{r}\]
The algorithm enciphers as follows:
\begin{enumerate}
    \item The cleartext message $M$ is composed of two parts: $L_{0}, R_{0}$, where $|L_{0}| = |R_{0}| = \mathcal{l}$.
    \item Consider that $K_{i}$ is the $i$-th key produced.
    \item For every $i: 1 \leq i \leq r$, $C_{i}$ is computed:
    \[C_{i} = [L_{i}, R_{i}] = [R_{i-1}, L_{i-1} \oplus f_{K_{i}}(R_{i-1})\]
    \item The message that is sent is $F_{K_{r}} = [R_{r}, L_{r}]$
\end{enumerate}
The deciphering method proceeds as follows:
\begin{enumerate}
    \item Note that the keys must be used in reverse order.
    \item At each step, it is computed:
    \[R_{i} \gets L_{i-1} \oplus f_{K_{i}}(R_{i-1}), L_{i} \gets R_{i-1}\]
    \item Also, note that $f_{K_{i}} = f^{-1}_{K_{i}}$, since it's used with the XOR operator.
\end{enumerate}
An important remark is that the complexity of this algorithm depends on $f_{K}$

\section{Data Encryption Standard (DES)}
\subsection{Single DES}
The DES cryptosystem works as a Feistel's Cipher, where:
\begin{itemize}
    \item $\mathcal{l} = 32$ bits;
    \item $r = 16$ rounds;
\end{itemize}
Let now:
\begin{itemize}
    \item $\pi(b_{1}, \dots, b_{64}) = (b_{58}, b_{50}, b_{42}, \dots, b_{23}, b_{15}, b_{7})$ be a permutation;
    \item $E$ be an expansion;
    \item $S$ be a substitution;
    \item $P$ be a round permutation.
    \item Also, the set of the keys is defined as follows:
    \[\mathcal{K} = \{(b_{0}, b_{1}, \dots, b_{64}) \in \{0,1\}^{64}: \forall j \in \{0, 1, \dots,7\}: \sum_{i=1}^{8} b_{8j + i} \equiv_{2} 1\}\]
\end{itemize}

\RestyleAlgo{ruled}
\begin{algorithm}
\caption{Data Encryption Standard [Encryption]}\label{alg:DES_encrypt}
$M \rightarrow \pi(M) \in \{0,1\}^{64}$\;
\For{$1 \leq i \leq 16$}{
    $[L_{i}, R_{i}] \gets [R_{i-1}, L_{i-1} \oplus P(S(E(R_{i-1} \oplus k_{i})))]$\;
}
$C \gets \pi^{-1}[R_{16}, L_{16}]$\;
\Return{$C$}
\end{algorithm}
The keys are produced according to the following procedure:
\begin{itemize}
    \item To obtain $k_{i} \in \{0,1\}^{48}$ we have to remove the parity bits from the positions $8, 16, 24, 32, 40, 48, 56, 64$.
    \item Then, $k_{0} \gets \hat{k}$ (Then, $\hat{k} \in \{0,1\}^{56}$).
    \item $k_{i}$ is generated from $k_{i-1} = [left_part | right_part]$, in which both $left_part, right_part$ are circular shifted of 1 or 2 bits. Then, 48 bits out of the 56 are chosen. That is why $K_{i}$ is called the \emph{Rotated Key}.
\end{itemize}

The expansion function $E$ is defined as follows:
\[E: \{0,1\}^{32} \rightarrow \{0,1\}^{48}\]
\begin{itemize}
    \item Start with 6 bits and write them:
    \[b_{32}, b_{1}, b_{2}, b_{3}, b_{4}, b_{5};\]
    \item Come back of \textbf{two positions} and write the next line:
    \[b_{4}, b_{5}, b_{6}, b_{7}, b_{8}, b_{9};\]
    \item Repeat until there are no bits, and align.
\end{itemize}

The $S$ function, also called \emph{S-box}, defined as
\[S: \{0,1\}^{48} \rightarrow \{0,1\}^{32}\]
works as follows:
\begin{itemize}
    \item The function collects the output from 8 S-boxes:
    \[S_{j}; \{0,1\}^{6} \rightarrow \{0,1\}^{4}\]
    \item Let $T_{j}$ be a $4 \times 16$ matrix. Then, each cell $T_{a,b}$ can be represented as a couple of addresses with respectively 2 and 4 bits.
    \item Each row of $T$ has then a fixed permutation of $\mathbb{Z}_{16}$.
    \item Given the input $b$, $S_{j}$ returns the cell which row is at the address $(b_{1}b_{6}, b_{2}b_{3}b_{4}b_{5})$.
\end{itemize}
This is the main point of strength of the method, but if the S-boxes are not conserved properly, then the protocol is not safe anymore.

\RestyleAlgo{ruled}
\begin{algorithm}
\caption{Data Encryption Standard [Decryption]}\label{alg:DES_decrypt}
The keys are used in the inverse order $k_{16}, k_{15}, \dots, k_{1}$\;
\For{$1 \leq i \leq 16$}{
    $[R_{i-1}, L_{i-1}] \gets [L_{i}, R_{i} \oplus P(S(E(L_{i} \oplus k_{i})))]$\;
}
$M \gets [L_{1}, R_{1}]$\;
\Return{$M$}
\end{algorithm}

\subsection{Triple DES}
The Triple DES protocol uses three level of encryption, by adopting two different keys $k_{1} \neq k_{2}$.\newline
Let $E_{k_{i}}$ be the enciphering function of the Single DES, and let $D_{k_{i}}$ be the correspondent deciphering function. Then, the Triple DES enciphering function works as follows (EDE scheme):
\[m \rightarrow m_{1} = E_{k_{1}}(m) \rightarrow m_{2} = D_{k_{2}}(m_{1}) \rightarrow c = E_{k_{1}}(m_{2})\]
The deciphering function works analoguely (DED scheme):
\[c \rightarrow c_{1} = D_{k_{1}}(c) \rightarrow c_{2} = E_{k_{2}}(c_{1}) \rightarrow m = D_{k_{1}}(c_{2})\]
This means that by using $k_{1}, k_{2}$ we are using 112 bits for the key. \newline
Remark that Triple DES can be used as the Single DES, by picking $k_{1} = k_{2} = k$.

\section{Advanced Encryption Standard (AES)}
\subsection{Cryptosystem description}
The AES cryptosystem is defined as follows:
\begin{align*}
    \Sigma = \{0,1\} \\
    \mathcal{M} = \mathcal{C} = \Sigma^{128}\\
    \mathcal{K} =
    \begin{cases}
        \Sigma^{128} \text{ with AES128}\\
        \Sigma^{192} \text{ with AES192}\\
        \Sigma^{256} \text{ with AES256}\\
    \end{cases}\\
    r =
    \begin{cases}
        10 \text{ with AES128}\\
        12 \text{ with AES192}\\
        14 \text{ with AES256}\\
    \end{cases}
\end{align*}
The following auxiliary functions are defined:
\begin{itemize}
    \item $E$, the expansion function;
    \item $S$, the substitution function;
    \item $SR$, the row-shifting function;
    \item $MC$, the column-mixing function.
\end{itemize}

\subsection{AES arithmetic}
This cryptosystem uses the $\mathbb{F}_{256}$ arithmetic, that is a finite field: \[\mathbb{F}_{256} = \frac{\mathbb{F}_{2}[x]}{x^{8} + x^{4} + x^{3} + x + 1}\]
This means that each value is transformed in $\bmod (x^{8} + x^{4} + x^{3} + x + 1)$. This allows to implement some operation in a very efficient way, from the computational point of view and also from an hardware implementation point of view.\newline
Let's consider the polynomial $x^{8}$: if we transform it in the $\mathbb{F}_{256}$ field, we have that:
\[x^{8} \equiv x^{4} + x^{3} + x + 1\]
Also, we can consider just the coefficient of this result:
\[x^{8} \equiv 00011011_{2} = 1B_{16}\]
This result proves to be useful when we try to compute $\alpha \cdot x$, with $\alpha \in \mathbb{F}_{256}$:
\begin{itemize}
    \item Consider that $\alpha = b_{0} + b_{1} x + b_{2} x^{2} + \dots + b_{7} x^{7}$;
    \item Then, $\alpha \cdot x = b_{0} x + b_{1} x^{2} + b_{2} x^{3} + \dots + b_{7} x^{8}$;
    \item If we consider now the bit representation of $\alpha$ and $x$, we can observe that $\alpha = b_{0}b_{1}b_{2}\dots b_{7}, x = 00000010_{2}$, and therefore $\alpha \cdot x = (\alpha << 1) \oplus (1B)_{16}$.
    \item Also, we can easily compute the successive powers of $\alpha \cdot x^{i}$ by iterating that operation:
    \begin{itemize}
        \item $\alpha \cdot x^{2} = \alpha \cdot x \cdot x$. So, let $\beta = \alpha \cdot x \iff \alpha \cdot x^{2} = \beta \cdot x = (\beta << 1) \oplus (1B)_{16}$
    \end{itemize}
\end{itemize}

\subsection{Enciphering and Deciphering functions}
\RestyleAlgo{ruled}
\begin{algorithm}
\caption{Advanced Encryption Standard [Encryption]}\label{alg:AES_encrypt}
$(K_{0}, K_{1}, \dots, K_{10}) \gets E(k)$\;
$s \gets m \oplus K_{0}$\;
\For{$r = 1, \dots, 10$}{
    $s \gets S(s)$\;
    $s \gets SR(s)$\;
    \If{$r \leq 9$}{
        $s \gets MC(s)$\;
    }
    $s \gets s \oplus K_{r}$\;
}
\Return{$s$}
\end{algorithm}

In the decryption algorithm of AES, the round keys, and the operations as well, are used in the inverse order.
\RestyleAlgo{ruled}
\begin{algorithm}
\caption{Advanced Encryption Standard [DEcryption]}\label{alg:AES_decrypt}
$s \gets c \oplus K_{10}$\;
\For{$r = 10, \dots, 1$}{
    \If{$r > 9$}{
        $s \gets MC^{-1}(s)$\;
    }
    $s \gets SR^{-1}(s)$\;
    $s \gets S^{-1}(s)$\;
    $s \gets s \oplus K_{r}$\;
}
\Return{$s$}
\end{algorithm}

\subsection{Auxiliary functions}
\subsubsection{$E$, the expansion function}
This function serves the purpose of creating the round keys.\newline
This function uses \emph{round constants}, called $C_{i}$. These constants are composed as follows:
\[
C_{i} = [x^{i-1} | (00)_{16}| (00)_{16}| (00)_{16}]
\]
Where $x^{i-1} \in \frac{\mathbb{F}[x]}{x^{8} + x^{4} + x^{3} + 1}$. The first byte of each key contains the binary representation of the monomial $x^{i-1}$ in the aforementioned field.\newline
Consider now that the input of this function is the key $k$, that is composed of 4 words (recall that each word is 4 bytes). The function $\Pi_{S}^{'}$ applies the function $\Pi_{S}$ to each word of the input, separately. That is:
\[
\Pi_{S}^{'}(k[0],k[1],k[2],k[3]) \rightarrow (\Pi_{S}(k[0]),\Pi_{S}(k[1]),\Pi_{S}(k[2]),\Pi_{S}(k[3]))
\]
\RestyleAlgo{ruled}
\begin{algorithm}
\caption{AES expansion function ($E$)}\label{alg:AES_expansion}
$k_{0} \gets k$\;
\For{$j = 1, \dots, 10$}{
    $k_{i}[0] \gets k_{j-1}[0] \oplus C_{j} \oplus \Pi_{S}^{'}(k_{j-1}[3] << 8 )$\;
    \For{$i = 1, \dots, 3$}{
        $k_{j}[i] \gets k_{j-1}[i] \oplus k_{j}[i-1]$\;
    }
}
\Return{$(k_{0}, k_{1}, \dots, k_{10})$}
\end{algorithm}

\subsubsection{$S$, the substitution function}
This function serves a function similar to the S-boxes of the DES algorithm.\newline
Consider what follows:
\begin{itemize}
    \item Let $\operatorname{inv}: \mathbb{F}_{256}^{*} \rightarrow \mathbb{F}_{256}^{*}$ be the function that returns the inverse of the input in $\mathbb{F}_{256}^{*}$. Remark that $\operatorname{inv} = \operatorname{inv}^{-1}$, also that $\operatorname{inv}(0) = 0$ by definition.
    \item Let $\sigma: \mathbb{F}_{256} \rightarrow \mathbb{F}_{256}$ be an \textbf{affine transformation}: $\sigma(b_{0}b_{1}\dots b_{7}) = A \cdot (b_{0}b_{1}\dots b_{7}) + V$, where $A$ is a fixed matrix and $V$ is a fixed vector.
    \item Let $\Pi_{S} = \sigma \circ \operatorname{inv}: \mathbb{F}_{256} \rightarrow \mathbb{F}_{256}$. Since this is a function over a finite set, it's possible to implement it with a $16 \times 16$ matrix, where each entry's address is represented by a tuple $b_{0}b_{1}b_{2}b_{3}, b_{4}b_{5}b_{6}b_{7}$.
    \item Then $S: \{0, 1\}^{128} \rightarrow \{0, 1\}^{128}$ is defined as $S(\alpha_{0}, \alpha_{1}, \dots, \alpha_{15}) = (\Pi_{S}(\alpha_{0}), \Pi_{S}(\alpha_{1}), \dots, \Pi_{S}(\alpha_{15}))$, where $\alpha_{i}$ is the $i$-th byte of the input.
\end{itemize}
\subsubsection{$SR$, the row-shifting function}
This function serves the purpose of mixing the rows' content.\newline
Let: \[S = \begin{pmatrix}
s_{0} & s_{4} & s_{8} & s_{12} \\
s_{1} & s_{5} & s_{9} & s_{13} \\
s_{2} & s_{6} & s_{10} & s_{14} \\
s_{3} & s_{7} & s_{11} & s_{15}
\end{pmatrix}\]
The $SR$ function works as follows:
\begin{itemize}
    \item The first row is untouched.
    \item The second row is \textbf{round-shifted 1 byte to the left}.
    \item The third row is \textbf{round-shifted 2 bytes to the left}.
    \item The fourth row is \textbf{round-shifted 3 bytes to the left}.
\end{itemize}
Then: \[SR(S) = \begin{pmatrix}
s_{0} & s_{4} & s_{8} & s_{12} \\
s_{13} & s_{1} & s_{5} & s_{9} \\
s_{10} & s_{14} & s_{2} & s_{6} \\
s_{7} & s_{11} & s_{15} & s_{3}
\end{pmatrix}\]

\subsubsection{$MC$, the column-mixing function}
This function serves the purpose of mixing the columns' content.\newline
$MC$ is defined as $MC(\alpha_{0}, \alpha_{1}, \alpha_{2}, \alpha_{3}) = M \cdot (\alpha_{0}, \alpha_{1}, \alpha_{2}, \alpha_{3})$, where $M$ is a fixed matrix and $\alpha_{i}$ is the $i$-th word of the input.
Consider that if $M$ is invertible, then also $MC$ is invertible.


\chapter{Cryptography Principles}
\section{Cryptosystems}
\subsection{Definitions}
\begin{definition}[Cryptographic transformation]
    A cryptographic transformation is defined as \textbf{any injective function} $\mathscr{f}$ such that:
    \[\mathscr{f}: \mathcal{M} \rightarrow \mathcal{C}\]
\end{definition}
\begin{definition}[Enciphering and Deciphering functions]
    An \textbf{Enciphering function} is a cryptographic transformation. Its inverse $\mathscr{f}^{-1}$ is called \textbf{Deciphering function} and is defined as:
    \[\mathscr{f}^{-1}: \mathcal{C} \rightarrow \mathcal{M}\]
\end{definition}
\begin{definition}[Cryptosystem]
    A \textbf{Cryptosystem} is a tuple $(\mathcal{M},\mathcal{C}, \mathcal{K}, \mathscr{f}_{k_{e}},\mathscr{f}_{k_{d}}^{-1})$, where:
    \begin{itemize}
        \item $\mathcal{M}$ is the space of clear-text messages;
        \item $\mathcal{C}$ is the space of ciphered messages;
        \item $\mathcal{K}$ is the space of the keys;
        \item $\mathscr{f}_{k_e}$ is the enciphering function with key $k_{e}$;
        \item $\mathscr{f}_{k_d}^{-1}$ is the deciphering function with key $k_{d}$.
    \end{itemize}
\end{definition}


\chapter{Asymmetric Algorithms}
\section{Definition}
Given:
\begin{itemize}
    \item $\mathcal{l}$, the length of the block;
    \item $\Sigma$, the alphabet;
\end{itemize}
A block cipher is a cryptosystem such that:
\[M = \mathcal{C} = \Sigma^{\mathcal{l}}\]
\begin{proposition}
    Consider waht follows:\newline
    The enciphering function of a block cypher is a permutation of $\Sigma^{\mathcal{l}}$.
\end{proposition}
\begin{proof}
    The proof proceeds as follows:
    \begin{itemize}
        \item Let $\mathcal{f}: \Sigma^{\mathcal{l}} \rightarrow \Sigma^{\mathcal{l}}$.
        \item Let $\mathcal{f}(m) = \mathcal{f}([p_{1}, p_{2}, \dots, p_{e}]) = \Pi(p_{1}, p_{2}, \dots, p_{e})$, where $p_{i} \in \Sigma$.
        \item Then, let $\mathcal{f}^{-1}(c) = \mathcal{f}([c_{1}, c_{2}, \dots, c_{e}]) = \Pi(c_{1}, c_{2}, \dots, c_{e})$, where $c_{i} \in \Sigma$.
        \item Let then $K_{E} = \Pi$ be the set of encyphering keys;
        \item Let then $K_{D} = \Pi^{-1}$ be the set of decyphering keys;
        \item Then, $|K| = (|\Sigma|^{\mathcal{l}})!$
        \item Due to the Pigeonhole principle, if $\mathcal{f}$ is injective, then it's also surjective, and therefore a bijection.
    \end{itemize}
\end{proof}

\section{Enciphering methods}
Since block cyphers encrypt just message of length $\mathcal{l}$, it's necessary to define the different ways in which the blocks can be used in order to compute the ciphertext.
\subsection{Electronic Code Book mode (ECB)}
Consider what follows:
\begin{itemize}
    \item Let $M \in \Sigma^{t}$ be the message, and $t = |M|$.
    \item Let $\mathcal{l}$ be the length of the block, such that $\mathcal{l} \nmid t \land t > \mathcal{l}$.
    \item $M$ is splited in blocks of length $\mathcal{l}$, and the remaining part is filled with garbage. This concatenation is called $M'$.
    \item $M'$ is sent to the receiver.
    \item If $M'$ is received in the correct order, the  message is received correctly.
\end{itemize}

\begin{figure}[h]
    \centering
    \includegraphics[width=0.75\textwidth]{img/ECB.png}
\end{figure}

\subsection{Cipher Block Chaining mode (CBC)}
This method introduces a XOR encryption to the communication. \newline
Consider what follows:
\begin{itemize}
    \item Let $\Sigma = \{0,1\}, M, C \in \{0,1\}^{\mathcal{l}}$
    \item Let $\oplus$ be the XOR operator (as known as the sum in $\mathbb{Z}_{2}$).
    \item Let $P \in \{0,1\}^{\mathcal{l}}$ the initial fixed plaintext, pre-agreed.
    \item Assume that $|M| = k \cdot \mathcal{l}$.
    \item Alice sends the ciphered message $C = [c_{0}, c_{1}, \dots, c_{k}]$ to Bob. That is:
    \begin{align*}
        \begin{cases}
          c_{0} & = \mathcal{f}(P)\\
          c_{j} & = \mathcal{f}(c_{j-1} \oplus m_{j}) \text{, for } 1 \leq j \leq k
        \end{cases}
    \end{align*}
    \item In order to receive correctly the message, it's important that Bob receives correctly $c_{0}$ and computes $\mathcal{f}^{-1}(c_{0})$. If that matches the original $P$, then he can proceed by computing the other blocks:
    \[m_{j} = c_{j-1} \oplus \mathcal{f}^{-1}(c_{j}), 1 \leq j \leq k\]
    This works because:
    \begin{align*}
        c_{1} = \mathcal{f}(c_{0} \oplus m_{i}) & \land c_{0} = \mathcal{f}(c_{0}) \\
        c_{0} \oplus \mathcal{f}^{-1}(c_{1}) & = c_{0} \oplus \mathcal{f}^{-1}(\mathcal{f}(c_{0 \oplus m_{1}})) \\
        & = c_{0} \oplus c_{0} \oplus m_{1} = m_{1}
    \end{align*}
\end{itemize}

\begin{figure}[h]
    \centering
    \includegraphics[width=0.75\textwidth]{img/CBC.png}
\end{figure}


\subsection{Cipher Feedback mode (CFB)}
Let the common knowledge of the communication be:
\begin{itemize}
    \item $P$ an initial value;
    \item $\mathcal{l}$, the length of the block.
    \item $1 \leq r \leq \mathcal{l}$ be a number.
\end{itemize}
Then, let $M = [m_{1}, \dots, m_{t}]$ be the cleartext message, where:
\[|m_{i}| = r \implies r | |M| \]
The communication is executed as follows:
\RestyleAlgo{ruled}
\begin{algorithm}
\caption{Cipher FeedBack Mode communication (CFB) [Sender]}\label{alg:CFB_sender}
$I_{1} \gets P \in \{0,1\}^{\mathcal{l}}$\;
\For{$j \in 1, \dots, t$}{
    $O_{j} \gets \mathcal{f}(I_{j})$\;
    $u_{j} \gets O_{j} \bmod 2^{r}$\;
    $c_{j} \gets m_{j} \oplus u_{j}$\;
    $I_{j+1} \gets 2^{r} I_{j} + c_{j} \bmod 2^{\mathcal{l}}$\;
}
\Return{$C$}
\end{algorithm}

\begin{figure}[h]
    \centering
    \includegraphics[width=0.75\textwidth]{img/CFB.png}
\end{figure}

\RestyleAlgo{ruled}
\begin{algorithm}
\caption{Cipher FeedBack Mode communication (CFB) [Receiver]}\label{alg:CFB_receiver}
$I_{1} \gets P $\;
\For{$j \in 1, \dots, t$}{
    $O_{j} \gets \mathcal{f}(I_{j})$\;
    $u_{j} \gets O_{j} \bmod 2^{r}$\;
    $m_{j} \gets c_{j} \oplus u_{j}$\;
    $I_{j+1} \gets 2^{r} I_{j} + c_{j} \bmod 2^{\mathcal{l}}$\;
}
\Return{$C$}
\end{algorithm}


\subsection{Output Feedback mode (OFB)}
Each output feedback block cipher operation depends on all previous ones, and so cannot be performed in parallel. However, because the plaintext or ciphertext is only used for the final XOR, the block cipher operations may be performed in advance, allowing the final step to be performed in parallel once the plaintext or ciphertext is available.
\begin{figure}[h]
    \centering
    \includegraphics[width=0.75\textwidth]{img/OFB.png}
\end{figure}

\section{Feistel's Ciphers}
The Feistel's cipher cryptosystem is the predecessor of the DES. It is defined as follows:
\begin{align*}
    \Sigma = \{0,1\} \\
    \mathcal{M} = \mathcal{C} = \mathcal{K} = \Sigma^{\mathcal{l}}\\
    \mathcal{f}_{k}: \Sigma^{\mathcal{l}} \rightarrow \Sigma^{\mathcal{l}}\\
    \mathcal{F}_{k}: \Sigma^{2\mathcal{l}} \rightarrow \Sigma^{2\mathcal{l}}
\end{align*}
This cipher loops the enciphering function $F$ $r$ times, in which the blocks have a lenght of $2 \cdot \mathcal{l}$. $f$ is a sort of internal enciphering function. \newline
There's also a key generating function, that has the following signature:
\[\mathcal{K} \rightarrow \mathcal{K}^{r}\]
The algorithm enciphers as follows:
\begin{enumerate}
    \item The cleartext message $M$ is composed of two parts: $L_{0}, R_{0}$, where $|L_{0}| = |R_{0}| = \mathcal{l}$.
    \item Consider that $K_{i}$ is the $i$-th key produced.
    \item For every $i: 1 \leq i \leq r$, $C_{i}$ is computed:
    \[C_{i} = [L_{i}, R_{i}] = [R_{i-1}, L_{i-1} \oplus f_{K_{i}}(R_{i-1})\]
    \item The message that is sent is $F_{K_{r}} = [R_{r}, L_{r}]$
\end{enumerate}
The deciphering method proceeds as follows:
\begin{enumerate}
    \item Note that the keys must be used in reverse order.
    \item At each step, it is computed:
    \[R_{i} \gets L_{i-1} \oplus f_{K_{i}}(R_{i-1}), L_{i} \gets R_{i-1}\]
    \item Also, note that $f_{K_{i}} = f^{-1}_{K_{i}}$, since it's used with the XOR operator.
\end{enumerate}
An important remark is that the complexity of this algorithm depends on $f_{K}$

\section{Data Encryption Standard (DES)}
\subsection{Single DES}
The DES cryptosystem works as a Feistel's Cipher, where:
\begin{itemize}
    \item $\mathcal{l} = 32$ bits;
    \item $r = 16$ rounds;
\end{itemize}
Let now:
\begin{itemize}
    \item $\pi(b_{1}, \dots, b_{64}) = (b_{58}, b_{50}, b_{42}, \dots, b_{23}, b_{15}, b_{7})$ be a permutation;
    \item $E$ be an expansion;
    \item $S$ be a substitution;
    \item $P$ be a round permutation.
    \item Also, the set of the keys is defined as follows:
    \[\mathcal{K} = \{(b_{0}, b_{1}, \dots, b_{64}) \in \{0,1\}^{64}: \forall j \in \{0, 1, \dots,7\}: \sum_{i=1}^{8} b_{8j + i} \equiv_{2} 1\}\]
\end{itemize}

\RestyleAlgo{ruled}
\begin{algorithm}
\caption{Data Encryption Standard [Encryption]}\label{alg:DES_encrypt}
$M \rightarrow \pi(M) \in \{0,1\}^{64}$\;
\For{$1 \leq i \leq 16$}{
    $[L_{i}, R_{i}] \gets [R_{i-1}, L_{i-1} \oplus P(S(E(R_{i-1} \oplus k_{i})))]$\;
}
$C \gets \pi^{-1}[R_{16}, L_{16}]$\;
\Return{$C$}
\end{algorithm}
The keys are produced according to the following procedure:
\begin{itemize}
    \item To obtain $k_{i} \in \{0,1\}^{48}$ we have to remove the parity bits from the positions $8, 16, 24, 32, 40, 48, 56, 64$.
    \item Then, $k_{0} \gets \hat{k}$ (Then, $\hat{k} \in \{0,1\}^{56}$).
    \item $k_{i}$ is generated from $k_{i-1} = [left_part | right_part]$, in which both $left_part, right_part$ are circular shifted of 1 or 2 bits. Then, 48 bits out of the 56 are chosen. That is why $K_{i}$ is called the \emph{Rotated Key}.
\end{itemize}

The expansion function $E$ is defined as follows:
\[E: \{0,1\}^{32} \rightarrow \{0,1\}^{48}\]
\begin{itemize}
    \item Start with 6 bits and write them:
    \[b_{32}, b_{1}, b_{2}, b_{3}, b_{4}, b_{5};\]
    \item Come back of \textbf{two positions} and write the next line:
    \[b_{4}, b_{5}, b_{6}, b_{7}, b_{8}, b_{9};\]
    \item Repeat until there are no bits, and align.
\end{itemize}

The $S$ function, also called \emph{S-box}, defined as
\[S: \{0,1\}^{48} \rightarrow \{0,1\}^{32}\]
works as follows:
\begin{itemize}
    \item The function collects the output from 8 S-boxes:
    \[S_{j}; \{0,1\}^{6} \rightarrow \{0,1\}^{4}\]
    \item Let $T_{j}$ be a $4 \times 16$ matrix. Then, each cell $T_{a,b}$ can be represented as a couple of addresses with respectively 2 and 4 bits.
    \item Each row of $T$ has then a fixed permutation of $\mathbb{Z}_{16}$.
    \item Given the input $b$, $S_{j}$ returns the cell which row is at the address $(b_{1}b_{6}, b_{2}b_{3}b_{4}b_{5})$.
\end{itemize}
This is the main point of strength of the method, but if the S-boxes are not conserved properly, then the protocol is not safe anymore.

\RestyleAlgo{ruled}
\begin{algorithm}
\caption{Data Encryption Standard [Decryption]}\label{alg:DES_decrypt}
The keys are used in the inverse order $k_{16}, k_{15}, \dots, k_{1}$\;
\For{$1 \leq i \leq 16$}{
    $[R_{i-1}, L_{i-1}] \gets [L_{i}, R_{i} \oplus P(S(E(L_{i} \oplus k_{i})))]$\;
}
$M \gets [L_{1}, R_{1}]$\;
\Return{$M$}
\end{algorithm}

\subsection{Triple DES}
The Triple DES protocol uses three level of encryption, by adopting two different keys $k_{1} \neq k_{2}$.\newline
Let $E_{k_{i}}$ be the enciphering function of the Single DES, and let $D_{k_{i}}$ be the correspondent deciphering function. Then, the Triple DES enciphering function works as follows (EDE scheme):
\[m \rightarrow m_{1} = E_{k_{1}}(m) \rightarrow m_{2} = D_{k_{2}}(m_{1}) \rightarrow c = E_{k_{1}}(m_{2})\]
The deciphering function works analoguely (DED scheme):
\[c \rightarrow c_{1} = D_{k_{1}}(c) \rightarrow c_{2} = E_{k_{2}}(c_{1}) \rightarrow m = D_{k_{1}}(c_{2})\]
This means that by using $k_{1}, k_{2}$ we are using 112 bits for the key. \newline
Remark that Triple DES can be used as the Single DES, by picking $k_{1} = k_{2} = k$.

\section{Advanced Encryption Standard (AES)}
\subsection{Cryptosystem description}
The AES cryptosystem is defined as follows:
\begin{align*}
    \Sigma = \{0,1\} \\
    \mathcal{M} = \mathcal{C} = \Sigma^{128}\\
    \mathcal{K} =
    \begin{cases}
        \Sigma^{128} \text{ with AES128}\\
        \Sigma^{192} \text{ with AES192}\\
        \Sigma^{256} \text{ with AES256}\\
    \end{cases}\\
    r =
    \begin{cases}
        10 \text{ with AES128}\\
        12 \text{ with AES192}\\
        14 \text{ with AES256}\\
    \end{cases}
\end{align*}
The following auxiliary functions are defined:
\begin{itemize}
    \item $E$, the expansion function;
    \item $S$, the substitution function;
    \item $SR$, the row-shifting function;
    \item $MC$, the column-mixing function.
\end{itemize}

\subsection{AES arithmetic}
This cryptosystem uses the $\mathbb{F}_{256}$ arithmetic, that is a finite field: \[\mathbb{F}_{256} = \frac{\mathbb{F}_{2}[x]}{x^{8} + x^{4} + x^{3} + x + 1}\]
This means that each value is transformed in $\bmod (x^{8} + x^{4} + x^{3} + x + 1)$. This allows to implement some operation in a very efficient way, from the computational point of view and also from an hardware implementation point of view.\newline
Let's consider the polynomial $x^{8}$: if we transform it in the $\mathbb{F}_{256}$ field, we have that:
\[x^{8} \equiv x^{4} + x^{3} + x + 1\]
Also, we can consider just the coefficient of this result:
\[x^{8} \equiv 00011011_{2} = 1B_{16}\]
This result proves to be useful when we try to compute $\alpha \cdot x$, with $\alpha \in \mathbb{F}_{256}$:
\begin{itemize}
    \item Consider that $\alpha = b_{0} + b_{1} x + b_{2} x^{2} + \dots + b_{7} x^{7}$;
    \item Then, $\alpha \cdot x = b_{0} x + b_{1} x^{2} + b_{2} x^{3} + \dots + b_{7} x^{8}$;
    \item If we consider now the bit representation of $\alpha$ and $x$, we can observe that $\alpha = b_{0}b_{1}b_{2}\dots b_{7}, x = 00000010_{2}$, and therefore $\alpha \cdot x = (\alpha << 1) \oplus (1B)_{16}$.
    \item Also, we can easily compute the successive powers of $\alpha \cdot x^{i}$ by iterating that operation:
    \begin{itemize}
        \item $\alpha \cdot x^{2} = \alpha \cdot x \cdot x$. So, let $\beta = \alpha \cdot x \iff \alpha \cdot x^{2} = \beta \cdot x = (\beta << 1) \oplus (1B)_{16}$
    \end{itemize}
\end{itemize}

\subsection{Enciphering and Deciphering functions}
\RestyleAlgo{ruled}
\begin{algorithm}
\caption{Advanced Encryption Standard [Encryption]}\label{alg:AES_encrypt}
$(K_{0}, K_{1}, \dots, K_{10}) \gets E(k)$\;
$s \gets m \oplus K_{0}$\;
\For{$r = 1, \dots, 10$}{
    $s \gets S(s)$\;
    $s \gets SR(s)$\;
    \If{$r \leq 9$}{
        $s \gets MC(s)$\;
    }
    $s \gets s \oplus K_{r}$\;
}
\Return{$s$}
\end{algorithm}

In the decryption algorithm of AES, the round keys, and the operations as well, are used in the inverse order.
\RestyleAlgo{ruled}
\begin{algorithm}
\caption{Advanced Encryption Standard [DEcryption]}\label{alg:AES_decrypt}
$s \gets c \oplus K_{10}$\;
\For{$r = 10, \dots, 1$}{
    \If{$r > 9$}{
        $s \gets MC^{-1}(s)$\;
    }
    $s \gets SR^{-1}(s)$\;
    $s \gets S^{-1}(s)$\;
    $s \gets s \oplus K_{r}$\;
}
\Return{$s$}
\end{algorithm}

\subsection{Auxiliary functions}
\subsubsection{$E$, the expansion function}
This function serves the purpose of creating the round keys.\newline
This function uses \emph{round constants}, called $C_{i}$. These constants are composed as follows:
\[
C_{i} = [x^{i-1} | (00)_{16}| (00)_{16}| (00)_{16}]
\]
Where $x^{i-1} \in \frac{\mathbb{F}[x]}{x^{8} + x^{4} + x^{3} + 1}$. The first byte of each key contains the binary representation of the monomial $x^{i-1}$ in the aforementioned field.\newline
Consider now that the input of this function is the key $k$, that is composed of 4 words (recall that each word is 4 bytes). The function $\Pi_{S}^{'}$ applies the function $\Pi_{S}$ to each word of the input, separately. That is:
\[
\Pi_{S}^{'}(k[0],k[1],k[2],k[3]) \rightarrow (\Pi_{S}(k[0]),\Pi_{S}(k[1]),\Pi_{S}(k[2]),\Pi_{S}(k[3]))
\]
\RestyleAlgo{ruled}
\begin{algorithm}
\caption{AES expansion function ($E$)}\label{alg:AES_expansion}
$k_{0} \gets k$\;
\For{$j = 1, \dots, 10$}{
    $k_{i}[0] \gets k_{j-1}[0] \oplus C_{j} \oplus \Pi_{S}^{'}(k_{j-1}[3] << 8 )$\;
    \For{$i = 1, \dots, 3$}{
        $k_{j}[i] \gets k_{j-1}[i] \oplus k_{j}[i-1]$\;
    }
}
\Return{$(k_{0}, k_{1}, \dots, k_{10})$}
\end{algorithm}

\subsubsection{$S$, the substitution function}
This function serves a function similar to the S-boxes of the DES algorithm.\newline
Consider what follows:
\begin{itemize}
    \item Let $\operatorname{inv}: \mathbb{F}_{256}^{*} \rightarrow \mathbb{F}_{256}^{*}$ be the function that returns the inverse of the input in $\mathbb{F}_{256}^{*}$. Remark that $\operatorname{inv} = \operatorname{inv}^{-1}$, also that $\operatorname{inv}(0) = 0$ by definition.
    \item Let $\sigma: \mathbb{F}_{256} \rightarrow \mathbb{F}_{256}$ be an \textbf{affine transformation}: $\sigma(b_{0}b_{1}\dots b_{7}) = A \cdot (b_{0}b_{1}\dots b_{7}) + V$, where $A$ is a fixed matrix and $V$ is a fixed vector.
    \item Let $\Pi_{S} = \sigma \circ \operatorname{inv}: \mathbb{F}_{256} \rightarrow \mathbb{F}_{256}$. Since this is a function over a finite set, it's possible to implement it with a $16 \times 16$ matrix, where each entry's address is represented by a tuple $b_{0}b_{1}b_{2}b_{3}, b_{4}b_{5}b_{6}b_{7}$.
    \item Then $S: \{0, 1\}^{128} \rightarrow \{0, 1\}^{128}$ is defined as $S(\alpha_{0}, \alpha_{1}, \dots, \alpha_{15}) = (\Pi_{S}(\alpha_{0}), \Pi_{S}(\alpha_{1}), \dots, \Pi_{S}(\alpha_{15}))$, where $\alpha_{i}$ is the $i$-th byte of the input.
\end{itemize}
\subsubsection{$SR$, the row-shifting function}
This function serves the purpose of mixing the rows' content.\newline
Let: \[S = \begin{pmatrix}
s_{0} & s_{4} & s_{8} & s_{12} \\
s_{1} & s_{5} & s_{9} & s_{13} \\
s_{2} & s_{6} & s_{10} & s_{14} \\
s_{3} & s_{7} & s_{11} & s_{15}
\end{pmatrix}\]
The $SR$ function works as follows:
\begin{itemize}
    \item The first row is untouched.
    \item The second row is \textbf{round-shifted 1 byte to the left}.
    \item The third row is \textbf{round-shifted 2 bytes to the left}.
    \item The fourth row is \textbf{round-shifted 3 bytes to the left}.
\end{itemize}
Then: \[SR(S) = \begin{pmatrix}
s_{0} & s_{4} & s_{8} & s_{12} \\
s_{13} & s_{1} & s_{5} & s_{9} \\
s_{10} & s_{14} & s_{2} & s_{6} \\
s_{7} & s_{11} & s_{15} & s_{3}
\end{pmatrix}\]

\subsubsection{$MC$, the column-mixing function}
This function serves the purpose of mixing the columns' content.\newline
$MC$ is defined as $MC(\alpha_{0}, \alpha_{1}, \alpha_{2}, \alpha_{3}) = M \cdot (\alpha_{0}, \alpha_{1}, \alpha_{2}, \alpha_{3})$, where $M$ is a fixed matrix and $\alpha_{i}$ is the $i$-th word of the input.
Consider that if $M$ is invertible, then also $MC$ is invertible.


\chapter{Block Ciphers}
\section{Definition}
Given:
\begin{itemize}
    \item $\mathcal{l}$, the length of the block;
    \item $\Sigma$, the alphabet;
\end{itemize}
A block cipher is a cryptosystem such that:
\[M = \mathcal{C} = \Sigma^{\mathcal{l}}\]
\begin{proposition}
    Consider waht follows:\newline
    The enciphering function of a block cypher is a permutation of $\Sigma^{\mathcal{l}}$.
\end{proposition}
\begin{proof}
    The proof proceeds as follows:
    \begin{itemize}
        \item Let $\mathcal{f}: \Sigma^{\mathcal{l}} \rightarrow \Sigma^{\mathcal{l}}$.
        \item Let $\mathcal{f}(m) = \mathcal{f}([p_{1}, p_{2}, \dots, p_{e}]) = \Pi(p_{1}, p_{2}, \dots, p_{e})$, where $p_{i} \in \Sigma$.
        \item Then, let $\mathcal{f}^{-1}(c) = \mathcal{f}([c_{1}, c_{2}, \dots, c_{e}]) = \Pi(c_{1}, c_{2}, \dots, c_{e})$, where $c_{i} \in \Sigma$.
        \item Let then $K_{E} = \Pi$ be the set of encyphering keys;
        \item Let then $K_{D} = \Pi^{-1}$ be the set of decyphering keys;
        \item Then, $|K| = (|\Sigma|^{\mathcal{l}})!$
        \item Due to the Pigeonhole principle, if $\mathcal{f}$ is injective, then it's also surjective, and therefore a bijection.
    \end{itemize}
\end{proof}

\section{Enciphering methods}
Since block cyphers encrypt just message of length $\mathcal{l}$, it's necessary to define the different ways in which the blocks can be used in order to compute the ciphertext.
\subsection{Electronic Code Book mode (ECB)}
Consider what follows:
\begin{itemize}
    \item Let $M \in \Sigma^{t}$ be the message, and $t = |M|$.
    \item Let $\mathcal{l}$ be the length of the block, such that $\mathcal{l} \nmid t \land t > \mathcal{l}$.
    \item $M$ is splited in blocks of length $\mathcal{l}$, and the remaining part is filled with garbage. This concatenation is called $M'$.
    \item $M'$ is sent to the receiver.
    \item If $M'$ is received in the correct order, the  message is received correctly.
\end{itemize}

\begin{figure}[h]
    \centering
    \includegraphics[width=0.75\textwidth]{img/ECB.png}
\end{figure}

\subsection{Cipher Block Chaining mode (CBC)}
This method introduces a XOR encryption to the communication. \newline
Consider what follows:
\begin{itemize}
    \item Let $\Sigma = \{0,1\}, M, C \in \{0,1\}^{\mathcal{l}}$
    \item Let $\oplus$ be the XOR operator (as known as the sum in $\mathbb{Z}_{2}$).
    \item Let $P \in \{0,1\}^{\mathcal{l}}$ the initial fixed plaintext, pre-agreed.
    \item Assume that $|M| = k \cdot \mathcal{l}$.
    \item Alice sends the ciphered message $C = [c_{0}, c_{1}, \dots, c_{k}]$ to Bob. That is:
    \begin{align*}
        \begin{cases}
          c_{0} & = \mathcal{f}(P)\\
          c_{j} & = \mathcal{f}(c_{j-1} \oplus m_{j}) \text{, for } 1 \leq j \leq k
        \end{cases}
    \end{align*}
    \item In order to receive correctly the message, it's important that Bob receives correctly $c_{0}$ and computes $\mathcal{f}^{-1}(c_{0})$. If that matches the original $P$, then he can proceed by computing the other blocks:
    \[m_{j} = c_{j-1} \oplus \mathcal{f}^{-1}(c_{j}), 1 \leq j \leq k\]
    This works because:
    \begin{align*}
        c_{1} = \mathcal{f}(c_{0} \oplus m_{i}) & \land c_{0} = \mathcal{f}(c_{0}) \\
        c_{0} \oplus \mathcal{f}^{-1}(c_{1}) & = c_{0} \oplus \mathcal{f}^{-1}(\mathcal{f}(c_{0 \oplus m_{1}})) \\
        & = c_{0} \oplus c_{0} \oplus m_{1} = m_{1}
    \end{align*}
\end{itemize}

\begin{figure}[h]
    \centering
    \includegraphics[width=0.75\textwidth]{img/CBC.png}
\end{figure}


\subsection{Cipher Feedback mode (CFB)}
Let the common knowledge of the communication be:
\begin{itemize}
    \item $P$ an initial value;
    \item $\mathcal{l}$, the length of the block.
    \item $1 \leq r \leq \mathcal{l}$ be a number.
\end{itemize}
Then, let $M = [m_{1}, \dots, m_{t}]$ be the cleartext message, where:
\[|m_{i}| = r \implies r | |M| \]
The communication is executed as follows:
\RestyleAlgo{ruled}
\begin{algorithm}
\caption{Cipher FeedBack Mode communication (CFB) [Sender]}\label{alg:CFB_sender}
$I_{1} \gets P \in \{0,1\}^{\mathcal{l}}$\;
\For{$j \in 1, \dots, t$}{
    $O_{j} \gets \mathcal{f}(I_{j})$\;
    $u_{j} \gets O_{j} \bmod 2^{r}$\;
    $c_{j} \gets m_{j} \oplus u_{j}$\;
    $I_{j+1} \gets 2^{r} I_{j} + c_{j} \bmod 2^{\mathcal{l}}$\;
}
\Return{$C$}
\end{algorithm}

\begin{figure}[h]
    \centering
    \includegraphics[width=0.75\textwidth]{img/CFB.png}
\end{figure}

\RestyleAlgo{ruled}
\begin{algorithm}
\caption{Cipher FeedBack Mode communication (CFB) [Receiver]}\label{alg:CFB_receiver}
$I_{1} \gets P $\;
\For{$j \in 1, \dots, t$}{
    $O_{j} \gets \mathcal{f}(I_{j})$\;
    $u_{j} \gets O_{j} \bmod 2^{r}$\;
    $m_{j} \gets c_{j} \oplus u_{j}$\;
    $I_{j+1} \gets 2^{r} I_{j} + c_{j} \bmod 2^{\mathcal{l}}$\;
}
\Return{$C$}
\end{algorithm}


\subsection{Output Feedback mode (OFB)}
Each output feedback block cipher operation depends on all previous ones, and so cannot be performed in parallel. However, because the plaintext or ciphertext is only used for the final XOR, the block cipher operations may be performed in advance, allowing the final step to be performed in parallel once the plaintext or ciphertext is available.
\begin{figure}[h]
    \centering
    \includegraphics[width=0.75\textwidth]{img/OFB.png}
\end{figure}

\section{Feistel's Ciphers}
The Feistel's cipher cryptosystem is the predecessor of the DES. It is defined as follows:
\begin{align*}
    \Sigma = \{0,1\} \\
    \mathcal{M} = \mathcal{C} = \mathcal{K} = \Sigma^{\mathcal{l}}\\
    \mathcal{f}_{k}: \Sigma^{\mathcal{l}} \rightarrow \Sigma^{\mathcal{l}}\\
    \mathcal{F}_{k}: \Sigma^{2\mathcal{l}} \rightarrow \Sigma^{2\mathcal{l}}
\end{align*}
This cipher loops the enciphering function $F$ $r$ times, in which the blocks have a lenght of $2 \cdot \mathcal{l}$. $f$ is a sort of internal enciphering function. \newline
There's also a key generating function, that has the following signature:
\[\mathcal{K} \rightarrow \mathcal{K}^{r}\]
The algorithm enciphers as follows:
\begin{enumerate}
    \item The cleartext message $M$ is composed of two parts: $L_{0}, R_{0}$, where $|L_{0}| = |R_{0}| = \mathcal{l}$.
    \item Consider that $K_{i}$ is the $i$-th key produced.
    \item For every $i: 1 \leq i \leq r$, $C_{i}$ is computed:
    \[C_{i} = [L_{i}, R_{i}] = [R_{i-1}, L_{i-1} \oplus f_{K_{i}}(R_{i-1})\]
    \item The message that is sent is $F_{K_{r}} = [R_{r}, L_{r}]$
\end{enumerate}
The deciphering method proceeds as follows:
\begin{enumerate}
    \item Note that the keys must be used in reverse order.
    \item At each step, it is computed:
    \[R_{i} \gets L_{i-1} \oplus f_{K_{i}}(R_{i-1}), L_{i} \gets R_{i-1}\]
    \item Also, note that $f_{K_{i}} = f^{-1}_{K_{i}}$, since it's used with the XOR operator.
\end{enumerate}
An important remark is that the complexity of this algorithm depends on $f_{K}$

\section{Data Encryption Standard (DES)}
\subsection{Single DES}
The DES cryptosystem works as a Feistel's Cipher, where:
\begin{itemize}
    \item $\mathcal{l} = 32$ bits;
    \item $r = 16$ rounds;
\end{itemize}
Let now:
\begin{itemize}
    \item $\pi(b_{1}, \dots, b_{64}) = (b_{58}, b_{50}, b_{42}, \dots, b_{23}, b_{15}, b_{7})$ be a permutation;
    \item $E$ be an expansion;
    \item $S$ be a substitution;
    \item $P$ be a round permutation.
    \item Also, the set of the keys is defined as follows:
    \[\mathcal{K} = \{(b_{0}, b_{1}, \dots, b_{64}) \in \{0,1\}^{64}: \forall j \in \{0, 1, \dots,7\}: \sum_{i=1}^{8} b_{8j + i} \equiv_{2} 1\}\]
\end{itemize}

\RestyleAlgo{ruled}
\begin{algorithm}
\caption{Data Encryption Standard [Encryption]}\label{alg:DES_encrypt}
$M \rightarrow \pi(M) \in \{0,1\}^{64}$\;
\For{$1 \leq i \leq 16$}{
    $[L_{i}, R_{i}] \gets [R_{i-1}, L_{i-1} \oplus P(S(E(R_{i-1} \oplus k_{i})))]$\;
}
$C \gets \pi^{-1}[R_{16}, L_{16}]$\;
\Return{$C$}
\end{algorithm}
The keys are produced according to the following procedure:
\begin{itemize}
    \item To obtain $k_{i} \in \{0,1\}^{48}$ we have to remove the parity bits from the positions $8, 16, 24, 32, 40, 48, 56, 64$.
    \item Then, $k_{0} \gets \hat{k}$ (Then, $\hat{k} \in \{0,1\}^{56}$).
    \item $k_{i}$ is generated from $k_{i-1} = [left_part | right_part]$, in which both $left_part, right_part$ are circular shifted of 1 or 2 bits. Then, 48 bits out of the 56 are chosen. That is why $K_{i}$ is called the \emph{Rotated Key}.
\end{itemize}

The expansion function $E$ is defined as follows:
\[E: \{0,1\}^{32} \rightarrow \{0,1\}^{48}\]
\begin{itemize}
    \item Start with 6 bits and write them:
    \[b_{32}, b_{1}, b_{2}, b_{3}, b_{4}, b_{5};\]
    \item Come back of \textbf{two positions} and write the next line:
    \[b_{4}, b_{5}, b_{6}, b_{7}, b_{8}, b_{9};\]
    \item Repeat until there are no bits, and align.
\end{itemize}

The $S$ function, also called \emph{S-box}, defined as
\[S: \{0,1\}^{48} \rightarrow \{0,1\}^{32}\]
works as follows:
\begin{itemize}
    \item The function collects the output from 8 S-boxes:
    \[S_{j}; \{0,1\}^{6} \rightarrow \{0,1\}^{4}\]
    \item Let $T_{j}$ be a $4 \times 16$ matrix. Then, each cell $T_{a,b}$ can be represented as a couple of addresses with respectively 2 and 4 bits.
    \item Each row of $T$ has then a fixed permutation of $\mathbb{Z}_{16}$.
    \item Given the input $b$, $S_{j}$ returns the cell which row is at the address $(b_{1}b_{6}, b_{2}b_{3}b_{4}b_{5})$.
\end{itemize}
This is the main point of strength of the method, but if the S-boxes are not conserved properly, then the protocol is not safe anymore.

\RestyleAlgo{ruled}
\begin{algorithm}
\caption{Data Encryption Standard [Decryption]}\label{alg:DES_decrypt}
The keys are used in the inverse order $k_{16}, k_{15}, \dots, k_{1}$\;
\For{$1 \leq i \leq 16$}{
    $[R_{i-1}, L_{i-1}] \gets [L_{i}, R_{i} \oplus P(S(E(L_{i} \oplus k_{i})))]$\;
}
$M \gets [L_{1}, R_{1}]$\;
\Return{$M$}
\end{algorithm}

\subsection{Triple DES}
The Triple DES protocol uses three level of encryption, by adopting two different keys $k_{1} \neq k_{2}$.\newline
Let $E_{k_{i}}$ be the enciphering function of the Single DES, and let $D_{k_{i}}$ be the correspondent deciphering function. Then, the Triple DES enciphering function works as follows (EDE scheme):
\[m \rightarrow m_{1} = E_{k_{1}}(m) \rightarrow m_{2} = D_{k_{2}}(m_{1}) \rightarrow c = E_{k_{1}}(m_{2})\]
The deciphering function works analoguely (DED scheme):
\[c \rightarrow c_{1} = D_{k_{1}}(c) \rightarrow c_{2} = E_{k_{2}}(c_{1}) \rightarrow m = D_{k_{1}}(c_{2})\]
This means that by using $k_{1}, k_{2}$ we are using 112 bits for the key. \newline
Remark that Triple DES can be used as the Single DES, by picking $k_{1} = k_{2} = k$.

\section{Advanced Encryption Standard (AES)}
\subsection{Cryptosystem description}
The AES cryptosystem is defined as follows:
\begin{align*}
    \Sigma = \{0,1\} \\
    \mathcal{M} = \mathcal{C} = \Sigma^{128}\\
    \mathcal{K} =
    \begin{cases}
        \Sigma^{128} \text{ with AES128}\\
        \Sigma^{192} \text{ with AES192}\\
        \Sigma^{256} \text{ with AES256}\\
    \end{cases}\\
    r =
    \begin{cases}
        10 \text{ with AES128}\\
        12 \text{ with AES192}\\
        14 \text{ with AES256}\\
    \end{cases}
\end{align*}
The following auxiliary functions are defined:
\begin{itemize}
    \item $E$, the expansion function;
    \item $S$, the substitution function;
    \item $SR$, the row-shifting function;
    \item $MC$, the column-mixing function.
\end{itemize}

\subsection{AES arithmetic}
This cryptosystem uses the $\mathbb{F}_{256}$ arithmetic, that is a finite field: \[\mathbb{F}_{256} = \frac{\mathbb{F}_{2}[x]}{x^{8} + x^{4} + x^{3} + x + 1}\]
This means that each value is transformed in $\bmod (x^{8} + x^{4} + x^{3} + x + 1)$. This allows to implement some operation in a very efficient way, from the computational point of view and also from an hardware implementation point of view.\newline
Let's consider the polynomial $x^{8}$: if we transform it in the $\mathbb{F}_{256}$ field, we have that:
\[x^{8} \equiv x^{4} + x^{3} + x + 1\]
Also, we can consider just the coefficient of this result:
\[x^{8} \equiv 00011011_{2} = 1B_{16}\]
This result proves to be useful when we try to compute $\alpha \cdot x$, with $\alpha \in \mathbb{F}_{256}$:
\begin{itemize}
    \item Consider that $\alpha = b_{0} + b_{1} x + b_{2} x^{2} + \dots + b_{7} x^{7}$;
    \item Then, $\alpha \cdot x = b_{0} x + b_{1} x^{2} + b_{2} x^{3} + \dots + b_{7} x^{8}$;
    \item If we consider now the bit representation of $\alpha$ and $x$, we can observe that $\alpha = b_{0}b_{1}b_{2}\dots b_{7}, x = 00000010_{2}$, and therefore $\alpha \cdot x = (\alpha << 1) \oplus (1B)_{16}$.
    \item Also, we can easily compute the successive powers of $\alpha \cdot x^{i}$ by iterating that operation:
    \begin{itemize}
        \item $\alpha \cdot x^{2} = \alpha \cdot x \cdot x$. So, let $\beta = \alpha \cdot x \iff \alpha \cdot x^{2} = \beta \cdot x = (\beta << 1) \oplus (1B)_{16}$
    \end{itemize}
\end{itemize}

\subsection{Enciphering and Deciphering functions}
\RestyleAlgo{ruled}
\begin{algorithm}
\caption{Advanced Encryption Standard [Encryption]}\label{alg:AES_encrypt}
$(K_{0}, K_{1}, \dots, K_{10}) \gets E(k)$\;
$s \gets m \oplus K_{0}$\;
\For{$r = 1, \dots, 10$}{
    $s \gets S(s)$\;
    $s \gets SR(s)$\;
    \If{$r \leq 9$}{
        $s \gets MC(s)$\;
    }
    $s \gets s \oplus K_{r}$\;
}
\Return{$s$}
\end{algorithm}

In the decryption algorithm of AES, the round keys, and the operations as well, are used in the inverse order.
\RestyleAlgo{ruled}
\begin{algorithm}
\caption{Advanced Encryption Standard [DEcryption]}\label{alg:AES_decrypt}
$s \gets c \oplus K_{10}$\;
\For{$r = 10, \dots, 1$}{
    \If{$r > 9$}{
        $s \gets MC^{-1}(s)$\;
    }
    $s \gets SR^{-1}(s)$\;
    $s \gets S^{-1}(s)$\;
    $s \gets s \oplus K_{r}$\;
}
\Return{$s$}
\end{algorithm}

\subsection{Auxiliary functions}
\subsubsection{$E$, the expansion function}
This function serves the purpose of creating the round keys.\newline
This function uses \emph{round constants}, called $C_{i}$. These constants are composed as follows:
\[
C_{i} = [x^{i-1} | (00)_{16}| (00)_{16}| (00)_{16}]
\]
Where $x^{i-1} \in \frac{\mathbb{F}[x]}{x^{8} + x^{4} + x^{3} + 1}$. The first byte of each key contains the binary representation of the monomial $x^{i-1}$ in the aforementioned field.\newline
Consider now that the input of this function is the key $k$, that is composed of 4 words (recall that each word is 4 bytes). The function $\Pi_{S}^{'}$ applies the function $\Pi_{S}$ to each word of the input, separately. That is:
\[
\Pi_{S}^{'}(k[0],k[1],k[2],k[3]) \rightarrow (\Pi_{S}(k[0]),\Pi_{S}(k[1]),\Pi_{S}(k[2]),\Pi_{S}(k[3]))
\]
\RestyleAlgo{ruled}
\begin{algorithm}
\caption{AES expansion function ($E$)}\label{alg:AES_expansion}
$k_{0} \gets k$\;
\For{$j = 1, \dots, 10$}{
    $k_{i}[0] \gets k_{j-1}[0] \oplus C_{j} \oplus \Pi_{S}^{'}(k_{j-1}[3] << 8 )$\;
    \For{$i = 1, \dots, 3$}{
        $k_{j}[i] \gets k_{j-1}[i] \oplus k_{j}[i-1]$\;
    }
}
\Return{$(k_{0}, k_{1}, \dots, k_{10})$}
\end{algorithm}

\subsubsection{$S$, the substitution function}
This function serves a function similar to the S-boxes of the DES algorithm.\newline
Consider what follows:
\begin{itemize}
    \item Let $\operatorname{inv}: \mathbb{F}_{256}^{*} \rightarrow \mathbb{F}_{256}^{*}$ be the function that returns the inverse of the input in $\mathbb{F}_{256}^{*}$. Remark that $\operatorname{inv} = \operatorname{inv}^{-1}$, also that $\operatorname{inv}(0) = 0$ by definition.
    \item Let $\sigma: \mathbb{F}_{256} \rightarrow \mathbb{F}_{256}$ be an \textbf{affine transformation}: $\sigma(b_{0}b_{1}\dots b_{7}) = A \cdot (b_{0}b_{1}\dots b_{7}) + V$, where $A$ is a fixed matrix and $V$ is a fixed vector.
    \item Let $\Pi_{S} = \sigma \circ \operatorname{inv}: \mathbb{F}_{256} \rightarrow \mathbb{F}_{256}$. Since this is a function over a finite set, it's possible to implement it with a $16 \times 16$ matrix, where each entry's address is represented by a tuple $b_{0}b_{1}b_{2}b_{3}, b_{4}b_{5}b_{6}b_{7}$.
    \item Then $S: \{0, 1\}^{128} \rightarrow \{0, 1\}^{128}$ is defined as $S(\alpha_{0}, \alpha_{1}, \dots, \alpha_{15}) = (\Pi_{S}(\alpha_{0}), \Pi_{S}(\alpha_{1}), \dots, \Pi_{S}(\alpha_{15}))$, where $\alpha_{i}$ is the $i$-th byte of the input.
\end{itemize}
\subsubsection{$SR$, the row-shifting function}
This function serves the purpose of mixing the rows' content.\newline
Let: \[S = \begin{pmatrix}
s_{0} & s_{4} & s_{8} & s_{12} \\
s_{1} & s_{5} & s_{9} & s_{13} \\
s_{2} & s_{6} & s_{10} & s_{14} \\
s_{3} & s_{7} & s_{11} & s_{15}
\end{pmatrix}\]
The $SR$ function works as follows:
\begin{itemize}
    \item The first row is untouched.
    \item The second row is \textbf{round-shifted 1 byte to the left}.
    \item The third row is \textbf{round-shifted 2 bytes to the left}.
    \item The fourth row is \textbf{round-shifted 3 bytes to the left}.
\end{itemize}
Then: \[SR(S) = \begin{pmatrix}
s_{0} & s_{4} & s_{8} & s_{12} \\
s_{13} & s_{1} & s_{5} & s_{9} \\
s_{10} & s_{14} & s_{2} & s_{6} \\
s_{7} & s_{11} & s_{15} & s_{3}
\end{pmatrix}\]

\subsubsection{$MC$, the column-mixing function}
This function serves the purpose of mixing the columns' content.\newline
$MC$ is defined as $MC(\alpha_{0}, \alpha_{1}, \alpha_{2}, \alpha_{3}) = M \cdot (\alpha_{0}, \alpha_{1}, \alpha_{2}, \alpha_{3})$, where $M$ is a fixed matrix and $\alpha_{i}$ is the $i$-th word of the input.
Consider that if $M$ is invertible, then also $MC$ is invertible.


\chapter{Key Exchange problem}
\section{Definition}
Given:
\begin{itemize}
    \item $\mathcal{l}$, the length of the block;
    \item $\Sigma$, the alphabet;
\end{itemize}
A block cipher is a cryptosystem such that:
\[M = \mathcal{C} = \Sigma^{\mathcal{l}}\]
\begin{proposition}
    Consider waht follows:\newline
    The enciphering function of a block cypher is a permutation of $\Sigma^{\mathcal{l}}$.
\end{proposition}
\begin{proof}
    The proof proceeds as follows:
    \begin{itemize}
        \item Let $\mathcal{f}: \Sigma^{\mathcal{l}} \rightarrow \Sigma^{\mathcal{l}}$.
        \item Let $\mathcal{f}(m) = \mathcal{f}([p_{1}, p_{2}, \dots, p_{e}]) = \Pi(p_{1}, p_{2}, \dots, p_{e})$, where $p_{i} \in \Sigma$.
        \item Then, let $\mathcal{f}^{-1}(c) = \mathcal{f}([c_{1}, c_{2}, \dots, c_{e}]) = \Pi(c_{1}, c_{2}, \dots, c_{e})$, where $c_{i} \in \Sigma$.
        \item Let then $K_{E} = \Pi$ be the set of encyphering keys;
        \item Let then $K_{D} = \Pi^{-1}$ be the set of decyphering keys;
        \item Then, $|K| = (|\Sigma|^{\mathcal{l}})!$
        \item Due to the Pigeonhole principle, if $\mathcal{f}$ is injective, then it's also surjective, and therefore a bijection.
    \end{itemize}
\end{proof}

\section{Enciphering methods}
Since block cyphers encrypt just message of length $\mathcal{l}$, it's necessary to define the different ways in which the blocks can be used in order to compute the ciphertext.
\subsection{Electronic Code Book mode (ECB)}
Consider what follows:
\begin{itemize}
    \item Let $M \in \Sigma^{t}$ be the message, and $t = |M|$.
    \item Let $\mathcal{l}$ be the length of the block, such that $\mathcal{l} \nmid t \land t > \mathcal{l}$.
    \item $M$ is splited in blocks of length $\mathcal{l}$, and the remaining part is filled with garbage. This concatenation is called $M'$.
    \item $M'$ is sent to the receiver.
    \item If $M'$ is received in the correct order, the  message is received correctly.
\end{itemize}

\begin{figure}[h]
    \centering
    \includegraphics[width=0.75\textwidth]{img/ECB.png}
\end{figure}

\subsection{Cipher Block Chaining mode (CBC)}
This method introduces a XOR encryption to the communication. \newline
Consider what follows:
\begin{itemize}
    \item Let $\Sigma = \{0,1\}, M, C \in \{0,1\}^{\mathcal{l}}$
    \item Let $\oplus$ be the XOR operator (as known as the sum in $\mathbb{Z}_{2}$).
    \item Let $P \in \{0,1\}^{\mathcal{l}}$ the initial fixed plaintext, pre-agreed.
    \item Assume that $|M| = k \cdot \mathcal{l}$.
    \item Alice sends the ciphered message $C = [c_{0}, c_{1}, \dots, c_{k}]$ to Bob. That is:
    \begin{align*}
        \begin{cases}
          c_{0} & = \mathcal{f}(P)\\
          c_{j} & = \mathcal{f}(c_{j-1} \oplus m_{j}) \text{, for } 1 \leq j \leq k
        \end{cases}
    \end{align*}
    \item In order to receive correctly the message, it's important that Bob receives correctly $c_{0}$ and computes $\mathcal{f}^{-1}(c_{0})$. If that matches the original $P$, then he can proceed by computing the other blocks:
    \[m_{j} = c_{j-1} \oplus \mathcal{f}^{-1}(c_{j}), 1 \leq j \leq k\]
    This works because:
    \begin{align*}
        c_{1} = \mathcal{f}(c_{0} \oplus m_{i}) & \land c_{0} = \mathcal{f}(c_{0}) \\
        c_{0} \oplus \mathcal{f}^{-1}(c_{1}) & = c_{0} \oplus \mathcal{f}^{-1}(\mathcal{f}(c_{0 \oplus m_{1}})) \\
        & = c_{0} \oplus c_{0} \oplus m_{1} = m_{1}
    \end{align*}
\end{itemize}

\begin{figure}[h]
    \centering
    \includegraphics[width=0.75\textwidth]{img/CBC.png}
\end{figure}


\subsection{Cipher Feedback mode (CFB)}
Let the common knowledge of the communication be:
\begin{itemize}
    \item $P$ an initial value;
    \item $\mathcal{l}$, the length of the block.
    \item $1 \leq r \leq \mathcal{l}$ be a number.
\end{itemize}
Then, let $M = [m_{1}, \dots, m_{t}]$ be the cleartext message, where:
\[|m_{i}| = r \implies r | |M| \]
The communication is executed as follows:
\RestyleAlgo{ruled}
\begin{algorithm}
\caption{Cipher FeedBack Mode communication (CFB) [Sender]}\label{alg:CFB_sender}
$I_{1} \gets P \in \{0,1\}^{\mathcal{l}}$\;
\For{$j \in 1, \dots, t$}{
    $O_{j} \gets \mathcal{f}(I_{j})$\;
    $u_{j} \gets O_{j} \bmod 2^{r}$\;
    $c_{j} \gets m_{j} \oplus u_{j}$\;
    $I_{j+1} \gets 2^{r} I_{j} + c_{j} \bmod 2^{\mathcal{l}}$\;
}
\Return{$C$}
\end{algorithm}

\begin{figure}[h]
    \centering
    \includegraphics[width=0.75\textwidth]{img/CFB.png}
\end{figure}

\RestyleAlgo{ruled}
\begin{algorithm}
\caption{Cipher FeedBack Mode communication (CFB) [Receiver]}\label{alg:CFB_receiver}
$I_{1} \gets P $\;
\For{$j \in 1, \dots, t$}{
    $O_{j} \gets \mathcal{f}(I_{j})$\;
    $u_{j} \gets O_{j} \bmod 2^{r}$\;
    $m_{j} \gets c_{j} \oplus u_{j}$\;
    $I_{j+1} \gets 2^{r} I_{j} + c_{j} \bmod 2^{\mathcal{l}}$\;
}
\Return{$C$}
\end{algorithm}


\subsection{Output Feedback mode (OFB)}
Each output feedback block cipher operation depends on all previous ones, and so cannot be performed in parallel. However, because the plaintext or ciphertext is only used for the final XOR, the block cipher operations may be performed in advance, allowing the final step to be performed in parallel once the plaintext or ciphertext is available.
\begin{figure}[h]
    \centering
    \includegraphics[width=0.75\textwidth]{img/OFB.png}
\end{figure}

\section{Feistel's Ciphers}
The Feistel's cipher cryptosystem is the predecessor of the DES. It is defined as follows:
\begin{align*}
    \Sigma = \{0,1\} \\
    \mathcal{M} = \mathcal{C} = \mathcal{K} = \Sigma^{\mathcal{l}}\\
    \mathcal{f}_{k}: \Sigma^{\mathcal{l}} \rightarrow \Sigma^{\mathcal{l}}\\
    \mathcal{F}_{k}: \Sigma^{2\mathcal{l}} \rightarrow \Sigma^{2\mathcal{l}}
\end{align*}
This cipher loops the enciphering function $F$ $r$ times, in which the blocks have a lenght of $2 \cdot \mathcal{l}$. $f$ is a sort of internal enciphering function. \newline
There's also a key generating function, that has the following signature:
\[\mathcal{K} \rightarrow \mathcal{K}^{r}\]
The algorithm enciphers as follows:
\begin{enumerate}
    \item The cleartext message $M$ is composed of two parts: $L_{0}, R_{0}$, where $|L_{0}| = |R_{0}| = \mathcal{l}$.
    \item Consider that $K_{i}$ is the $i$-th key produced.
    \item For every $i: 1 \leq i \leq r$, $C_{i}$ is computed:
    \[C_{i} = [L_{i}, R_{i}] = [R_{i-1}, L_{i-1} \oplus f_{K_{i}}(R_{i-1})\]
    \item The message that is sent is $F_{K_{r}} = [R_{r}, L_{r}]$
\end{enumerate}
The deciphering method proceeds as follows:
\begin{enumerate}
    \item Note that the keys must be used in reverse order.
    \item At each step, it is computed:
    \[R_{i} \gets L_{i-1} \oplus f_{K_{i}}(R_{i-1}), L_{i} \gets R_{i-1}\]
    \item Also, note that $f_{K_{i}} = f^{-1}_{K_{i}}$, since it's used with the XOR operator.
\end{enumerate}
An important remark is that the complexity of this algorithm depends on $f_{K}$

\section{Data Encryption Standard (DES)}
\subsection{Single DES}
The DES cryptosystem works as a Feistel's Cipher, where:
\begin{itemize}
    \item $\mathcal{l} = 32$ bits;
    \item $r = 16$ rounds;
\end{itemize}
Let now:
\begin{itemize}
    \item $\pi(b_{1}, \dots, b_{64}) = (b_{58}, b_{50}, b_{42}, \dots, b_{23}, b_{15}, b_{7})$ be a permutation;
    \item $E$ be an expansion;
    \item $S$ be a substitution;
    \item $P$ be a round permutation.
    \item Also, the set of the keys is defined as follows:
    \[\mathcal{K} = \{(b_{0}, b_{1}, \dots, b_{64}) \in \{0,1\}^{64}: \forall j \in \{0, 1, \dots,7\}: \sum_{i=1}^{8} b_{8j + i} \equiv_{2} 1\}\]
\end{itemize}

\RestyleAlgo{ruled}
\begin{algorithm}
\caption{Data Encryption Standard [Encryption]}\label{alg:DES_encrypt}
$M \rightarrow \pi(M) \in \{0,1\}^{64}$\;
\For{$1 \leq i \leq 16$}{
    $[L_{i}, R_{i}] \gets [R_{i-1}, L_{i-1} \oplus P(S(E(R_{i-1} \oplus k_{i})))]$\;
}
$C \gets \pi^{-1}[R_{16}, L_{16}]$\;
\Return{$C$}
\end{algorithm}
The keys are produced according to the following procedure:
\begin{itemize}
    \item To obtain $k_{i} \in \{0,1\}^{48}$ we have to remove the parity bits from the positions $8, 16, 24, 32, 40, 48, 56, 64$.
    \item Then, $k_{0} \gets \hat{k}$ (Then, $\hat{k} \in \{0,1\}^{56}$).
    \item $k_{i}$ is generated from $k_{i-1} = [left_part | right_part]$, in which both $left_part, right_part$ are circular shifted of 1 or 2 bits. Then, 48 bits out of the 56 are chosen. That is why $K_{i}$ is called the \emph{Rotated Key}.
\end{itemize}

The expansion function $E$ is defined as follows:
\[E: \{0,1\}^{32} \rightarrow \{0,1\}^{48}\]
\begin{itemize}
    \item Start with 6 bits and write them:
    \[b_{32}, b_{1}, b_{2}, b_{3}, b_{4}, b_{5};\]
    \item Come back of \textbf{two positions} and write the next line:
    \[b_{4}, b_{5}, b_{6}, b_{7}, b_{8}, b_{9};\]
    \item Repeat until there are no bits, and align.
\end{itemize}

The $S$ function, also called \emph{S-box}, defined as
\[S: \{0,1\}^{48} \rightarrow \{0,1\}^{32}\]
works as follows:
\begin{itemize}
    \item The function collects the output from 8 S-boxes:
    \[S_{j}; \{0,1\}^{6} \rightarrow \{0,1\}^{4}\]
    \item Let $T_{j}$ be a $4 \times 16$ matrix. Then, each cell $T_{a,b}$ can be represented as a couple of addresses with respectively 2 and 4 bits.
    \item Each row of $T$ has then a fixed permutation of $\mathbb{Z}_{16}$.
    \item Given the input $b$, $S_{j}$ returns the cell which row is at the address $(b_{1}b_{6}, b_{2}b_{3}b_{4}b_{5})$.
\end{itemize}
This is the main point of strength of the method, but if the S-boxes are not conserved properly, then the protocol is not safe anymore.

\RestyleAlgo{ruled}
\begin{algorithm}
\caption{Data Encryption Standard [Decryption]}\label{alg:DES_decrypt}
The keys are used in the inverse order $k_{16}, k_{15}, \dots, k_{1}$\;
\For{$1 \leq i \leq 16$}{
    $[R_{i-1}, L_{i-1}] \gets [L_{i}, R_{i} \oplus P(S(E(L_{i} \oplus k_{i})))]$\;
}
$M \gets [L_{1}, R_{1}]$\;
\Return{$M$}
\end{algorithm}

\subsection{Triple DES}
The Triple DES protocol uses three level of encryption, by adopting two different keys $k_{1} \neq k_{2}$.\newline
Let $E_{k_{i}}$ be the enciphering function of the Single DES, and let $D_{k_{i}}$ be the correspondent deciphering function. Then, the Triple DES enciphering function works as follows (EDE scheme):
\[m \rightarrow m_{1} = E_{k_{1}}(m) \rightarrow m_{2} = D_{k_{2}}(m_{1}) \rightarrow c = E_{k_{1}}(m_{2})\]
The deciphering function works analoguely (DED scheme):
\[c \rightarrow c_{1} = D_{k_{1}}(c) \rightarrow c_{2} = E_{k_{2}}(c_{1}) \rightarrow m = D_{k_{1}}(c_{2})\]
This means that by using $k_{1}, k_{2}$ we are using 112 bits for the key. \newline
Remark that Triple DES can be used as the Single DES, by picking $k_{1} = k_{2} = k$.

\section{Advanced Encryption Standard (AES)}
\subsection{Cryptosystem description}
The AES cryptosystem is defined as follows:
\begin{align*}
    \Sigma = \{0,1\} \\
    \mathcal{M} = \mathcal{C} = \Sigma^{128}\\
    \mathcal{K} =
    \begin{cases}
        \Sigma^{128} \text{ with AES128}\\
        \Sigma^{192} \text{ with AES192}\\
        \Sigma^{256} \text{ with AES256}\\
    \end{cases}\\
    r =
    \begin{cases}
        10 \text{ with AES128}\\
        12 \text{ with AES192}\\
        14 \text{ with AES256}\\
    \end{cases}
\end{align*}
The following auxiliary functions are defined:
\begin{itemize}
    \item $E$, the expansion function;
    \item $S$, the substitution function;
    \item $SR$, the row-shifting function;
    \item $MC$, the column-mixing function.
\end{itemize}

\subsection{AES arithmetic}
This cryptosystem uses the $\mathbb{F}_{256}$ arithmetic, that is a finite field: \[\mathbb{F}_{256} = \frac{\mathbb{F}_{2}[x]}{x^{8} + x^{4} + x^{3} + x + 1}\]
This means that each value is transformed in $\bmod (x^{8} + x^{4} + x^{3} + x + 1)$. This allows to implement some operation in a very efficient way, from the computational point of view and also from an hardware implementation point of view.\newline
Let's consider the polynomial $x^{8}$: if we transform it in the $\mathbb{F}_{256}$ field, we have that:
\[x^{8} \equiv x^{4} + x^{3} + x + 1\]
Also, we can consider just the coefficient of this result:
\[x^{8} \equiv 00011011_{2} = 1B_{16}\]
This result proves to be useful when we try to compute $\alpha \cdot x$, with $\alpha \in \mathbb{F}_{256}$:
\begin{itemize}
    \item Consider that $\alpha = b_{0} + b_{1} x + b_{2} x^{2} + \dots + b_{7} x^{7}$;
    \item Then, $\alpha \cdot x = b_{0} x + b_{1} x^{2} + b_{2} x^{3} + \dots + b_{7} x^{8}$;
    \item If we consider now the bit representation of $\alpha$ and $x$, we can observe that $\alpha = b_{0}b_{1}b_{2}\dots b_{7}, x = 00000010_{2}$, and therefore $\alpha \cdot x = (\alpha << 1) \oplus (1B)_{16}$.
    \item Also, we can easily compute the successive powers of $\alpha \cdot x^{i}$ by iterating that operation:
    \begin{itemize}
        \item $\alpha \cdot x^{2} = \alpha \cdot x \cdot x$. So, let $\beta = \alpha \cdot x \iff \alpha \cdot x^{2} = \beta \cdot x = (\beta << 1) \oplus (1B)_{16}$
    \end{itemize}
\end{itemize}

\subsection{Enciphering and Deciphering functions}
\RestyleAlgo{ruled}
\begin{algorithm}
\caption{Advanced Encryption Standard [Encryption]}\label{alg:AES_encrypt}
$(K_{0}, K_{1}, \dots, K_{10}) \gets E(k)$\;
$s \gets m \oplus K_{0}$\;
\For{$r = 1, \dots, 10$}{
    $s \gets S(s)$\;
    $s \gets SR(s)$\;
    \If{$r \leq 9$}{
        $s \gets MC(s)$\;
    }
    $s \gets s \oplus K_{r}$\;
}
\Return{$s$}
\end{algorithm}

In the decryption algorithm of AES, the round keys, and the operations as well, are used in the inverse order.
\RestyleAlgo{ruled}
\begin{algorithm}
\caption{Advanced Encryption Standard [DEcryption]}\label{alg:AES_decrypt}
$s \gets c \oplus K_{10}$\;
\For{$r = 10, \dots, 1$}{
    \If{$r > 9$}{
        $s \gets MC^{-1}(s)$\;
    }
    $s \gets SR^{-1}(s)$\;
    $s \gets S^{-1}(s)$\;
    $s \gets s \oplus K_{r}$\;
}
\Return{$s$}
\end{algorithm}

\subsection{Auxiliary functions}
\subsubsection{$E$, the expansion function}
This function serves the purpose of creating the round keys.\newline
This function uses \emph{round constants}, called $C_{i}$. These constants are composed as follows:
\[
C_{i} = [x^{i-1} | (00)_{16}| (00)_{16}| (00)_{16}]
\]
Where $x^{i-1} \in \frac{\mathbb{F}[x]}{x^{8} + x^{4} + x^{3} + 1}$. The first byte of each key contains the binary representation of the monomial $x^{i-1}$ in the aforementioned field.\newline
Consider now that the input of this function is the key $k$, that is composed of 4 words (recall that each word is 4 bytes). The function $\Pi_{S}^{'}$ applies the function $\Pi_{S}$ to each word of the input, separately. That is:
\[
\Pi_{S}^{'}(k[0],k[1],k[2],k[3]) \rightarrow (\Pi_{S}(k[0]),\Pi_{S}(k[1]),\Pi_{S}(k[2]),\Pi_{S}(k[3]))
\]
\RestyleAlgo{ruled}
\begin{algorithm}
\caption{AES expansion function ($E$)}\label{alg:AES_expansion}
$k_{0} \gets k$\;
\For{$j = 1, \dots, 10$}{
    $k_{i}[0] \gets k_{j-1}[0] \oplus C_{j} \oplus \Pi_{S}^{'}(k_{j-1}[3] << 8 )$\;
    \For{$i = 1, \dots, 3$}{
        $k_{j}[i] \gets k_{j-1}[i] \oplus k_{j}[i-1]$\;
    }
}
\Return{$(k_{0}, k_{1}, \dots, k_{10})$}
\end{algorithm}

\subsubsection{$S$, the substitution function}
This function serves a function similar to the S-boxes of the DES algorithm.\newline
Consider what follows:
\begin{itemize}
    \item Let $\operatorname{inv}: \mathbb{F}_{256}^{*} \rightarrow \mathbb{F}_{256}^{*}$ be the function that returns the inverse of the input in $\mathbb{F}_{256}^{*}$. Remark that $\operatorname{inv} = \operatorname{inv}^{-1}$, also that $\operatorname{inv}(0) = 0$ by definition.
    \item Let $\sigma: \mathbb{F}_{256} \rightarrow \mathbb{F}_{256}$ be an \textbf{affine transformation}: $\sigma(b_{0}b_{1}\dots b_{7}) = A \cdot (b_{0}b_{1}\dots b_{7}) + V$, where $A$ is a fixed matrix and $V$ is a fixed vector.
    \item Let $\Pi_{S} = \sigma \circ \operatorname{inv}: \mathbb{F}_{256} \rightarrow \mathbb{F}_{256}$. Since this is a function over a finite set, it's possible to implement it with a $16 \times 16$ matrix, where each entry's address is represented by a tuple $b_{0}b_{1}b_{2}b_{3}, b_{4}b_{5}b_{6}b_{7}$.
    \item Then $S: \{0, 1\}^{128} \rightarrow \{0, 1\}^{128}$ is defined as $S(\alpha_{0}, \alpha_{1}, \dots, \alpha_{15}) = (\Pi_{S}(\alpha_{0}), \Pi_{S}(\alpha_{1}), \dots, \Pi_{S}(\alpha_{15}))$, where $\alpha_{i}$ is the $i$-th byte of the input.
\end{itemize}
\subsubsection{$SR$, the row-shifting function}
This function serves the purpose of mixing the rows' content.\newline
Let: \[S = \begin{pmatrix}
s_{0} & s_{4} & s_{8} & s_{12} \\
s_{1} & s_{5} & s_{9} & s_{13} \\
s_{2} & s_{6} & s_{10} & s_{14} \\
s_{3} & s_{7} & s_{11} & s_{15}
\end{pmatrix}\]
The $SR$ function works as follows:
\begin{itemize}
    \item The first row is untouched.
    \item The second row is \textbf{round-shifted 1 byte to the left}.
    \item The third row is \textbf{round-shifted 2 bytes to the left}.
    \item The fourth row is \textbf{round-shifted 3 bytes to the left}.
\end{itemize}
Then: \[SR(S) = \begin{pmatrix}
s_{0} & s_{4} & s_{8} & s_{12} \\
s_{13} & s_{1} & s_{5} & s_{9} \\
s_{10} & s_{14} & s_{2} & s_{6} \\
s_{7} & s_{11} & s_{15} & s_{3}
\end{pmatrix}\]

\subsubsection{$MC$, the column-mixing function}
This function serves the purpose of mixing the columns' content.\newline
$MC$ is defined as $MC(\alpha_{0}, \alpha_{1}, \alpha_{2}, \alpha_{3}) = M \cdot (\alpha_{0}, \alpha_{1}, \alpha_{2}, \alpha_{3})$, where $M$ is a fixed matrix and $\alpha_{i}$ is the $i$-th word of the input.
Consider that if $M$ is invertible, then also $MC$ is invertible.


\chapter*{Useful Facts}
\begin{itemize}
    \item The \href{https://gmplib.org/}{GMP library} is a free library for arbitrary precision arithmetic. It implements all the basic arithmetic operations with the maximum efficency possible.
\end{itemize}

\listofalgorithms
\listoftheorems[ignoreall, show={theorem}]

\begingroup               % Temporarily disable \clearpage to show both lists on one page
  \let\clearpage\relax    % http://tex.stackexchange.com/a/14511/104449
  \renewcommand{\listtheoremname}{List of Lemmas}
  \listoftheorems[ignoreall, show={lemma}]
\endgroup

\end{document}
